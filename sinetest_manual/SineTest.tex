\chapter{Frequency Response Analysis of a Single Board Heater System by the Application of Sine Wave}
The aim of this experiment is to do a frequency response analysis of a Single Board Heater System by the 
application of sine wave. The target group is anyone who has basic knowledge of control engineering.\\
\begin{figure}
\centering
\includegraphics[width=\linewidth]{sinetest_manual/sine_test.jpg}
\caption{Xcos for this experiment}
\label{xcos_sine}
\end{figure}
We have used Scilab with Xcos as an interface for sending and receiving data. 
This interface is shown in figure \ref{xcos_sine}. Heater current and fan speed are the two inputs to the system. 
The heater current is varied sinusoidally. A provision is made to set the parameters related to it like frequency, amplitude 
and offset. The temperature profile thus obtained is the output.\\
In this experiment we are applying a sine change in the heater current by keeping the fan speed constant. 
After application of sine change, wait for sufficient amount of time to allow the temperature to reach a steady-state.
\section{Theory}
 Frequency response of a system means its steady-state response to a sinusoidal input. 
 For obtaining a frequency response of a system, we vary the frequency of the input signal over a spectrum of interest. 
 The analysis is useful and simple because it can be carried out with the available signal generators and measuring devices.\\
Consider a sinusoidal input
\begin{align}
U(t) &= Asin \omega t
\intertext{The Laplace transform of the above equation yields}
U(s) &= \frac{A\omega}{s^2 + \omega^2} \label{lap_tran}
\intertext{Consider the standard first order transfer function given below}
G(s) &= \frac {Y(s)}{U(s)} = \frac K{s + 1}
\intertext{Replacing the value of U(s) from equation \ref{lap_tran}, we get}
Y(s) &= \frac{KA\omega}{(\tau s + 1)(s^2 + \omega ^2)}\\
&=\frac{KA}{\omega ^2\tau ^2 + 1}\left[\frac{\omega \tau ^2}{\tau s +1}- \frac{\tau s \omega}{s^2 + \omega^2}+\frac{\omega}
{s^2 + \omega^2}\right]
\intertext{Taking Laplace Inverse, we get}
y(t) &= \left[\frac {KA}{\omega^2\tau^2+ 1}\right]\left[\omega \tau e^{\frac {-t}{\tau}}-\omega \tau cos(\omega t)+
sin(\omega t)\right] 
\intertext{The above equation has an exponential term $e^\frac{-t}{\tau}$. Hence, for large value of time, its value will 
approach to zero and the equation will yield a pure sine wave. One can also use trigonometric identities to make the equation 
look more simple.}
y(t) &= \left[\frac{KA}{\sqrt{\omega^2 \tau^2 + 1}}\right]\left[sin (\omega t) + \phi \right]
\intertext{where,}
\phi &= -tan^{-1}(\omega \tau)
\intertext{By observing the above equation, one can easily make out that for a sinusoidal input the output is also sinusoidal
but has some phase difference. 
Also, the amplitude of the output signal, $\hat{A}$, has become a function of the input signal frequency, $\omega$.}
\hat{A}&=\frac{KA}{\sqrt{\omega^2 \tau^2 + 1}}
\intertext{The amplitude ratio (AR) can be calculated by dividing both sides by the input signal amplitude A.}
AR &=\frac{\hat{A}}{A}=\frac{K}{\sqrt{\omega^2 \tau^2 + 1}}
\intertext{Dividing the above equation by the process gain K yields the normalized amplitude ratio $(AR_n)$}
AR_n &=\frac{AR}{K}=\frac{1}{\sqrt{\omega^2 \tau^2 + 1}}
\end{align}
Because the process steady state gain is constant, the normalized amplitude ratio is often used for 
frequency response analysis \cite{dale04}.


\section{Procedure to perform Sine Test}
Follow the procedure explained in section \ref{scilab_sbhs}.
\begin{enumerate}
 \item Change the current working directory of Scilab to the folder {\tt Sine\_Test}. 
 \item Execute the code {\tt sinetest.sce} and {\tt sinetest.sci}.
 \item Open the Xcos file {\tt sine\_test.xcos}. 
 \item Initiate a sine input to the system by setting sinusoid generator block properties with some value of the frequency  
($0.007Hz$) and amplitude ($10$).
\end{enumerate}
Note that at high frequencies the plant output is not sinusoidal, which is not of any use. 
Hence, avoid choosing frequencies above $0.04Hz$.

\begin{figure}
\includegraphics[width=\linewidth]{sinetest_manual/sine_resp.png}
\caption{Plot for sine input 0.007Hz}
\label{fig:scope}
\end{figure}
\begin{table}

\begin{verbatim}
 0.100E+00  0.200E+02  0.100E+03  0.239E+02
 0.200E+00  0.201E+02  0.100E+03  0.238E+02
 0.300E+00  0.201E+02  0.100E+03  0.238E+02
.
.
.
 0.749E+03  0.300E+02  0.100E+03  0.301E+02
 0.749E+03  0.300E+02  0.100E+03  0.302E+02
 0.749E+03  0.300E+02  0.100E+03  0.302E+02
\end{verbatim}
\caption{Data obtained after application of sine input of $0.04Hz$}
\label{sine_data}
\end{table}
The sine test data file is shown in table \ref{sine_data}. Refering to table \ref{sine_data} the first column represents samples. 
The second column represents heater in percentage. Here, it is sinusoidally varied. 
The third column represents fan in percentage. Note that its value is 100 throughout the experiment. 
The fourth column represents the output temperature.
It should be taken into consideration that all the values mentioned in the data file are in percentage of maximum output,
except for the temperature which is in \textcelsius. 

\section{Sine Test Analysis}
Now let us calculate amplitude ratio and phase difference. 
\begin{enumerate}
 \item Change the current working directory of Scilab to the 
folder {\tt Sine\_Analysis}.
\item Copy the data files generated after the completion of the experiment into the 
{\tt Sine\_Analysis} folder.
\item Place the arguments {\ttfamily f} and {\ttfamily filename} in the Scilab code 
{\ttfamily sine2.sce} for the calculation of the above parameters and execute it. Here {\ttfamily f}
means input frequency.
\end{enumerate}

  
\begin{figure}
\includegraphics[width=\linewidth]{sinetest_manual/bode_calc.png}
\caption{Scilab Output}
\label{scilab_op}
\end{figure}
It could be seen from figure \ref{scilab_op} that the amplitude r
atio turns out to be $-2.047$dB and phase difference to be $-60.732$\textdegree.
The plot thus obtained is shown in figure \ref{plot0.4}
\begin{figure}
\includegraphics[width=\linewidth]{sinetest_manual/bode_resp}
\caption{Plot of Input and Output vs time}
\label{plot0.4}
\end{figure}

Repeat this calculation over a range of frequencies and note down the values of amplitude ratio in dB and phase difference. 
Input these values for the appropriate frequencies into the Scilab code {\ttfamily TFbode.sce} and execute it to get a 
Bode plot of the plant which is illustrated in figure \ref{bode_plot}.
\begin{figure}
\includegraphics[width=\linewidth]{sinetest_manual/bodeplot}
\caption{Bode plot obtained from the plant}
\label{bode_plot}
\end{figure}

Bode plot can be obtained directly from the plant's second order transfer function \cite{kmm09} with the help of Scilab code
{\ttfamily TFbode.sce}, as shown in figure \ref{tfbode}. A visual comparison of the two Bode plots can be done to 
validate the Bode diagram obtained from the plant.
\begin{figure}
\includegraphics[width=\linewidth]{sinetest_manual/plant_bode_tf}
\caption{Bode plot obtained through plant's transfer function}
\label{tfbode}
\end{figure}

To compare the two plots, we plot it on the same graph as shown in figure \ref{compare_bode}
\begin{figure}
\centering
\includegraphics[width=\linewidth]{sinetest_manual/bode_comparison}
\caption{Comparison of Bode plots}
\label{compare_bode}
\end{figure}


\section{Conducting Sine Test on SBHS, virtually}
The step by step procedure for conducting an experiment virtually is explained in section \ref{vlabsexpt}. 
\begin{enumerate}
 \item Go to {\tt virtual} folder and then {\tt SineTest} directory .
 \item Execute {\tt sinetest.sce}.
 \item Perform sine test analysis by executing the file {\tt sine2\_virtual.sce} in {\tt Sine\_Analysis} directory 
 under {\tt virtual} folder.
\end{enumerate}
The necessary codes are listed in the section \ref{sinecodes}.


\section{Scilab Code}\label{sinecodes}

\begin{code}
\ccaption{sine\_test.sci}{\ttfamily sine\_test.sci}
\lstinputlisting{Scilab/local/Sine_Test/sine_test.sci}
\end{code}


\begin{code}
\ccaption{sinetest.sce}{\ttfamily sinetest.sce}
\lstinputlisting{Scilab/virtual/SineTest/sinetest.sce}
\end{code}

\begin{code}
\ccaption{sinetest.sci}{\ttfamily sinetest.sci}
\lstinputlisting{Scilab/virtual/SineTest/sinetest.sci}
\end{code}

\begin{code}
\ccaption{sine2.sce}{\ttfamily sine2.sce}
\lstinputlisting{Scilab/local/Sine_Analysis/sine2.sce}
\end{code}

\begin{code}
\ccaption{lable.sci}{\ttfamily label.sci}
\lstinputlisting{Scilab/local/Sine_Analysis/label.sci}
\end{code}

\begin{code}
\ccaption{bodeplot.sce}{\ttfamily bodeplot.sce}
\lstinputlisting{Scilab/local/Sine_Analysis/bodeplot.sce}
\end{code}


\begin{code}
\ccaption{labelbode.sci}{\ttfamily labelbode.sci}
\lstinputlisting{Scilab/local/Sine_Analysis/labelbode.sci}
\end{code}


\begin{code}
\ccaption{TFbode.sce}{\ttfamily TFbode.sce}
\lstinputlisting{Scilab/local/Sine_Analysis/TFbode.sce}
\end{code}

\begin{code}
\ccaption{comparison.sce}{\ttfamily comparison.sce}
\lstinputlisting{Scilab/local/Sine_Analysis/comparison.sce}
\end{code}

\begin{code}
\ccaption{sine2\_virtual.sce}{\ttfamily sine2\_virtual.sce}
\lstinputlisting{Scilab/virtual/Sine_Analysis/sine2_virtual.sce}
\end{code}

