\section{Conclusions}
In this work, we presented a virtual laboratory that is based entirely
on free and open source software.  This solution is scalable as it is
affordable.  It is also in line with the objectives of the funding
agency \cite{kmm010} that aims to promote open educational resources.

The solution presented in this work allows about 100,000 hours of
experimentation time in a year with one or at most two server PCs
working with about 20 SBHS.  Thus, this configuration has the
potential to cater to the million students who enroll in engineering
colleges in India every year.

The main shortcomings of the approach followed in this work are due to
the delay introduced in the loop by the Internet.  We are at present
studying the efficacy of delay observers \cite{kempf96,jones07} to
address this issue.


%The paper explains a low cost, scalable, virtual control laboratory.  This cost effective solution can be extended to any type of setup thereby giving more hands-on experimentation to the students.  Even the colleges with the lack of infrastructure can train their students using this approach.  This web based solution has a potential to bridge the gap between theory and practice.  


