\subsection{Firmware}
Programming of ATmega16L is described next.

ATmega16L features an 8-channel, 10-bit successive approximation ADC
with 0-Vcc input voltage range. The output of instrumentation
amplifier feeds into channel-2 of ADC to obtain digital value of
temperature. The ADC channel is programmed for 208 $\mu$s of
conversion time using successive approximation method. Timer0 is used
to trigger it 50 times every second. The ADC module is used with
10-bit resolution to sense voltage in the multiples of 4.8828 mV. ADC
Conversion Complete Interrupt is used to indicate the end of
conversion.

ATmega16L can produce analog output through pulse width modulation, a
process to modulate duty cycle of a constant frequency square wave.
Timer1 of the microcontroller generates 8-bit PWM signal at 488 Hz
that is fed to heating element and fan.

% Pulse width modulation is a process to The duty cycle is defined as
% $\frac{T_{on}}{T}$, where $T_{on}$ is the ON time and $T$ is the
% time period of the wave. Power delivered to the load is proportional
% to $T_{on}$ of the signal. PWM is used to control total amount of
% power delivered to the load without power losses which generally
% occur in resistive methods of power control.

The microcontroller has been programmed to handle commands to
operate heater and fan and also to communicate digital values of
temperature read by the sensor.  The number 253 followed by a byte
containing the fan speed in PWM units regulates the fan speed.
Similarly, 254 followed by a byte indicates the heater power.  The
number 255, however, indicates that the following two bytes denote
the temperature reading, to be read by the computer.

All the above commands are sent by the computer to the microcontroller
via serial port interface through a programmable Universal
Asynchronous serial Receiver and Transmitter (UART) of ATmega16L. The
UART is configured in 8-bit mode, with baud rate of 9600 bps and
parity bit disabled.

As 253, 254 and 255 are reserved as command words, the allowed input
range is from 0 through 252.  However, the heater input is limited to
40 while the fan is allowed the entire range of up to 252.

A 2x16 character alphanumeric LCD is used to display temperature, the
PWM value given to the heating element and the fan, and the last
serially communicated byte.  The LCD is used in 4-bit interfacing mode
to reduce the number of GPIOs required.

We have used the open source IDE WinAVR and AVR Studio and the
commercial ICCAVR as IDE.

The time constant of this system is less than a minute and hence, one
can do a realistic experiment in ten minutes.  One can also see
meaningful (and noisy) measurements with naked eye.  It is suitable
for carrying out ALL experiments of a few control courses.  One only
needs a 220V power supply and a PC.  It is easy to carry.  The design,
firmware and data acquisition interface are all in open source!  The
bill of material costs only \$30.  As a finished product, it is
available for about \$50 plus shipping charges.  Manuals to build,
commission and to experiment with this device are available
\cite{VL010}.
