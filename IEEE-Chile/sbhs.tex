\section{Single Board Heater System (SBHS)}
A single-board heater system is a low cost, open source, lab-in-a-box
setup \cite{ia010}.  It consists of a heater assembly, fan,
temperature sensor, ATmega16 microcontroller and associated
circuitry. A stainless steel blade whose temperature has to be
controlled serves as the plant.  Nichrome helical coil with 20 turns
kept at a small distance from the steel blade, acts as the heater
element.  AD590, a monolithic integrated circuit temperature
transducer, is soldered beneath the steel plate.  A computer fan, a
low cost and commercially off the shelf component, is used to cool the
plate from below.

The plant has a small time constant, less than a minute, that allows
completion of an experiment in a short time.  This in turn facilitates
performance of a large number of experiments in a single laboratory
session.  The speed of response not being too fast allows the
measurements to be seen with naked eye, as it happens in industrial
systems.  It also demonstrates other measurement issues, such as,
noise. 

The codes for hardware interface and control experiments, and manuals
are available at \cite{vl010,moodle}.  Unlike \cite{ia010}, for the
purpose of remote access, 252 until 255 are reserved as command words
where 252 is meant for communication of machine I.D. This allows fan
input to vary from 0 to 251 PWM (Pulse Width Modulation) units. 
