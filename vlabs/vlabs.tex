\chapter{Using Single Board Heater System, Virtually!}
\section{Introduction to Virtual Labs at \\IIT Bombay}
The concept of virtual laboratory is a brilliant step towards strengthening the education system of an university/college, a metropolitan area or even an entire nation. The idea is to use the ICT i.e. Information and Communications Technology, mainly the Internet for imparting education or exchange of educational information. Virtual Laboratory mainly focuses on providing the laboratory facility, virtually. Various experimental set-ups are hooked up to the internet and made available to use for the external world. Hence, anybody can connect to that equipment over the internet and carry out various experiments pertaining to it. The beauty of this idea is that a college who cannot afford to have some experimental equipments can still provide laboratory support to their students through virtual lab, and all that will cost it is a fair Internet connection! Moreover, the laboratory work does not ends with the college hours, one can always use the virtual lab at any time and at any place assuming the availability of an internet connection. 

A virtual laboratory for SBHS is launched at IIT Bombay. Here is the url to access it: {\bf http://www.co-learn.in/web\_sbhs/}. A set of 15 SBHS are made available to use over the internet 24$\times$ 7. These individual kits are made available to the users on hourly basis. We have a slot booking mechanism to achieve this. Since there are 15 SBHS connected with an hours slot for 24 hrs a day, we have 360 one hour slots a day. This means that 360 individual users can access the SBHS in a day for an hour. This also means that up to 2520 users can use the SBHS for an hour in a week and more than 10,000 in a month! A web page is hosted which is the first interface to the user. The user registers/logs in himself/herself here. The user is also supposed to book a slot for accessing the SBHS. A database server maintains a record of the data generated through the web interface. A python script is hosted on the server side and it helps in connecting the user with the corresponding SBHS placed remotely. The client is also suppose to run a java client on his computer. A free and open source scientific computing Software, Scilab, is used by the user for implementing the experiment on SBHS, in terms of simple Scilab coding. 
%\section{Evolution of SBHS virtual labs}
In \cite{vlabs-kmm}, 
the control algorithm is implemented at the server end and the remote
student just keys in the parameters, as shown in Figure
\ref{fig:initial}. 
\begin{figure}
\includegraphics[width=\linewidth]{IEEE-Chile/figures/vlab-1.png}
\caption{SBHS virtual laboratory with remote access using LabVIEW}
\label{fig:initial}
\end{figure}
LabVIEW was used for the implementation of the same. The
server end consisted of a computer connected with an SBHS with a full
blown copy of LabVIEW installed on it. The client has a LabVIEW run
time engine available for free download from the National Instruments
website.  A few
LabVIEW algorithms/experiments were hosted on the server. The client
accesses these algorithm/experiment over the Internet using a web
browser by entering appropriate parameters.

It was realized that the learning experience is not complete for this
structure. This is because the server hosts some pre-built LabVIEW
algorithms and a user can only access these few algorithms. The user
can in no way change the program and can only input experimental
parameters. 
Hence, we came up with a new architecture
as shown in the Figure \ref{fig:second} that used full blown copies of
LabVIEW at both server and client ends.  
\begin{figure}
\includegraphics[width=\linewidth]{IEEE-Chile/figures/vlab-2.png}
\caption{SBHS virtual laboratory with remote access and live data sharing using LabVIEW}
\label{fig:second}
\end{figure}
 
 This idea uses the DataSocket technology of LabVIEW. Since now the
 client is having a complete LabVIEW installation on his/her computer
 she can now implement her own algorithms.  Thus this architecture did
 provide a complete learning experience to the students.  There are
 some shortcomings as well:

\begin{itemize}
\item LabVIEW is expensive and students may not be able to afford to
  buy it.  It is also prohibitively expensive for the Government to
  distribute it.

\item We used the LabVIEW version 8.04, which had restricted scripting
  language.  It was tedious to create new control algorithms in it.
\end{itemize}
\begin{figure}
\includegraphics[width=\linewidth]{IEEE-Chile/figures/vlab-3.png}
\caption{SBHS virtual laboratory using open source software}
\label{fig:third}
\end{figure}
This made us shift to free and open source (FOSS) software. We
replaced LabVIEW with Java and Scilab as shown in Figure
\ref{fig:third}. Scilab at the server end is used for communicating
with SBHS. Scilab at the client end is used for implementing the
algorithms. Java is used at both the server as well as client end for
communication over the Internet thereby connecting the client with the
server. 

For the above solution, we need a dedicated copy of scilab running at
the server end for every SBHS. One way to do this is to host it on
multiple computers with unique IPs. Hence the number of SBHS we want
to host requires as many computer's and public IPs thereby making
it expensive. Moreover, it also limits its scalability. The other way
to do this is to host multiple java and scilab servers on the same
computer.  Hosting many copies of Scilab simultaneously requires a
powerful computer for the server.

For these reasons  we decided to take scilab off the server computer
and to use java alone to communicate with the SBHS directly.  Java
also 
communicates with the client computer.  We connected seven SBHS
systems to a USB port through a serial port hub.  This architecture
was 
implemented on a Windows Operating System.  We faced the following
difficulties in this solution.
\begin{itemize}
\item When we connected more than one serial hub to a PC, the port ID
  could not be retrieved correctly.  Port ID information is required
  if we want a student to use the same SBHS for all their experiments
  during different sessions.
\item The experiments required time stamping of the data communicated
  to and from the server. But this time stamping was not linear and
  suffered instability.  
\end{itemize}%
This made us to completely switch to FOSS with Ubuntu Linux as the OS
and is the current structure of the Virtual lab as shown in Figure
\ref{fig:detail-arch} 
\section{Current Architecture}
\label{sec:vlabarchi}
\begin{figure}
\centering
\includegraphics[width=\linewidth]{IEEE-Chile/figures/vlab-arch}
\caption{Virtual control lab hardware architecture}
\label{fig:hw-arch}
\end{figure}

\subsection{Hardware}
The architecture of the virtual single-board heater system lab as
shown in Figure \ref{fig:hw-arch} involves 7 single-board heater
systems connected to the server via a 7-port USB hub. The server
computer is connected to a high speed internetwork and has enough
processing capability to host data acquisition, database, and web
servers. The internetwork connected client computer needs only the
Java runtime engine and Scilab installation.  

A similar architecture but replacing Java with python and with 15 units connected to the server via two 7-port hubs has been successfully tested on intranet for the undergraduate Process Control course and the graduate Digital Control
and Embedded systems courses conducted at IIT Bombay as well as few workshops over the internet. Currently, this architecture is integrated with a cameras on each SBHS to facilitate live video streaming. This facility will be extended to all the units in future. This gives the user a feel of remote hands-on. Work on this camera facility is in progress.


\subsection{Software}
The current software architecture of this virtual control lab is shown
in Figure \ref{fig:detail-arch}. The server computer runs on Ubuntu
Linux 10.04 OS. It hosts a LAMP (Linux-Apache-MySQL-PHP) server. The
MySQL-based database server has the details of all the registered
users, their slot details, authentication keys to allow remote access,
etc. It also hosts a PHP based web server shown in Figure
\ref{fig:sbhs-website} that has pages for registration, login and slot
booking \cite{vl010}.  For communication with the device, Django - a Python based web framework is used on the Server side. On the client end has control
algorithms running in Scilab and communicates over the Internet
through the python client.

\begin{figure}
\includegraphics[width=\linewidth]{IEEE-Chile/figures/webpage}
\caption{Home page of SBHS V Labs}
\label{fig:sbhs-website}
\end{figure}
The steps to be performed before and during each experiment are explained next.

\begin{figure}
\centering
\includegraphics[width=\linewidth]{IEEE-Chile/figures/blk-dig.pdf}
\caption{Current Architecture of SBHS Virtual Labs}
\label{fig:detail-arch}
\end{figure}



\subsubsection{Registration}
A client willing to perform the experiments needs to register with us
by entering the personal details on the Registration page. This sends
an account activation link to the client’s mailbox. Upon clicking the
link, the account gets activated.
\subsubsection {Slot booking} The client can now login and book slots
to perform the experiments. Each slot lasts for an hour with 55
minutes for experimentation and 5 mins for resetting the setup. A
client can book up to 2 slots, per day, in advance. Besides this, if
the current slot is empty, it can be booked as a free slot. For each
slot being booked, the client gets a port number and an access
key. The interface depicting this is shown in Figure
\ref{fig:slot-booking}.
\begin{figure}
\includegraphics[width=\linewidth]{IEEE-Chile/figures/slot-book.png}
\caption{Screenshot of Slot Booking Page}
\label{fig:slot-booking}
\end{figure}
\subsubsection {Port number} In order to maintain consistency in
performing the experiments, each client is alloted the same setup
every time s/he requests for the slot. To facilitate this, each setup
has been alloted a machine I.D. For example, if the machine I.D. of
the unit is 5, it is displayed as {\tt SBHS Number: 5}. This
eliminates time and effort spent in plant modelling, each time, due to
change in the setup.

\subsubsection { Starting the Python client} On windows, after installing python, it will not be readily recognized by the OS. To make windows interpret python as a command, do the following steps.
\begin{enumerate}
\item Right click on My Computer and click on properties
\item Click on the Advanced tab.Now click on the Environment Variables button
In the system variable section, locate and click on “path”.
\item Click the edit button below it. You should see a “Edit system variable” window.
\item In the “Variable value” field, take the cursor to the last.
Type a semicolon and type C:$\backslash$Python27
\end{enumerate}
The experiment folder has a run.bat file, settings.txt file and a sbhsclient.py file. Open the settings.txt file and make the required changes. Beware not to make any unneccesary changes to this file. The example settings are anyways given in the end of this file. Linux users should bypass the terminal proxy settings, if any, by executing the command {\tt export http\_proxy=''}. Be informed that this is just a temporay disable of proxy and it should be done everytime you open a fresh terminal. Windows users simply double click on the run.bat file. It will open up a command prompt. Linux users instead of executing the run.bat file will type the following command on the terminal {\tt python sbhsclient.py}. This activity will ask you for the username and password. If the settings are not proper in the settings.txt file, you will get a {\tt Connection error message}. Contact the system administrator of your network to get the correct values of the parameters. Once you are done typing your username and password, you will get a Login Successfull message. In case if you are logging in at a time other that your booked slot, it will tell you that No slot is found. Else, it will notify that you have booked a slot and can now run the scilab code. It will also create {\tt scilabread.sce, scilabwrite.sce} and a log file in the experiment folder. The log file name will be in the format {\tt date\_month\_year\_hrs\_mins\_seconds.txt}. This unique file name will help you track the log file of you choice in the future. 
\begin{figure}
  \centering
\includegraphics[width=0.7\linewidth]{IEEE-Chile/figures/display.jpg}
\caption{Screenshot of Video Stream}
\label{video}
\end{figure}
\subsubsection { Video streaming} At present, the facility to provide video streaming for all users is in progress. The video
streaming of the capture display will be something as shown in Figure \ref{video}. The clients with lesser bandwidth will have an option to refrain from video streaming. This facility will be available for logged in users on the website.

\subsubsection { Executing the Scilab code} The client needs to
download the scilab code from the SBHS home page. This is available
under {\tt Downloads}. Various other experiments are also available on {\tt fossee.in/moodle} The client can modify this code to implement
his/her own algorithm. S/he should not edit the part of the code which
is used for communicating with python. This part is clearly marked in
the code. Refer the chapter specific to which experiment you want to run. It will explain you which codes and in what order they should be executed.

\subsubsection { File handling} This subsection explains the
significance of scilabread and scilabwrite files. During the
experiment, the client writes heater and fan inputs to the
scilabwrite.sce file. These values are read by the Python client. It writes iteration and client departure time stamp and send it over the network. Python server at the other end reads these values, puts a server arrival time stamp and gives them to the SBHS through its data acquisition
interface. After feeding the values, it reads the temperature of the
heated plate, puts a server departure time stamp and sends it to the Python client. The Python client puts a client arrival time stamp and then writes all these values i.e. heater, fan and temperature along with the
timestamps to the scilabread.sce file. Scilab client reads the latest
temperature value and does the calculation of heater and fan inputs in
accordance to the logic developed for the experiment. The live data
streaming involves heater and fan inputs being sent by the client and
temperature response in turn being sent by the server. The timestamp accompanying the data could be used for real-time
control of the setup.

\subsubsection { Concluding the experiment} Once the client is done
with the experiment, s/he can use the log file created in the experiment folder for analysis purpose. If you happen to lose the log file, you can download it from the website, after you login, by clicking on {\tt Download} link.
%\end{enumerate}

\subsection{Other Implementation Issues}
%\begin{enumerate}
\subsubsection { Mapping of machine id with the USB port number}
Whenever a SBHS unit is plugged in to a USB port, the dev id (/dev/ttyUSB*) assigned to it by OS is different. A script is run which creates a table of the mapping between the machine I.D. and the USB dev id. This is done to ensure that the client gets the same machine I.D. i.e. the same SBHS each time. 

\subsubsection { Auto log off problem}
In some ISPs, the network gets disconnected if it is inactive for some time. To handle this situation, we are running a shell script which checks for internet connectivity by pingging google periodically. If the network is down then it will try to reconnect.  


%\end{enumerate}


\subsection{Support}
The SBHS support activities are being funded by National Mission on Education through Information and Communication Technology \cite{kmm010}.  The major support activities include conducting workshops in several colleges across the country, active discussion through Fossee moodle and spoken tutorials for self-learning.

\subsubsection{Workshops}
The SBHS team has been conducting workshops to popularise the utility of SBHS.  The course content is around local and remote access of the SBHS. More than 50 college teachers from several Engineering colleges of the country have attended the workshops and have started using the setup for the relevant courses running in their colleges.

\subsubsection{Fossee moodle}


Fossee moodle is a web interface that allows discussions, post queries, upload files, upload grades, etc.  Also, there is a facility to provide e-mail notification whenever there is a post on the course website. Thus, such an interface allows to address common problems of many users at the same time. It also provides a platform for the users at different locations to share their experiences during the experiments.  The codes and manuals for about 8 experiments are available at \cite{moodle}.


\begin{figure}
\centering
\includegraphics[width=0.75\linewidth]{IEEE-Chile/figures/fossee-2}
\caption{Fossee moodle webpage to support SBHS activities}
\label{fig:fossee-sup}
\end{figure}



\subsubsection{Spoken tutorials}
A Spoken tutorial is a screen capture along with the audio that could be used to teach any computer application.  Currently, the spoken tutorials for various Free and Open source softwares are available in different Indian languages at \cite{spoken-sci}.  The scripts and tutorials for SBHS are in pipeline.





\section{Evolution of SBHS virtual labs}
In \cite{vlabs-kmm}, 
the control algorithm is implemented at the server end and the remote
student just keys in the parameters, as shown in Figure
\ref{fig:initial}. 
\begin{figure}
\includegraphics[width=\linewidth]{IEEE-Chile/figures/vlab-1.png}
\caption{SBHS virtual laboratory with remote access using LabVIEW}
\label{fig:initial}
\end{figure}
LabVIEW was used for the implementation of the same. The
server end consisted of a computer connected with an SBHS with a full
blown copy of LabVIEW installed on it. The client has a LabVIEW run
time engine available for free download from the National Instruments
website.  A few
LabVIEW algorithms/experiments were hosted on the server. The client
accesses these algorithm/experiment over the Internet using a web
browser by entering appropriate parameters.

It was realized that the learning experience is not complete for this
structure. This is because the server hosts some pre-built LabVIEW
algorithms and a user can only access these few algorithms. The user
can in no way change the program and can only input experimental
parameters. 
Hence, we came up with a new architecture
as shown in the Figure \ref{fig:second} that used full blown copies of
LabVIEW at both server and client ends.  
\begin{figure}
\includegraphics[width=\linewidth]{IEEE-Chile/figures/vlab-2.png}
\caption{SBHS virtual laboratory with remote access and live data sharing using LabVIEW}
\label{fig:second}
\end{figure}
 
 This idea uses the DataSocket technology of LabVIEW. Since now the
 client is having a complete LabVIEW installation on his/her computer
 she can now implement her own algorithms.  Thus this architecture did
 provide a complete learning experience to the students.  There are
 some shortcomings as well:

\begin{itemize}
\item LabVIEW is expensive and students may not be able to afford to
  buy it.  It is also prohibitively expensive for the Government to
  distribute it.

\item We used the LabVIEW version 8.04, which had restricted scripting
  language.  It was tedious to create new control algorithms in it.
\end{itemize}
\begin{figure}
\includegraphics[width=\linewidth]{IEEE-Chile/figures/vlab-3.png}
\caption{SBHS virtual laboratory using open source software}
\label{fig:third}
\end{figure}
This made us shift to free and open source (FOSS) software. We
replaced LabVIEW with Java and Scilab as shown in Figure
\ref{fig:third}. Scilab at the server end is used for communicating
with SBHS. Scilab at the client end is used for implementing the
algorithms. Java is used at both the server as well as client end for
communication over the Internet thereby connecting the client with the
server. 

For the above solution, we need a dedicated copy of scilab running at
the server end for every SBHS. One way to do this is to host it on
multiple computers with unique IPs. Hence the number of SBHS we want
to host requires as many computer's and public IPs thereby making
it expensive. Moreover, it also limits its scalability. The other way
to do this is to host multiple java and scilab servers on the same
computer.  Hosting many copies of Scilab simultaneously requires a
powerful computer for the server.

For these reasons  we decided to take scilab off the server computer
and to use java alone to communicate with the SBHS directly.  Java
also 
communicates with the client computer.  We connected seven SBHS
systems to a USB port through a serial port hub.  This architecture
was 
implemented on a Windows Operating System.  We faced the following
difficulties in this solution.
\begin{itemize}
\item When we connected more than one serial hub to a PC, the port ID
  could not be retrieved correctly.  Port ID information is required
  if we want a student to use the same SBHS for all their experiments
  during different sessions.
\item The experiments required time stamping of the data communicated
  to and from the server. But this time stamping was not linear and
  suffered instability.  
\end{itemize}%
This made us to completely switch to FOSS with Ubuntu Linux as the OS
and is the current structure of the Virtual lab as shown in Figure
\ref{fig:detail-arch} 
\section{Current Architecture}
\label{sec:vlabarchi}
\begin{figure}
\centering
\includegraphics[width=\linewidth]{IEEE-Chile/figures/vlab-arch}
\caption{Virtual control lab hardware architecture}
\label{fig:hw-arch}
\end{figure}

\subsection{Hardware}
The architecture of the virtual single-board heater system lab as
shown in Figure \ref{fig:hw-arch} involves 7 single-board heater
systems connected to the server via a 7-port USB hub. The server
computer is connected to a high speed internetwork and has enough
processing capability to host data acquisition, database, and web
servers. The internetwork connected client computer needs only the
Java runtime engine and Scilab installation.  

A similar architecture but replacing Java with python and with 15 units connected to the server via two 7-port hubs has been successfully tested on intranet for the undergraduate Process Control course and the graduate Digital Control
and Embedded systems courses conducted at IIT Bombay as well as few workshops over the internet. Currently, this architecture is integrated with a cameras on each SBHS to facilitate live video streaming. This facility will be extended to all the units in future. This gives the user a feel of remote hands-on. Work on this camera facility is in progress.


\subsection{Software}
The current software architecture of this virtual control lab is shown
in Figure \ref{fig:detail-arch}. The server computer runs on Ubuntu
Linux 10.04 OS. It hosts a LAMP (Linux-Apache-MySQL-PHP) server. The
MySQL-based database server has the details of all the registered
users, their slot details, authentication keys to allow remote access,
etc. It also hosts a PHP based web server shown in Figure
\ref{fig:sbhs-website} that has pages for registration, login and slot
booking \cite{vl010}.  For communication with the device, Django - a Python based web framework is used on the Server side. On the client end has control
algorithms running in Scilab and communicates over the Internet
through the python client.

\begin{figure}
\includegraphics[width=\linewidth]{IEEE-Chile/figures/webpage}
\caption{Home page of SBHS V Labs}
\label{fig:sbhs-website}
\end{figure}
The steps to be performed before and during each experiment are explained next.

\begin{figure}
\centering
\includegraphics[width=\linewidth]{IEEE-Chile/figures/blk-dig.pdf}
\caption{Current Architecture of SBHS Virtual Labs}
\label{fig:detail-arch}
\end{figure}



\subsubsection{Registration}
A client willing to perform the experiments needs to register with us
by entering the personal details on the Registration page. This sends
an account activation link to the client’s mailbox. Upon clicking the
link, the account gets activated.
\subsubsection {Slot booking} The client can now login and book slots
to perform the experiments. Each slot lasts for an hour with 55
minutes for experimentation and 5 mins for resetting the setup. A
client can book up to 2 slots, per day, in advance. Besides this, if
the current slot is empty, it can be booked as a free slot. For each
slot being booked, the client gets a port number and an access
key. The interface depicting this is shown in Figure
\ref{fig:slot-booking}.
\begin{figure}
\includegraphics[width=\linewidth]{IEEE-Chile/figures/slot-book.png}
\caption{Screenshot of Slot Booking Page}
\label{fig:slot-booking}
\end{figure}
\subsubsection {Port number} In order to maintain consistency in
performing the experiments, each client is alloted the same setup
every time s/he requests for the slot. To facilitate this, each setup
has been alloted a machine I.D. For example, if the machine I.D. of
the unit is 5, it is displayed as {\tt SBHS Number: 5}. This
eliminates time and effort spent in plant modelling, each time, due to
change in the setup.

\subsubsection { Starting the Python client} On windows, after installing python, it will not be readily recognized by the OS. To make windows interpret python as a command, do the following steps.
\begin{enumerate}
\item Right click on My Computer and click on properties
\item Click on the Advanced tab.Now click on the Environment Variables button
In the system variable section, locate and click on “path”.
\item Click the edit button below it. You should see a “Edit system variable” window.
\item In the “Variable value” field, take the cursor to the last.
Type a semicolon and type C:$\backslash$Python27
\end{enumerate}
The experiment folder has a run.bat file, settings.txt file and a sbhsclient.py file. Open the settings.txt file and make the required changes. Beware not to make any unneccesary changes to this file. The example settings are anyways given in the end of this file. Linux users should bypass the terminal proxy settings, if any, by executing the command {\tt export http\_proxy=''}. Be informed that this is just a temporay disable of proxy and it should be done everytime you open a fresh terminal. Windows users simply double click on the run.bat file. It will open up a command prompt. Linux users instead of executing the run.bat file will type the following command on the terminal {\tt python sbhsclient.py}. This activity will ask you for the username and password. If the settings are not proper in the settings.txt file, you will get a {\tt Connection error message}. Contact the system administrator of your network to get the correct values of the parameters. Once you are done typing your username and password, you will get a Login Successfull message. In case if you are logging in at a time other that your booked slot, it will tell you that No slot is found. Else, it will notify that you have booked a slot and can now run the scilab code. It will also create {\tt scilabread.sce, scilabwrite.sce} and a log file in the experiment folder. The log file name will be in the format {\tt date\_month\_year\_hrs\_mins\_seconds.txt}. This unique file name will help you track the log file of you choice in the future. 
\begin{figure}
  \centering
\includegraphics[width=0.7\linewidth]{IEEE-Chile/figures/display.jpg}
\caption{Screenshot of Video Stream}
\label{video}
\end{figure}
\subsubsection { Video streaming} At present, the facility to provide video streaming for all users is in progress. The video
streaming of the capture display will be something as shown in Figure \ref{video}. The clients with lesser bandwidth will have an option to refrain from video streaming. This facility will be available for logged in users on the website.

\subsubsection { Executing the Scilab code} The client needs to
download the scilab code from the SBHS home page. This is available
under {\tt Downloads}. Various other experiments are also available on {\tt fossee.in/moodle} The client can modify this code to implement
his/her own algorithm. S/he should not edit the part of the code which
is used for communicating with python. This part is clearly marked in
the code. Refer the chapter specific to which experiment you want to run. It will explain you which codes and in what order they should be executed.

\subsubsection { File handling} This subsection explains the
significance of scilabread and scilabwrite files. During the
experiment, the client writes heater and fan inputs to the
scilabwrite.sce file. These values are read by the Python client. It writes iteration and client departure time stamp and send it over the network. Python server at the other end reads these values, puts a server arrival time stamp and gives them to the SBHS through its data acquisition
interface. After feeding the values, it reads the temperature of the
heated plate, puts a server departure time stamp and sends it to the Python client. The Python client puts a client arrival time stamp and then writes all these values i.e. heater, fan and temperature along with the
timestamps to the scilabread.sce file. Scilab client reads the latest
temperature value and does the calculation of heater and fan inputs in
accordance to the logic developed for the experiment. The live data
streaming involves heater and fan inputs being sent by the client and
temperature response in turn being sent by the server. The timestamp accompanying the data could be used for real-time
control of the setup.

\subsubsection { Concluding the experiment} Once the client is done
with the experiment, s/he can use the log file created in the experiment folder for analysis purpose. If you happen to lose the log file, you can download it from the website, after you login, by clicking on {\tt Download} link.
%\end{enumerate}

\subsection{Other Implementation Issues}
%\begin{enumerate}
\subsubsection { Mapping of machine id with the USB port number}
Whenever a SBHS unit is plugged in to a USB port, the dev id (/dev/ttyUSB*) assigned to it by OS is different. A script is run which creates a table of the mapping between the machine I.D. and the USB dev id. This is done to ensure that the client gets the same machine I.D. i.e. the same SBHS each time. 

\subsubsection { Auto log off problem}
In some ISPs, the network gets disconnected if it is inactive for some time. To handle this situation, we are running a shell script which checks for internet connectivity by pingging google periodically. If the network is down then it will try to reconnect.  


%\end{enumerate}


\subsection{Support}
The SBHS support activities are being funded by National Mission on Education through Information and Communication Technology \cite{kmm010}.  The major support activities include conducting workshops in several colleges across the country, active discussion through Fossee moodle and spoken tutorials for self-learning.

\subsubsection{Workshops}
The SBHS team has been conducting workshops to popularise the utility of SBHS.  The course content is around local and remote access of the SBHS. More than 50 college teachers from several Engineering colleges of the country have attended the workshops and have started using the setup for the relevant courses running in their colleges.

\subsubsection{Fossee moodle}


Fossee moodle is a web interface that allows discussions, post queries, upload files, upload grades, etc.  Also, there is a facility to provide e-mail notification whenever there is a post on the course website. Thus, such an interface allows to address common problems of many users at the same time. It also provides a platform for the users at different locations to share their experiences during the experiments.  The codes and manuals for about 8 experiments are available at \cite{moodle}.


\begin{figure}
\centering
\includegraphics[width=0.75\linewidth]{IEEE-Chile/figures/fossee-2}
\caption{Fossee moodle webpage to support SBHS activities}
\label{fig:fossee-sup}
\end{figure}



\subsubsection{Spoken tutorials}
A Spoken tutorial is a screen capture along with the audio that could be used to teach any computer application.  Currently, the spoken tutorials for various Free and Open source softwares are available in different Indian languages at \cite{spoken-sci}.  The scripts and tutorials for SBHS are in pipeline.





\section{Conducting experiments using the Virtual lab}\label{vlabsexpt}
In this section we will see the steps involved in conducting experiments using virtual lab.
%In this section we will see the details of the building blocks of virtual lab. Figure \ref{block_dig} shows the block diagram for the same. A virtual lab is similar to a remote lab where in communication takes place between two computers over the internet. The server computer consists of a web server, a data base server and a java application sever. The client computer should contain a web browser, Scilab Software and a java client. A web server is hosted using apache software. The web pages are written in php language and are hosted on this server. Some of these php pages are necessarily linked to a data base server. The Database Server is hosted using mysql. It keeps a record of the information generated during a person registers and books a slot. The java server communicates with the SBHS connected to the computer on a USB port. It also accesses the data base and communicates with the java client. 
%
%
%%\begin{figure}
%%\centering
%%\includegraphics[width=\linewidth]{vlabs/block_dig.pdf}
%%\caption{Block diagram of the Virtual lab setup}
%%\label{block_dig}
%%\end{figure}
%On the client side, the web browser is any internet browser available. The user visits the specified web page, logs in/ registers and book/view/delete a slot(s) using the web brwser available to him. The java client is a simple java application provided by us. The user keys in the information provided on the web page pertaining to the booked slot in to this java client. The interface also provides an option to view a live streaming of the display of the SBHS to which he has connected to. This facility is however limitied to one SBHS for now and an elegent method is currently under development to accomodate all of the SBHS. The Scilab software must be installed by the client on his computer. Here, he will write the experiment he wants to conduct on the SBHS in simple scilab language. A primary scilab code will be provided and the process of modification of the code will be explained. 
This section assumes that the necessary files and software are available on the users computer. Files can be downloaded from the vlabs home page {\bf http://www.co-learn.in/web\_sbhs/} as shown in fig \ref{fig:sbhs-website}
\begin{enumerate}
\item The user goes to the vlabs home page and if he is a first time visitor, he registers himself and follows the information provided there, else he directly logs in.
\item After logging in, the user will book a slot.
\item In the experiment folder the settings.txt file will be edited as per requirement. The client should take care not to make any unneccesary changes to this file. The example settings are anyways given in the end of this file.

\item The user will now start the python client. On windows OS, double clicking on the run.bat file will open the command prompt. On linux OS, one has to first go to the directory where the experiment files (.sce and .py) are kept. This can be done by typing {\tt cd<space>directory name} on the terminal. Linux users should also bypass the terminal proxy settings, if any, by executing the command {\tt export http proxy=’’} on the terminal. Be informed that this is just a temporay disable of terminal proxy and it should be done everytime you open a fresh terminal. Linux user should now type the command {\tt python sbhsclient.py} on the linux  terminal. 
\item The python client will first do some network checks. If it finds the network to be NOT ok for experimentation, it will put the message "No network connection".Else, if it finds the network to be ok for experimentation, it will communicate with the server and will authenticate the user if he is authorised to access the particular SBHS at the booked time. If the user is not authorised it he/she will get the message "Authentication failed. Please check your username and password". If the user is authenticate but has not booked the slot, he/she will get the message "No valid slot found. Please book a slot before starting the experiment". Also if the settings.txt file is not properly configured, it will give error {\tt Invalid settings in the "setttings.txt" file}. If every thing is fine, the client will get the "Login Successfull message." It will also display the log file name and the time left for experimentation.
\item The user now launches scilab, changes the directory to the folder where the necessary .sce, .sci and .xcos files of an experiment of interest resides. The user will execute the scilab code (.sce file) pertaining to the experiment. If the network connection is fine, it will automatically open the corresponding .xcos file. Else it will output a message on the scilab console {\tt No network connection}.
\item The user will run the xcos file. It will open a plot of the various experimental parameters. This process will continue until the experiment is stopped or the simulation time is lapsed. The simulation time can be changed by changing the {\tt final integration time} parameter available in the {\tt set up} option in the {\tt simulation} menu on the Xcos window.

\item The slot is made to last for 55 minutes. The last 5 minutes of the slot are used to reset the SBHS so that the next user will get the SBHS at a normal operating condition. The python client ceases connection automatically as soon as 55 minutes are over. Please note that there will be no pop up warning and the experiment will be stoped automatically.
\item A log of the experimental data with time stamp is maintained on the client side. It is also available to download using the "Download" link once you login on the sbhs vlabs website.
\end{enumerate} 

