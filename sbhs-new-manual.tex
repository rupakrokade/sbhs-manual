\documentclass[12pt]{report}

\usepackage{layouts}
\usepackage{morefloats}
\usepackage{epsfig,amsfonts,amsmath,amstext,amssymb,latexsym}
\usepackage{theorem,algorithmic,algorithm,graphicx,pxfonts,txfonts}
%\usepackage{fancyheadings}
\usepackage{fancyhdr}
\usepackage{boxedminipage,mathrsfs,verbatim}
\usepackage{float,calc}
\usepackage{subfigure}
\usepackage[draft]{varioref}
\vrefwarning
\usepackage{color}
\usepackage{listings} 
\usepackage{textcomp,pgfpict2e,tikz}



\usepackage[utf8]{inputenc}
\usepackage[english]{babel}
\usepackage{float}
\setcounter{secnumdepth}{3}



\bibliographystyle{plain}

\lstset{ %configuring the display of scilab codes
			tabsize=4,
			language=scilab,
			aboveskip={1\baselineskip},
			showstringspaces=false,
			breaklines=true,
			showspaces=false,
			numbers=left,
			numberstyle=\tiny,
			commentstyle=\scriptsize}
\usetikzlibrary{shapes,arrows}
\tikzstyle{block} = [draw, fill=white!20, rectangle, 
    minimum height=3em, minimum width=6em]
\tikzstyle{sum} = [draw, fill=white!20, circle, node distance=1cm]
\tikzstyle{input} = [coordinate]
\tikzstyle{output} = [coordinate]
\tikzstyle{pinstyle} = [pin edge={to-,thin,black}]
\tikzstyle{branch} = [circle,inner sep=0pt,minimum size=1mm,fill=black,draw=black]

% code environment

{\theorembodyfont{\rmfamily} \newtheorem{codemass}{Scilab Code}[chapter]}
\newenvironment{code}%
{\begin{codemass}}{\hrule \end{codemass}}

% create listing for code

\newcommand\ccaption[1]
     {\addcontentsline{cod}{section}{\protect\numberline {\thecodemass}#1}}
\makeatletter \newcommand\listofcode
     {\chapter*{List of Scilab Code\markboth%
                        {\bf List of Scilab Code}{}}%
\renewcommand*\l@section{\@dottedtocline{1}{1.5em}{3em}}%
\addcontentsline{toc}{chapter}{\protect\numberline{List of Scilab\ Code}}
\@starttoc{cod}}
\newcommand\l@matlab[3]
     {#1 \par\noindent#2, #3 \par} 
\renewcommand\@pnumwidth{2.1em}
\makeatother

\title{Documentation 
for \\ Single Board Heater System}
\author{Rakhi R\\Rupak Rokade \\ Inderpreet Arora \\ Kannan M. Moudgalya \\ Kaushik Venkata Belusonti\vspace{1in}}


\date{
\begin{figure}[h]
\centering
\includegraphics[width=0.3\linewidth]{iitblogo.pdf}
\end{figure}IIT Bombay \\\today}

\begin{document}


%\begin{figure}
%\begin{center}
%\begin{tikzpicture}[auto,node distance=2cm]
%\node[input](input){};
%\node[block,right of=input](tcbyrc){$\gamma \frac{T_c(z)}{R_c(z)}$};
%\draw[->](input)--node{$r$}(tcbyrc);
%\node[sum,right of=tcbyrc,node distance=2cm](sum1){};
%\draw[->](tcbyrc)--node{}(sum1);
%\node[block,right of=sum1](plant){$G=z^{-k}\frac{B(z)}{A(z)}$};
%\draw[->](sum1)--node{$u$}(plant);
%\node[output,right of=plant](output){};
%\draw[->](plant)--node[name=y]{$y$}(output);
%\node[block,below of=plant](scbyrc){$\frac{S_c}{R_c}$};
%\draw[->](y)|-node{}(scbyrc);
%\draw[->](scbyrc)-|node[pos=0.99]{$-$}(sum1);
%\end{tikzpicture}
%\end{center}
%\caption{2-DOF pole placement controller}
%\label{2dofppc}
%\end{figure}


\maketitle
\tableofcontents
\listofcode
\chapter{Block diagram explanation of Single Board Heater System}
\begin{figure}
\centering
\includegraphics[width=\linewidth]{blkdiag.jpg}
\caption{Block Diagram}
\label{blkdig}
\end{figure}
Figure\ref{blkdig} shows the block diagram of \textquoteleft Single Board Heater System\textquoteright. Microcontroller ATmega16 is used and is the heart of the setup. Two serial communication ports namely RS232 and USB can be used to connect the setup to a computer. A particular port can be selected by setting the jumper to its appropriate place. The communication between PC and setup takes place via a serial to TTL interface. The microcontroller can be programmed with the help of an In-system programmer port(ISP) available on the board. The \textmu C operates the Heater and Fan with the help of separate drivers. The driver comprises of a power MOSFET. A temperature sensor is used to sense the temperature and fed to the \textmu C through an Instrumentation amplifier. Some required parameter values are displayed along with some LED indications.
\section{Microcontroller}
Some salient features of ATmega16 are listed below
\begin{enumerate}
\item 32 x 8 general purpose registers.
\item 16K Bytes of In-System Self-Programmable flash memory
\item 512 Bytes of EEPROM
\item 1K Bytes of internal Static RAM (SRAM)
\item Two 8-bit Timer/Counters
\item One 16-bit Timer/Counter
\item Four PWM channels
\item 8-channel,10-bit ADC
\item Programmable Serial USART
\item Up to 16 MIPS throughput at 16 MHz
\end{enumerate}
Microcontroller plays a very important role. It controls every single hardware present on the board, directly or indirectly. It executes various tasks like, setting up communication between PC and the equipment, controlling the amount of current passing through the heater coil, controlling the fan speed, reading the temperature, displaying some relevant parameter values and various other necessary operations.
\subsection{PWM for heat and speed control}
The Single Board Heater System has a Heater coil and a Fan. The heater assembly consists of an iron plate placed at a distance of about 3.5mm from a nichrome coil. When current passes through the coil it gets heated and in turn raises the temperature of the iron plate. We are interested to alter the heat generated by the coil and also the speed at which the fan is operated.
There are many ways to control the amount of power delivered to the Fan and Heater. We are using the PWM technique.
PWM or pulse width modulation is a process in which the duty cycle of the square wave is modulated.
\begin{align}
      \text{Duty cycle} = \frac{T_{ON}}{T}
\end{align}
Where $T_{ON}$ is the ON time of the wave and  T is the total time period of the wave. Power delivered to the load is proportional to T- ON time of the signal. This is used to control the current flowing through the heating element and also speed of the fan.
\begin{figure}
\centering
\includegraphics[width=0.5\linewidth]{pwm}
\caption{Pulse Width Modulation (A): On time is $90\%$ of the total time period,    (B): ON time is $10\%$ of total time period}
\end{figure} 
An internal timer of the microcontroller is used to generate a square wave. The ON time of the square wave depends on a count value. Hence, by varying this count value one can vary the width of the waveform. Therefore, by using this technique, PWM waveform is generated at the appropriate pin of the microcontroller. This generated PWM waveform is used to control the power delivered to the load (Fan and Heater). A MOSFET is used to switch at the PWM frequency which indirectly controls the power delivered to the load. A separate MOSFET is used for the two loads. The timer is operated at 244Hz.
\subsection{Analog to Digital conversion}
As explained earlier, the heat generated by the heater coil is passed to the iron plate through convection. We would now want to measure the temperature of this plate. For this purpose a temperature sensor AD590 is used.
Some of the salient features of AD590 include
\begin{enumerate}
\item Linear current output: 1\textmu A/K
\item Wide range: -55\textcelsius to +150\textcelsius
\item Sensor isolation from case
\item Low cost
\end{enumerate}
The output of AD590 is then fed to the microcontroller through an Instrumentation amplifier. The signal so obtained would be in analog form. It has to be converted in to digital form to make it understandable for the microcontroller. An ADC would be an obvious solution. ATmega16 features an 8-channel , 10 bit successive approximation ADC with 0-Vcc(0 to Vcc) input voltage range. An interrupt is generated on completion of analog to digital conversion. Here, ADC is initialize to have 206 $\mu$ sec of conversion time . Digital data so obtained is sent to computer via serial port as well as for further processing for on board display.
\section{Instrumentation amplifier}
Whenever there is a temperature measurement task at hand, instrumentation amplifiers are often used. A typical three Op-Amp Instrumentation amplifier is shown in the figure\ref{instramp}.
\begin{figure}
\centering
\includegraphics[width=0.6\linewidth]{Inst_Amp}
\caption{3 Op-Amp Instrumentation Amplifier}
\label{instramp}
\end{figure}
The instrumentation amplifiers (IA)have the advantage of very low DC offsets, high input impedance, very high Common mode rejection ratio (CMRR) etc. They are mostly prefered where the sensor is located at a remote place, susceptible to signal attenuation. The IA's have a very high input impedance and hence dose not loads the input signal source. IC LM348 is used to construct a 3 Op-Amp IA. IC LM348 contains a set of four Op-Amps. Gain of the amplifier is given by equation \ref{ia}
\begin{align}
\frac{V_o}{V_2-V_1}=\left\{1+\frac{2R_f}{R_g}\right\}\frac{R_2}{R_1}
\label{ia}
\end{align} 
The value of $R_g$ is kept variable to change the overall gain of the amplifier.
The signal generated by AD590 is in \textmu A/\textdegree K.  It is converted to mV/\textdegree K by taking it across a 1K\textohm  resistor. The  \textdegree K to\textdegree C conversion is done by subtracting 237 from the \textdegree K result. One input of the IA is fed with the mV/\textdegree K reading and the other with 273mV. The resulting output is now in mV/\textdegree C. The output of the IA is fed to microcontroller for further processing.
\section{Communication}
The set up has the facility to use either USB or RS232 for communication between set up and computer. A jumper is been provided to switch between USB and RS232.
\begin{figure}
\centering
\includegraphics[width=0.5\linewidth]{c.jpg}
\caption{Jumper arrangement}
\end{figure}
The voltages available at the TXD terminal of microcontroller are in TTL logic. The RS232 standard uses a different terminology. Voltage level below -5V is treated as logic 1 and voltage level above +5V is treated as logic 0. There are many reasons for RS232 using such terminology. One reason is to ensure error free transmission over long distances. For solving this compatibility issue an external hardware interface is required. IC MAX202 serves the purpose. IC MAX202 is a +5V RS232 transreceiver.
\subsection{Serial port communication}
Serial port is a full duplex device i.e. it can transmit and receive data at the same time.
\begin{figure}
\centering
\includegraphics[width=0.5\linewidth]{RS232cable.jpg}
\caption{RS232 cable}
\end{figure}
\begin{figure}
\centering
\includegraphics[width=0.5\linewidth]{Serialport.jpg}
\caption{Serial port}
\end{figure}
ATmega16 support a programmable Universal Synchronous and Asynchronous serial receiver and transmitter(USART) . Its baud rate is fixed at 9600 bps with character size set to 8 bits and parity bits disabled.
\subsection{Using USB for Communication}
After setting the jumper to USB mode connect the set up to the computer using a USB cable at appropriate ports as shown. To make the setup USB compatible USB to serial conversion is necessary. IC FT232R is used for this purpose.
\begin{figure}
\centering
\includegraphics[width=0.5\linewidth]{usbc.jpg}
\caption{USB communication}
\end{figure}
\begin{figure}
\centering
\includegraphics[width=0.5\linewidth]{usbp.jpg}
\caption{USB PORT}
\end{figure}
You need to have proper USB driver being installed on your computer.
\section{Display and Resetting the setup}
The temperature of the plate, percentage values of Heat and Fan and the machine identification number (MID) are displayed on LCD connected to the microcontroller.
As shown in figure, numerals below TEMP , indicate the actual temperature of the heater plate . Numeral below HEA and FAN indicate the respective percentage value at which heater and fan are being operated. Numerals below MID corresponds to the Device identification number.
\begin{figure}
\centering
\includegraphics[width=0.5\linewidth]{display.jpg}
\caption{Display}
\end{figure}

The set up could be reset at any time by pressing the reset button on it as shown in figure \ref{reset}. Reseting the setup takes it to the standby mode where the heater current is forced to be zero and fan speed to be the maximum value. Though these reset values are not displayed on the LCD display but they are loaded to the appropriate units. 
\begin{figure}
\centering
\includegraphics[width=0.5\linewidth]{r.jpg}
\caption{Reset}
\label{reset}
\end{figure}
\chapter{Using Scilab with Single Board Heater System}\label{sercomm}
This section explains the procedure to use Single Board Heater System with Scilab. An open loop experiment, step test is used for demonstrating this procedure. The process however remains the same for performing any other experiment explained in this document, unless specified. 
\section*{Hardware and Software Requirements}\label{umlauts}
For working with the Single Board Heater system, you would require the following:

\begin{enumerate}
\item SBHS with USB cable and power cable.
\item PC/Laptop with Scilab software installed (preferably the latest version). You may download it from:\\ {\tt http://www.scilab.org/products/scilab/download}
\item FTDI Virtual Com Port driver corresponding to the OS on your PC. (You may download it from: 
{\tt http://www.ftdichip.com/Drivers/VCP.htm})
\item Example StepTest provided along with this document.
\end{enumerate}
\section{Accessing Single Board Heater system on a Windows System}
\label{win_procedure}
This section deals with the procedure to use SBHS on a windows Operating System. The Operating System used for this document is Windows 7, 64-bit OS. In case if you are using some other Operating System or the steps explianed here are not sufficient to understand, you can refer to the official document available on  the main ftdi website. For doing so, go to www.ftdichip.com. On the left hand side panel, click on 'Drivers'. In the dropdown menu, choose 'VCP Drivers'. Then on the web page page, click on 'Installation Guides link'. Choose the required OS document. We would now begin with the procedure.
\subsection{Installing necessary driver and COM Port Settings}
After powering ON the SBHS and plugging in the USB cable to the PC (making sure you check the jumper settings on the board) for the very first time, the {\tt Welcome to Found New Hardware Wizard} dialog box pops up. You have to choose the option {\tt Install from a list or specific location}. Choose {\tt Search for best driver in these locations}. Check the button {\tt Include this location in the search}. Click on {\tt Browse}. Specify the path where you have copied the driver (item no.3) and install it by clicking {\tt Next}. Once the wizard has successfully installed the driver, your SBHS is ready for use. Please note that this procedure has to be repeated twice.

Now, we would check the communication port number assigned to the computer port to which you connect the Single Board Heater System, via an RS232 or USB cable.  For checking the port number, right click on My Computer and click on properties. Here, select the hardware tab and then click on Device Manager. You would see the list of hard ware devices. Look for Ports(COM \& LPT) option and open it. You would see the various communication port your computer is using. If you have connected RS232 cable, then look for Communications Port(COM1) else look for USB Serial Port. For RS232 connection, the port number mostly remains COM1. For USB connection it may change to some other number. Note the appropriate COM number. This process is illustrated in figure \ref{com_number}
\begin{figure}
\centering
\includegraphics[width=0.7\linewidth]{COM.jpg}
\caption{Checking Communication Port number}
\label{com_number}
\end{figure}

Sometimes the COM port number you get after connecting a USB cable is more then single digit number 9. Since the serial tool box can handle only single digit port number, it is necessary to change your COM port number. Following is the procedure to do the same.
After following the procedure for knowing your com port number described above, double click that particular port. Click on Port Settings tab and then click on Advanced. In the COM port number dropdown menu, choose the port number to any other number less than 10. This procedure is illustrated in figure \ref{com_change}
\begin{figure}
\centering
\includegraphics[width=0.7\linewidth]{port2.jpg}
\caption{Changing Com port number}
\label{com_change}
\end{figure}
\subsection{Configuring Scilab}
\label{scilab_sbhs}
Launch Scilab from start menu or double click the scilab icon on the desktop. Before executing any scripts which are dependent on other files or scripts, one has to change the working directory of scilab. Doing so will tell scilab from where it has to load the other necessary files. If you have other files saved to any other directory, you have to say {\ttfamily  getd<space>folder\_path} in the scilab console. Type {\ttfamily ls} to see the if the files are available. Here the directory is changed to the place where the relevent files for performiong step test resides. To change the directory, click on file menu and then choose "Change directory". You can also change the directory by typing {\tt cd<space>folder path}. Next, click on {\ttfamily editor} from the menu bar to open the scilab editor or simply type {\ttfamily editor} in the scilab console and open the file {\ttfamily ser\_init.sce}. Change the port number (the first argument of the {\tt openserial()} function) to the COM port number that you have noted down before. The second argument of the {\tt openserial()} function requires baud rate, parity, data bits and stop bits as a string.You should give it as {\tt "9600,n,8"}. Since stop bit is zero in our case, omit the parameter from the string to indicate it as zero. Execute this .sce file. You will get a message {\tt COM Port Opened}. If it complains, reconnecting the USB cable and/or restarting Scilab may help. Now execute the {\tt step\_test.sci} file.  The results are illustrated in figure \ref{loader}. 

\begin{figure}
\centering
\includegraphics[width=0.7\linewidth]{console.png}
\caption{Expected responses seen on the console}
\label{loader}
\end{figure}


\begin{figure}
\centering
\includegraphics[width=0.7\linewidth]{scilab1.jpg}
\caption{Executing script files}
\label{exec}
\end{figure}


Type {\ttfamily Xcos} on the scilab console or click on Applications and choose {\ttfamily Xcos} to open Xcos environment. Load the {\ttfamily step\_test.xcos} file from the File menu. The Xcos interface that will open is as shown in figure \ref{Xcosintr}. You can set the block parameters by double clicking on the block as shown in figure \ref{blk_parameters}. To run the code click on Simulation menu and choose start. After running Xcos successfully you would see the plots as shown in figure \ref{plots}. See that the values of fan and heater you input to the xcos file is getting reflected on the board display. To stop the experiment click on the {\ttfamily stop} option on the menu bar of the Xcos environment.
\begin{figure}
\centering
\includegraphics[width=0.7\linewidth]{xcos.png}
\caption{Xcos Interface}
\label{Xcosintr}
\end{figure}

\begin{figure}
\centering
\includegraphics[width=0.7\linewidth]{xcos_block.png}
\caption{Setting Block Parameters}
\label{blk_parameters}
\end{figure}


\begin{figure}
\centering
\includegraphics[width=0.7\linewidth]{plot.png}
\caption{Plot obtained after executing step\_test.xcos}
\label{plots}
\end{figure}

\section{Accessing Single Board Heater system on a Linux System}\label{linux_scilab}
This section deals with the procedure to use SBHS on a Linux Operating System. The Operating System used for this document is Ubuntu 10.04.
For Linux users, the instructions given in section \ref{win_procedure} hold true with a few changes as below:

You do not require FTDI COM port drivers for connecting your SBHS to the PC. After plugging in the USB cable to your PC, check the serial port number by typing {\tt ls /dev/ttyUSB*} on the terminal, refer Fig.\ref{lstty}. Usually, the highest numbered one will be your device port number. eg:- {\tt /dev/ttyUSB0}. If you want to connect more than one USB device, then type {\tt tail -f /var/log/messages|grep ttyUSB} on the linux terminal just before plugging in the individual USB cable, refer Fig.\ref{tailusb}. The USB number will then be shown on the screen. Press {\tt Ctrl+ c} to abort the command. 
\begin{figure}
\centering
\includegraphics[width=0.7\linewidth]{lstty.png}
\caption{Checking the port number in linux (1)}
\label{lstty}
\end{figure}
\begin{figure}
\centering
\includegraphics[width=0.7\linewidth]{tailusb.png}
\caption{Checking the port number in linux (2)}
\label{tailusb}
\end{figure}

Note this number and change the port number (the first argument of the \\ {\tt openserial()} function) in the {\tt ser\_init.sce} file with it. The second argument of the {\tt openserial()} function should be {\tt "9800,n,8,0"}, refer Fig.\ref{linuxserial}. This defines baud rate, parity, data bits and stop bits in that order. It has been found that if we omit the last parameter i.e., stop bits instead of specifying it as zero, scilab gives an error. Execute this file. Once the serial port initialisation is succesfully done, you get a message as shown in Fig.\ref{console_linux}. If it complains, reconnecting the USB cable and/or restarting Scilab may help.
\begin{figure}
\centering
\includegraphics[width=\linewidth]{linuxserial.png}
\caption{Configuring port number and other parameters}
\label{linuxserial}
\end{figure}


\begin{figure}
\centering
\includegraphics[width=0.7\linewidth]{SSscilab.png}
\caption{Scilab Console Message after Opening Serial Port}
\label{console_linux}
\end{figure}

Now execute the {\tt step\_test.sci} file.  The results are illustrated in figure \ref{loader}. Type {\ttfamily Xcos} on the scilab console or click on Applications and choose {\ttfamily Xcos} to open Xcos environment. Load the {\ttfamily step\_test.xcos} file from the File menu. The Xcos interface that will open is as shown in figure \ref{Xcosintr}. You can set the block parameters by double clicking on the block as shown in figure \ref{blk_parameters}. To run the code click on Simulation menu and choose start. After running Xcos successfully you would see the plots as shown in figure \ref{plots}. See that the values of fan and heater you input to the xcos file is getting reflected on the board display. To stop the experiment click on the {\ttfamily stop} option on the menu bar of the Xcos environment.
%Follow the instructions as given for Windows OS in section \ref{scilab_sbhs} for executing an example step test. This has been tested in Ubuntu Linux 10.04.


%\chapter{Using Single Board Heater System, Virtually!}
\section{Introduction to Virtual Labs at \\IIT Bombay}
The concept of virtual laboratory is a brilliant step towards strengthening the education system of an university/college, a metropolitan area or even an entire nation. The idea is to use the ICT i.e. Information and Communications Technology, mainly the Internet for imparting education or exchange of educational information. Virtual Laboratory mainly focuses on providing the laboratory facility, virtually. Various experimental set-ups are hooked up to the internet and made available to use for the external world. Hence, anybody can connect to that equipment over the internet and carry out various experiments pertaining to it. The beauty of this idea is that a college who cannot afford to have some experimental equipments can still provide laboratory support to their students through virtual lab, and all that will cost it is a fair Internet connection! Moreover, the laboratory work does not ends with the college hours, one can always use the virtual lab at any time and at any place assuming the availability of an internet connection. 

A virtual laboratory for SBHS is launched at IIT Bombay. Here is the url to access it: {\bf http://www.co-learn.in/web\_sbhs/}. A set of 15 SBHS are made available to use over the internet 24$\times$ 7. These individual kits are made available to the users on hourly basis. We have a slot booking mechanism to achieve this. Since there are 15 SBHS connected with an hours slot for 24 hrs a day, we have 360 one hour slots a day. This means that 360 individual users can access the SBHS in a day for an hour. This also means that up to 2520 users can use the SBHS for an hour in a week and more than 10,000 in a month! A web page is hosted which is the first interface to the user. The user registers/logs in himself/herself here. The user is also supposed to book a slot for accessing the SBHS. A database server maintains a record of the data generated through the web interface. A python script is hosted on the server side and it helps in connecting the user with the corresponding SBHS placed remotely. The client is also suppose to run a java client on his computer. A free and open source scientific computing Software, Scilab, is used by the user for implementing the experiment on SBHS, in terms of simple Scilab coding. 
%\section{Evolution of SBHS virtual labs}
In \cite{vlabs-kmm}, 
the control algorithm is implemented at the server end and the remote
student just keys in the parameters, as shown in Figure
\ref{fig:initial}. 
\begin{figure}
\includegraphics[width=\linewidth]{IEEE-Chile/figures/vlab-1.png}
\caption{SBHS virtual laboratory with remote access using LabVIEW}
\label{fig:initial}
\end{figure}
LabVIEW was used for the implementation of the same. The
server end consisted of a computer connected with an SBHS with a full
blown copy of LabVIEW installed on it. The client has a LabVIEW run
time engine available for free download from the National Instruments
website.  A few
LabVIEW algorithms/experiments were hosted on the server. The client
accesses these algorithm/experiment over the Internet using a web
browser by entering appropriate parameters.

It was realized that the learning experience is not complete for this
structure. This is because the server hosts some pre-built LabVIEW
algorithms and a user can only access these few algorithms. The user
can in no way change the program and can only input experimental
parameters. 
Hence, we came up with a new architecture
as shown in the Figure \ref{fig:second} that used full blown copies of
LabVIEW at both server and client ends.  
\begin{figure}
\includegraphics[width=\linewidth]{IEEE-Chile/figures/vlab-2.png}
\caption{SBHS virtual laboratory with remote access and live data sharing using LabVIEW}
\label{fig:second}
\end{figure}
 
 This idea uses the DataSocket technology of LabVIEW. Since now the
 client is having a complete LabVIEW installation on his/her computer
 she can now implement her own algorithms.  Thus this architecture did
 provide a complete learning experience to the students.  There are
 some shortcomings as well:

\begin{itemize}
\item LabVIEW is expensive and students may not be able to afford to
  buy it.  It is also prohibitively expensive for the Government to
  distribute it.

\item We used the LabVIEW version 8.04, which had restricted scripting
  language.  It was tedious to create new control algorithms in it.
\end{itemize}
\begin{figure}
\includegraphics[width=\linewidth]{IEEE-Chile/figures/vlab-3.png}
\caption{SBHS virtual laboratory using open source software}
\label{fig:third}
\end{figure}
This made us shift to free and open source (FOSS) software. We
replaced LabVIEW with Java and Scilab as shown in Figure
\ref{fig:third}. Scilab at the server end is used for communicating
with SBHS. Scilab at the client end is used for implementing the
algorithms. Java is used at both the server as well as client end for
communication over the Internet thereby connecting the client with the
server. 

For the above solution, we need a dedicated copy of scilab running at
the server end for every SBHS. One way to do this is to host it on
multiple computers with unique IPs. Hence the number of SBHS we want
to host requires as many computer's and public IPs thereby making
it expensive. Moreover, it also limits its scalability. The other way
to do this is to host multiple java and scilab servers on the same
computer.  Hosting many copies of Scilab simultaneously requires a
powerful computer for the server.

For these reasons  we decided to take scilab off the server computer
and to use java alone to communicate with the SBHS directly.  Java
also 
communicates with the client computer.  We connected seven SBHS
systems to a USB port through a serial port hub.  This architecture
was 
implemented on a Windows Operating System.  We faced the following
difficulties in this solution.
\begin{itemize}
\item When we connected more than one serial hub to a PC, the port ID
  could not be retrieved correctly.  Port ID information is required
  if we want a student to use the same SBHS for all their experiments
  during different sessions.
\item The experiments required time stamping of the data communicated
  to and from the server. But this time stamping was not linear and
  suffered instability.  
\end{itemize}%
This made us to completely switch to FOSS with Ubuntu Linux as the OS
and is the current structure of the Virtual lab as shown in Figure
\ref{fig:detail-arch} 
\section{Current Architecture}
\label{sec:vlabarchi}
\begin{figure}
\centering
\includegraphics[width=\linewidth]{IEEE-Chile/figures/vlab-arch}
\caption{Virtual control lab hardware architecture}
\label{fig:hw-arch}
\end{figure}

\subsection{Hardware}
The architecture of the virtual single-board heater system lab as
shown in Figure \ref{fig:hw-arch} involves 7 single-board heater
systems connected to the server via a 7-port USB hub. The server
computer is connected to a high speed internetwork and has enough
processing capability to host data acquisition, database, and web
servers. The internetwork connected client computer needs only the
Java runtime engine and Scilab installation.  

A similar architecture but replacing Java with python and with 15 units connected to the server via two 7-port hubs has been successfully tested on intranet for the undergraduate Process Control course and the graduate Digital Control
and Embedded systems courses conducted at IIT Bombay as well as few workshops over the internet. Currently, this architecture is integrated with a cameras on each SBHS to facilitate live video streaming. This facility will be extended to all the units in future. This gives the user a feel of remote hands-on. Work on this camera facility is in progress.


\subsection{Software}
The current software architecture of this virtual control lab is shown
in Figure \ref{fig:detail-arch}. The server computer runs on Ubuntu
Linux 10.04 OS. It hosts a LAMP (Linux-Apache-MySQL-PHP) server. The
MySQL-based database server has the details of all the registered
users, their slot details, authentication keys to allow remote access,
etc. It also hosts a PHP based web server shown in Figure
\ref{fig:sbhs-website} that has pages for registration, login and slot
booking \cite{vl010}.  For communication with the device, Django - a Python based web framework is used on the Server side. On the client end has control
algorithms running in Scilab and communicates over the Internet
through the python client.

\begin{figure}
\includegraphics[width=\linewidth]{IEEE-Chile/figures/webpage}
\caption{Home page of SBHS V Labs}
\label{fig:sbhs-website}
\end{figure}
The steps to be performed before and during each experiment are explained next.

\begin{figure}
\centering
\includegraphics[width=\linewidth]{IEEE-Chile/figures/blk-dig.pdf}
\caption{Current Architecture of SBHS Virtual Labs}
\label{fig:detail-arch}
\end{figure}



\subsubsection{Registration}
A client willing to perform the experiments needs to register with us
by entering the personal details on the Registration page. This sends
an account activation link to the client’s mailbox. Upon clicking the
link, the account gets activated.
\subsubsection {Slot booking} The client can now login and book slots
to perform the experiments. Each slot lasts for an hour with 55
minutes for experimentation and 5 mins for resetting the setup. A
client can book up to 2 slots, per day, in advance. Besides this, if
the current slot is empty, it can be booked as a free slot. For each
slot being booked, the client gets a port number and an access
key. The interface depicting this is shown in Figure
\ref{fig:slot-booking}.
\begin{figure}
\includegraphics[width=\linewidth]{IEEE-Chile/figures/slot-book.png}
\caption{Screenshot of Slot Booking Page}
\label{fig:slot-booking}
\end{figure}
\subsubsection {Port number} In order to maintain consistency in
performing the experiments, each client is alloted the same setup
every time s/he requests for the slot. To facilitate this, each setup
has been alloted a machine I.D. For example, if the machine I.D. of
the unit is 5, it is displayed as {\tt SBHS Number: 5}. This
eliminates time and effort spent in plant modelling, each time, due to
change in the setup.

\subsubsection { Starting the Python client} On windows, after installing python, it will not be readily recognized by the OS. To make windows interpret python as a command, do the following steps.
\begin{enumerate}
\item Right click on My Computer and click on properties
\item Click on the Advanced tab.Now click on the Environment Variables button
In the system variable section, locate and click on “path”.
\item Click the edit button below it. You should see a “Edit system variable” window.
\item In the “Variable value” field, take the cursor to the last.
Type a semicolon and type C:$\backslash$Python27
\end{enumerate}
The experiment folder has a run.bat file, settings.txt file and a sbhsclient.py file. Open the settings.txt file and make the required changes. Beware not to make any unneccesary changes to this file. The example settings are anyways given in the end of this file. Linux users should bypass the terminal proxy settings, if any, by executing the command {\tt export http\_proxy=''}. Be informed that this is just a temporay disable of proxy and it should be done everytime you open a fresh terminal. Windows users simply double click on the run.bat file. It will open up a command prompt. Linux users instead of executing the run.bat file will type the following command on the terminal {\tt python sbhsclient.py}. This activity will ask you for the username and password. If the settings are not proper in the settings.txt file, you will get a {\tt Connection error message}. Contact the system administrator of your network to get the correct values of the parameters. Once you are done typing your username and password, you will get a Login Successfull message. In case if you are logging in at a time other that your booked slot, it will tell you that No slot is found. Else, it will notify that you have booked a slot and can now run the scilab code. It will also create {\tt scilabread.sce, scilabwrite.sce} and a log file in the experiment folder. The log file name will be in the format {\tt date\_month\_year\_hrs\_mins\_seconds.txt}. This unique file name will help you track the log file of you choice in the future. 
\begin{figure}
  \centering
\includegraphics[width=0.7\linewidth]{IEEE-Chile/figures/display.jpg}
\caption{Screenshot of Video Stream}
\label{video}
\end{figure}
\subsubsection { Video streaming} At present, the facility to provide video streaming for all users is in progress. The video
streaming of the capture display will be something as shown in Figure \ref{video}. The clients with lesser bandwidth will have an option to refrain from video streaming. This facility will be available for logged in users on the website.

\subsubsection { Executing the Scilab code} The client needs to
download the scilab code from the SBHS home page. This is available
under {\tt Downloads}. Various other experiments are also available on {\tt fossee.in/moodle} The client can modify this code to implement
his/her own algorithm. S/he should not edit the part of the code which
is used for communicating with python. This part is clearly marked in
the code. Refer the chapter specific to which experiment you want to run. It will explain you which codes and in what order they should be executed.

\subsubsection { File handling} This subsection explains the
significance of scilabread and scilabwrite files. During the
experiment, the client writes heater and fan inputs to the
scilabwrite.sce file. These values are read by the Python client. It writes iteration and client departure time stamp and send it over the network. Python server at the other end reads these values, puts a server arrival time stamp and gives them to the SBHS through its data acquisition
interface. After feeding the values, it reads the temperature of the
heated plate, puts a server departure time stamp and sends it to the Python client. The Python client puts a client arrival time stamp and then writes all these values i.e. heater, fan and temperature along with the
timestamps to the scilabread.sce file. Scilab client reads the latest
temperature value and does the calculation of heater and fan inputs in
accordance to the logic developed for the experiment. The live data
streaming involves heater and fan inputs being sent by the client and
temperature response in turn being sent by the server. The timestamp accompanying the data could be used for real-time
control of the setup.

\subsubsection { Concluding the experiment} Once the client is done
with the experiment, s/he can use the log file created in the experiment folder for analysis purpose. If you happen to lose the log file, you can download it from the website, after you login, by clicking on {\tt Download} link.
%\end{enumerate}

\subsection{Other Implementation Issues}
%\begin{enumerate}
\subsubsection { Mapping of machine id with the USB port number}
Whenever a SBHS unit is plugged in to a USB port, the dev id (/dev/ttyUSB*) assigned to it by OS is different. A script is run which creates a table of the mapping between the machine I.D. and the USB dev id. This is done to ensure that the client gets the same machine I.D. i.e. the same SBHS each time. 

\subsubsection { Auto log off problem}
In some ISPs, the network gets disconnected if it is inactive for some time. To handle this situation, we are running a shell script which checks for internet connectivity by pingging google periodically. If the network is down then it will try to reconnect.  


%\end{enumerate}


\subsection{Support}
The SBHS support activities are being funded by National Mission on Education through Information and Communication Technology \cite{kmm010}.  The major support activities include conducting workshops in several colleges across the country, active discussion through Fossee moodle and spoken tutorials for self-learning.

\subsubsection{Workshops}
The SBHS team has been conducting workshops to popularise the utility of SBHS.  The course content is around local and remote access of the SBHS. More than 50 college teachers from several Engineering colleges of the country have attended the workshops and have started using the setup for the relevant courses running in their colleges.

\subsubsection{Fossee moodle}


Fossee moodle is a web interface that allows discussions, post queries, upload files, upload grades, etc.  Also, there is a facility to provide e-mail notification whenever there is a post on the course website. Thus, such an interface allows to address common problems of many users at the same time. It also provides a platform for the users at different locations to share their experiences during the experiments.  The codes and manuals for about 8 experiments are available at \cite{moodle}.


\begin{figure}
\centering
\includegraphics[width=0.75\linewidth]{IEEE-Chile/figures/fossee-2}
\caption{Fossee moodle webpage to support SBHS activities}
\label{fig:fossee-sup}
\end{figure}



\subsubsection{Spoken tutorials}
A Spoken tutorial is a screen capture along with the audio that could be used to teach any computer application.  Currently, the spoken tutorials for various Free and Open source softwares are available in different Indian languages at \cite{spoken-sci}.  The scripts and tutorials for SBHS are in pipeline.





\section{Evolution of SBHS virtual labs}
In \cite{vlabs-kmm}, 
the control algorithm is implemented at the server end and the remote
student just keys in the parameters, as shown in Figure
\ref{fig:initial}. 
\begin{figure}
\includegraphics[width=\linewidth]{IEEE-Chile/figures/vlab-1.png}
\caption{SBHS virtual laboratory with remote access using LabVIEW}
\label{fig:initial}
\end{figure}
LabVIEW was used for the implementation of the same. The
server end consisted of a computer connected with an SBHS with a full
blown copy of LabVIEW installed on it. The client has a LabVIEW run
time engine available for free download from the National Instruments
website.  A few
LabVIEW algorithms/experiments were hosted on the server. The client
accesses these algorithm/experiment over the Internet using a web
browser by entering appropriate parameters.

It was realized that the learning experience is not complete for this
structure. This is because the server hosts some pre-built LabVIEW
algorithms and a user can only access these few algorithms. The user
can in no way change the program and can only input experimental
parameters. 
Hence, we came up with a new architecture
as shown in the Figure \ref{fig:second} that used full blown copies of
LabVIEW at both server and client ends.  
\begin{figure}
\includegraphics[width=\linewidth]{IEEE-Chile/figures/vlab-2.png}
\caption{SBHS virtual laboratory with remote access and live data sharing using LabVIEW}
\label{fig:second}
\end{figure}
 
 This idea uses the DataSocket technology of LabVIEW. Since now the
 client is having a complete LabVIEW installation on his/her computer
 she can now implement her own algorithms.  Thus this architecture did
 provide a complete learning experience to the students.  There are
 some shortcomings as well:

\begin{itemize}
\item LabVIEW is expensive and students may not be able to afford to
  buy it.  It is also prohibitively expensive for the Government to
  distribute it.

\item We used the LabVIEW version 8.04, which had restricted scripting
  language.  It was tedious to create new control algorithms in it.
\end{itemize}
\begin{figure}
\includegraphics[width=\linewidth]{IEEE-Chile/figures/vlab-3.png}
\caption{SBHS virtual laboratory using open source software}
\label{fig:third}
\end{figure}
This made us shift to free and open source (FOSS) software. We
replaced LabVIEW with Java and Scilab as shown in Figure
\ref{fig:third}. Scilab at the server end is used for communicating
with SBHS. Scilab at the client end is used for implementing the
algorithms. Java is used at both the server as well as client end for
communication over the Internet thereby connecting the client with the
server. 

For the above solution, we need a dedicated copy of scilab running at
the server end for every SBHS. One way to do this is to host it on
multiple computers with unique IPs. Hence the number of SBHS we want
to host requires as many computer's and public IPs thereby making
it expensive. Moreover, it also limits its scalability. The other way
to do this is to host multiple java and scilab servers on the same
computer.  Hosting many copies of Scilab simultaneously requires a
powerful computer for the server.

For these reasons  we decided to take scilab off the server computer
and to use java alone to communicate with the SBHS directly.  Java
also 
communicates with the client computer.  We connected seven SBHS
systems to a USB port through a serial port hub.  This architecture
was 
implemented on a Windows Operating System.  We faced the following
difficulties in this solution.
\begin{itemize}
\item When we connected more than one serial hub to a PC, the port ID
  could not be retrieved correctly.  Port ID information is required
  if we want a student to use the same SBHS for all their experiments
  during different sessions.
\item The experiments required time stamping of the data communicated
  to and from the server. But this time stamping was not linear and
  suffered instability.  
\end{itemize}%
This made us to completely switch to FOSS with Ubuntu Linux as the OS
and is the current structure of the Virtual lab as shown in Figure
\ref{fig:detail-arch} 
\section{Current Architecture}
\label{sec:vlabarchi}
\begin{figure}
\centering
\includegraphics[width=\linewidth]{IEEE-Chile/figures/vlab-arch}
\caption{Virtual control lab hardware architecture}
\label{fig:hw-arch}
\end{figure}

\subsection{Hardware}
The architecture of the virtual single-board heater system lab as
shown in Figure \ref{fig:hw-arch} involves 7 single-board heater
systems connected to the server via a 7-port USB hub. The server
computer is connected to a high speed internetwork and has enough
processing capability to host data acquisition, database, and web
servers. The internetwork connected client computer needs only the
Java runtime engine and Scilab installation.  

A similar architecture but replacing Java with python and with 15 units connected to the server via two 7-port hubs has been successfully tested on intranet for the undergraduate Process Control course and the graduate Digital Control
and Embedded systems courses conducted at IIT Bombay as well as few workshops over the internet. Currently, this architecture is integrated with a cameras on each SBHS to facilitate live video streaming. This facility will be extended to all the units in future. This gives the user a feel of remote hands-on. Work on this camera facility is in progress.


\subsection{Software}
The current software architecture of this virtual control lab is shown
in Figure \ref{fig:detail-arch}. The server computer runs on Ubuntu
Linux 10.04 OS. It hosts a LAMP (Linux-Apache-MySQL-PHP) server. The
MySQL-based database server has the details of all the registered
users, their slot details, authentication keys to allow remote access,
etc. It also hosts a PHP based web server shown in Figure
\ref{fig:sbhs-website} that has pages for registration, login and slot
booking \cite{vl010}.  For communication with the device, Django - a Python based web framework is used on the Server side. On the client end has control
algorithms running in Scilab and communicates over the Internet
through the python client.

\begin{figure}
\includegraphics[width=\linewidth]{IEEE-Chile/figures/webpage}
\caption{Home page of SBHS V Labs}
\label{fig:sbhs-website}
\end{figure}
The steps to be performed before and during each experiment are explained next.

\begin{figure}
\centering
\includegraphics[width=\linewidth]{IEEE-Chile/figures/blk-dig.pdf}
\caption{Current Architecture of SBHS Virtual Labs}
\label{fig:detail-arch}
\end{figure}



\subsubsection{Registration}
A client willing to perform the experiments needs to register with us
by entering the personal details on the Registration page. This sends
an account activation link to the client’s mailbox. Upon clicking the
link, the account gets activated.
\subsubsection {Slot booking} The client can now login and book slots
to perform the experiments. Each slot lasts for an hour with 55
minutes for experimentation and 5 mins for resetting the setup. A
client can book up to 2 slots, per day, in advance. Besides this, if
the current slot is empty, it can be booked as a free slot. For each
slot being booked, the client gets a port number and an access
key. The interface depicting this is shown in Figure
\ref{fig:slot-booking}.
\begin{figure}
\includegraphics[width=\linewidth]{IEEE-Chile/figures/slot-book.png}
\caption{Screenshot of Slot Booking Page}
\label{fig:slot-booking}
\end{figure}
\subsubsection {Port number} In order to maintain consistency in
performing the experiments, each client is alloted the same setup
every time s/he requests for the slot. To facilitate this, each setup
has been alloted a machine I.D. For example, if the machine I.D. of
the unit is 5, it is displayed as {\tt SBHS Number: 5}. This
eliminates time and effort spent in plant modelling, each time, due to
change in the setup.

\subsubsection { Starting the Python client} On windows, after installing python, it will not be readily recognized by the OS. To make windows interpret python as a command, do the following steps.
\begin{enumerate}
\item Right click on My Computer and click on properties
\item Click on the Advanced tab.Now click on the Environment Variables button
In the system variable section, locate and click on “path”.
\item Click the edit button below it. You should see a “Edit system variable” window.
\item In the “Variable value” field, take the cursor to the last.
Type a semicolon and type C:$\backslash$Python27
\end{enumerate}
The experiment folder has a run.bat file, settings.txt file and a sbhsclient.py file. Open the settings.txt file and make the required changes. Beware not to make any unneccesary changes to this file. The example settings are anyways given in the end of this file. Linux users should bypass the terminal proxy settings, if any, by executing the command {\tt export http\_proxy=''}. Be informed that this is just a temporay disable of proxy and it should be done everytime you open a fresh terminal. Windows users simply double click on the run.bat file. It will open up a command prompt. Linux users instead of executing the run.bat file will type the following command on the terminal {\tt python sbhsclient.py}. This activity will ask you for the username and password. If the settings are not proper in the settings.txt file, you will get a {\tt Connection error message}. Contact the system administrator of your network to get the correct values of the parameters. Once you are done typing your username and password, you will get a Login Successfull message. In case if you are logging in at a time other that your booked slot, it will tell you that No slot is found. Else, it will notify that you have booked a slot and can now run the scilab code. It will also create {\tt scilabread.sce, scilabwrite.sce} and a log file in the experiment folder. The log file name will be in the format {\tt date\_month\_year\_hrs\_mins\_seconds.txt}. This unique file name will help you track the log file of you choice in the future. 
\begin{figure}
  \centering
\includegraphics[width=0.7\linewidth]{IEEE-Chile/figures/display.jpg}
\caption{Screenshot of Video Stream}
\label{video}
\end{figure}
\subsubsection { Video streaming} At present, the facility to provide video streaming for all users is in progress. The video
streaming of the capture display will be something as shown in Figure \ref{video}. The clients with lesser bandwidth will have an option to refrain from video streaming. This facility will be available for logged in users on the website.

\subsubsection { Executing the Scilab code} The client needs to
download the scilab code from the SBHS home page. This is available
under {\tt Downloads}. Various other experiments are also available on {\tt fossee.in/moodle} The client can modify this code to implement
his/her own algorithm. S/he should not edit the part of the code which
is used for communicating with python. This part is clearly marked in
the code. Refer the chapter specific to which experiment you want to run. It will explain you which codes and in what order they should be executed.

\subsubsection { File handling} This subsection explains the
significance of scilabread and scilabwrite files. During the
experiment, the client writes heater and fan inputs to the
scilabwrite.sce file. These values are read by the Python client. It writes iteration and client departure time stamp and send it over the network. Python server at the other end reads these values, puts a server arrival time stamp and gives them to the SBHS through its data acquisition
interface. After feeding the values, it reads the temperature of the
heated plate, puts a server departure time stamp and sends it to the Python client. The Python client puts a client arrival time stamp and then writes all these values i.e. heater, fan and temperature along with the
timestamps to the scilabread.sce file. Scilab client reads the latest
temperature value and does the calculation of heater and fan inputs in
accordance to the logic developed for the experiment. The live data
streaming involves heater and fan inputs being sent by the client and
temperature response in turn being sent by the server. The timestamp accompanying the data could be used for real-time
control of the setup.

\subsubsection { Concluding the experiment} Once the client is done
with the experiment, s/he can use the log file created in the experiment folder for analysis purpose. If you happen to lose the log file, you can download it from the website, after you login, by clicking on {\tt Download} link.
%\end{enumerate}

\subsection{Other Implementation Issues}
%\begin{enumerate}
\subsubsection { Mapping of machine id with the USB port number}
Whenever a SBHS unit is plugged in to a USB port, the dev id (/dev/ttyUSB*) assigned to it by OS is different. A script is run which creates a table of the mapping between the machine I.D. and the USB dev id. This is done to ensure that the client gets the same machine I.D. i.e. the same SBHS each time. 

\subsubsection { Auto log off problem}
In some ISPs, the network gets disconnected if it is inactive for some time. To handle this situation, we are running a shell script which checks for internet connectivity by pingging google periodically. If the network is down then it will try to reconnect.  


%\end{enumerate}


\subsection{Support}
The SBHS support activities are being funded by National Mission on Education through Information and Communication Technology \cite{kmm010}.  The major support activities include conducting workshops in several colleges across the country, active discussion through Fossee moodle and spoken tutorials for self-learning.

\subsubsection{Workshops}
The SBHS team has been conducting workshops to popularise the utility of SBHS.  The course content is around local and remote access of the SBHS. More than 50 college teachers from several Engineering colleges of the country have attended the workshops and have started using the setup for the relevant courses running in their colleges.

\subsubsection{Fossee moodle}


Fossee moodle is a web interface that allows discussions, post queries, upload files, upload grades, etc.  Also, there is a facility to provide e-mail notification whenever there is a post on the course website. Thus, such an interface allows to address common problems of many users at the same time. It also provides a platform for the users at different locations to share their experiences during the experiments.  The codes and manuals for about 8 experiments are available at \cite{moodle}.


\begin{figure}
\centering
\includegraphics[width=0.75\linewidth]{IEEE-Chile/figures/fossee-2}
\caption{Fossee moodle webpage to support SBHS activities}
\label{fig:fossee-sup}
\end{figure}



\subsubsection{Spoken tutorials}
A Spoken tutorial is a screen capture along with the audio that could be used to teach any computer application.  Currently, the spoken tutorials for various Free and Open source softwares are available in different Indian languages at \cite{spoken-sci}.  The scripts and tutorials for SBHS are in pipeline.





\section{Conducting experiments using the Virtual lab}\label{vlabsexpt}
In this section we will see the steps involved in conducting experiments using virtual lab.
%In this section we will see the details of the building blocks of virtual lab. Figure \ref{block_dig} shows the block diagram for the same. A virtual lab is similar to a remote lab where in communication takes place between two computers over the internet. The server computer consists of a web server, a data base server and a java application sever. The client computer should contain a web browser, Scilab Software and a java client. A web server is hosted using apache software. The web pages are written in php language and are hosted on this server. Some of these php pages are necessarily linked to a data base server. The Database Server is hosted using mysql. It keeps a record of the information generated during a person registers and books a slot. The java server communicates with the SBHS connected to the computer on a USB port. It also accesses the data base and communicates with the java client. 
%
%
%%\begin{figure}
%%\centering
%%\includegraphics[width=\linewidth]{vlabs/block_dig.pdf}
%%\caption{Block diagram of the Virtual lab setup}
%%\label{block_dig}
%%\end{figure}
%On the client side, the web browser is any internet browser available. The user visits the specified web page, logs in/ registers and book/view/delete a slot(s) using the web brwser available to him. The java client is a simple java application provided by us. The user keys in the information provided on the web page pertaining to the booked slot in to this java client. The interface also provides an option to view a live streaming of the display of the SBHS to which he has connected to. This facility is however limitied to one SBHS for now and an elegent method is currently under development to accomodate all of the SBHS. The Scilab software must be installed by the client on his computer. Here, he will write the experiment he wants to conduct on the SBHS in simple scilab language. A primary scilab code will be provided and the process of modification of the code will be explained. 
This section assumes that the necessary files and software are available on the users computer. Files can be downloaded from the vlabs home page {\bf http://www.co-learn.in/web\_sbhs/} as shown in fig \ref{fig:sbhs-website}
\begin{enumerate}
\item The user goes to the vlabs home page and if he is a first time visitor, he registers himself and follows the information provided there, else he directly logs in.
\item After logging in, the user will book a slot.
\item In the experiment folder the settings.txt file will be edited as per requirement. The client should take care not to make any unneccesary changes to this file. The example settings are anyways given in the end of this file.

\item The user will now start the python client. On windows OS, double clicking on the run.bat file will open the command prompt. On linux OS, one has to first go to the directory where the experiment files (.sce and .py) are kept. This can be done by typing {\tt cd<space>directory name} on the terminal. Linux users should also bypass the terminal proxy settings, if any, by executing the command {\tt export http proxy=’’} on the terminal. Be informed that this is just a temporay disable of terminal proxy and it should be done everytime you open a fresh terminal. Linux user should now type the command {\tt python sbhsclient.py} on the linux  terminal. 
\item The python client will first do some network checks. If it finds the network to be NOT ok for experimentation, it will put the message "No network connection".Else, if it finds the network to be ok for experimentation, it will communicate with the server and will authenticate the user if he is authorised to access the particular SBHS at the booked time. If the user is not authorised it he/she will get the message "Authentication failed. Please check your username and password". If the user is authenticate but has not booked the slot, he/she will get the message "No valid slot found. Please book a slot before starting the experiment". Also if the settings.txt file is not properly configured, it will give error {\tt Invalid settings in the "setttings.txt" file}. If every thing is fine, the client will get the "Login Successfull message." It will also display the log file name and the time left for experimentation.
\item The user now launches scilab, changes the directory to the folder where the necessary .sce, .sci and .xcos files of an experiment of interest resides. The user will execute the scilab code (.sce file) pertaining to the experiment. If the network connection is fine, it will automatically open the corresponding .xcos file. Else it will output a message on the scilab console {\tt No network connection}.
\item The user will run the xcos file. It will open a plot of the various experimental parameters. This process will continue until the experiment is stopped or the simulation time is lapsed. The simulation time can be changed by changing the {\tt final integration time} parameter available in the {\tt set up} option in the {\tt simulation} menu on the Xcos window.

\item The slot is made to last for 55 minutes. The last 5 minutes of the slot are used to reset the SBHS so that the next user will get the SBHS at a normal operating condition. The python client ceases connection automatically as soon as 55 minutes are over. Please note that there will be no pop up warning and the experiment will be stoped automatically.
\item A log of the experimental data with time stamp is maintained on the client side. It is also available to download using the "Download" link once you login on the sbhs vlabs website.
\end{enumerate} 


%\chapter{Identification of Transfer Function of a Single Board Heater System through Step Response Experiment}\label{chap1}
The aim of this experiment is to perform step test on a Single Board Heater System and to identify system transfer 
function using step response data. The target group is anyone who has basic knowledge of control engineering.

\begin{figure}
\centering
\includegraphics[width=\linewidth]{Step-test_manual/step_xcos.jpg}
\caption{Xcos for this experiment}
\label{xcos_st}
\end{figure}

\begin{figure}
\centering
\includegraphics[width=\linewidth]{Step-test_manual/ste-test.png}
\caption{Graph shows heater current, fan speed and output temperature}
\label{fig:scope_st}
\end{figure}

We have used Scilab and Xcos as an interface for sending and receiving data. 
Xcos diagram is shown in figure \ref{xcos_st}. Heater current and fan speed are the two inputs for this system. 
They are given in percentage of maximum. These inputs can be varied by setting the properties of the input block's properties 
in Xcos. The plots of their amplitude versus number of collected samples are also available on the scope windows. 
The output temperature profile, as read by the sensor, is also plotted. The data acquired in the process is stored on the 
local drive and is available to the user for further calculations.

 In the {\tt step\_test.xcos} file, open the heater block's parameters to apply a step change of say 10 percent to the heater at operating point of 30 percent of heater after 250 seconds. The block parameters of the step input block will have {\tt Step time = 250}, {\tt Initial value = 30} and {\tt Final value = 40}. 
Keep the fan input constant at 50 percent. Start the experiment and let it continue until you see the temperature 
reach the steady state. 

\begin{table}
\begin{verbatim}
 1.0    30.0    50.0    29.3   1412400132192.0
 2.0    30.0    50.0    29.5   1412400133044.0
.
.
820.0    40.0     50.0    37.0   1412400950197.0
821.0    40.0     50.0    37.2   1412400951202.0
\end{verbatim}
\caption{Step data obtained after performing local Step Test}
\label{stepdata}
\end{table}

The step test data file will be saved in {\tt Step\_test} folder. The name of the file will be the date and time at which the experiment was conducted. A sample data file is provided in the same folder. The sample data file is named as {\tt step-data-local.txt} and {\tt step-data-virtual.txt}. Refer to the one depending on wheather you are performing a local or a virtual experiment. Referring to the data file thus obtained as shown in table \ref{stepdata}, the first column in this table denotes samples. The second column in this table denotes heater in percentage. It starts at 30 and increases with a step size of 10 units. The third column denotes the fan in percentage. It has been held constant at 50 percent. The fourth column refers to the value of temperature. The fifth column denotes time stamp. The virtual data file will have total four time stamp columns apart from first 3 columns. These four time stamp columns are client departure, server arrival, server departure and client arrival. These can be used for advanced control algorithms. These additional time stamps exist in virtual mode because of the presense of network delay.
\section{Conducting Step Test on SBHS locally}
The detailed procedure to perform a local experiment is explained in Chapter\ref{sercomm}. A summary of the same is provided in section \ref{local-summary} It is exactly the same for this section.
\section{Conducting Step Test on SBHS, virtually}
The steps for conducting a step test experiment virtually is exactly same as explained in section \ref{virtual-summary}.

\section{Identifying First Order and Second Order Transfer Functions}
In this section we shall determine the first and second order transer function model using the data obtained after performing step test experiment. Please note that this procedure is common for data obtained using both local and virtual experiments.


\subsection{Determination of First Order Transfer Function}
Identification of the transfer function of a system is important as it helps us to 
represent the physical system mathematically. Once the transfer function is obtained, one can acquire 
the response of the system for various inputs without actually applying them to the system.

Consider the standard first order transfer function given below
\begin{align}
G(s) &= \frac{ C(s)}{ R(s)}\label{1}\\
G(s)&=\frac 1{\tau s+1}\label{std_fo}                           
\intertext{Rewriting the equation, we get}
C(s)  &= \frac {R(s)}{\tau s + 1}\label{rw_std_fo}
\intertext{A step is given as input to the first order system. The Laplace 
transform of a step function is$ \frac {1}{s}$. Hence, substituting $ R(s) = \frac {1}{s}$ in equation \ref{rw_std_fo}, 
we obtain}
C(s) & = \frac 1{\tau s + 1}\frac 1{s}\label{sub_rs}\\
\intertext{Solving $C(s)$ using partial fraction expansion, we get}
C(s) &= \frac1{s} - \frac {1}{s + \frac1 \tau}\label{pf}
\intertext{Taking the Inverse Laplace transform of equation \ref{pf}, we get}
c(t)&= 1 - e^{\frac {-t}\tau }\label{lti} 
\end{align}
From the above equation it is clear that for t=0, the value of c(t) is zero. For t= $\infty$, c(t) 
approaches unity. Also, as the value of \textquoteleft t \textquoteright  becomes equal to $\tau$, 
the value of c(t) becomes 0.632. $\tau$ is called the time constant and represents the speed of 
response of the system. But it should be noted that, smaller the time constant- faster the system response.
By getting the value of $\tau$, one can identify the transfer function of the system. 

Consider the system to be first order. We try to fit a first order transfer function of the form
\begin{align}       
G(s) &= \frac K{\tau s + 1}\label{7}
\intertext{to the Single Board Heater System. Because the transfer function approach uses deviation 
variables, $ G(s)$ denotes the Laplace transform of the gain of the system between the change in heater 
current and the change in the system temperature. Let the change in the heater current be denoted by $\Delta u$.  
We denote both the time domain and the Laplace transform variable by the same lower case variable. Let the change 
in temperature be denoted by $y$. Let the current change by a step of size $u$. Then, we obtain the following 
relation between the current and the temperature.} 
y(s) &= G(s)u(s)\\ 
y(s)&= \frac K{\tau s + 1}{\frac  {\Delta u}{{s}}}
\intertext {Note that $\Delta$ u is the height of the step and hence is a constant. On inversion, we obtain}
y(s)&= K[1 - e^{\frac{-t}\tau}]\Delta u
\end{align}
\subsection{Procedure}
\begin{enumerate}
\item Download the Analysis folder from the sbhs website. It will be available under {\tt downloads} section. The download will be in zip format. Extrat the downloaded zip file. You will get a folder {\tt Analysis}. 
\item Open the {\tt Analysis} folder and then locate and open the folder {\tt Step\_Analysis}.
\item Open the {\tt Kp-tau-order1} folder.
 \item Copy the step test data file to the folder {\tt Kp-tau-order1}.
 \item Change the Scilab working directory to {\tt Kp-tau-order1} folder under {\tt Step\_Analysis} folder.
 \item Open the file {\ttfamily firstorder.sce} in scilab editor and enter the name of the data file (with extention) in the {\tt filename} field. 
\item Save and run this code and obtain the plot as shown in figure \ref{firstorder_output}. 
\end{enumerate}
This code uses the routines {\ttfamily label.sci} and {\ttfamily costf\_1.sci}

\begin{figure} 
\centering
\includegraphics[width=\linewidth]{Step-test_manual/local-1-order.png}
\caption{Output of the Scilab code \ttfamily firstorder.sce}
\label{firstorder_output}
\end{figure} \label{firstorderplot}

\begin{align}
\intertext{The plot thus obtained is reasonably good. See the Scilab plot to get the values of $\tau$ and $K$. 
The figure \ref{firstorder_output} shows a screen shot of the same. We obtain $\tau$ = 58.64, K = 0.23. The transfer function 
obtained here is at the operating point of 30 percentage of heat. If the experiment is repeated at a different operating point, 
the transfer function obtained will be different. The gain will correspondingly be more at a higher operating point. 
This means that the plant is faster at higher temperature. Thus the transfer function of the plant varies with the operating 
point. Let the transfer function we obtain in this experiment be denoted as $G_s$. We obtain}
G_s(s) =  \frac{0.23}{58.64s+1} \label{12}
\end{align}
% \begin{figure}
% \centering
% \includegraphics[width=\linewidth]{Step-test_manual/forder_console.png}
% \caption{The value of time constant and gain as shown on the console by \ttfamily firstorder.sce}
% \label{console}
% \end{figure}


\section{Determination of Second Order Transfer Function}
In this section, we explore the efficacy of a second order model of the form
\begin{align}
G(s) & = \frac K{(\tau_1s+1)(\tau_2s+1)} \label{eq:step-1100} 
\intertext{The response of the system to a step input of height $\Delta u$ is given by}
y(s) & = \frac K{(\tau_1s+1)(\tau_2s+1)} \frac{\Delta u}s 
\label{eq:step-1200} 
\end{align}

Splitting into partial fraction expansion, we obtain
\begin{align*}
y(s) & = \frac K{\tau_1\tau_2} \frac 1
{\left(s+\dfrac 1{\tau_1}\right)\left(s+\dfrac 1{\tau_2}\right)} =
\frac A s + \frac B{s+\dfrac 1{\tau_1}} + \frac C{s+\dfrac 1{\tau_2}}
\intertext{Through Heaviside expansion method, we determine the coefficients:}
A & = K \\
B & = -\frac{K\tau_1}{\tau_1-\tau_2} \\
C & = \frac{K\tau_2}{\tau_1-\tau_2}
\end{align*}

On substitution and inversion, we obtain
\begin{align}
y(t) & = K\left[ 1 - \frac 1{\tau_1-\tau_2}
\left( \tau_1 e^{-t/\tau_1} - \tau_2 e^{-t/\tau_2} \right)
\right] \label{eq:step-1300}
\end{align}

We have to determine three parameters $K$, $\tau_1$ and $\tau_2$
through optimization. Once again, we follow a procedure identical to the first order model.  
The only difference is that we now have to determine three parameters. Scilab code \\{\tt secondorder.sce} calculates
the gain and two time constants. 
\subsection {Procedure}
\begin{enumerate}
\item Download the Analysis folder from the sbhs website. It will be available under {\tt downloads} section. The download will be in zip format. Extrat the downloaded zip file. You will get a folder {\tt Analysis}. 
\item Open the {\tt Analysis} folder and then locate and open the folder {\tt Step\_Analysis}.
\item Open the {\tt Kp-tau-order2} folder.
 \item Copy the step test data file to the folder {\tt Kp-tau-order2}.
 \item Change the Scilab working directory to {\tt Kp-tau-order2} folder under {\tt Step\_Analysis} folder.
 \item Open the file {\ttfamily secondorder.sce} in scilab editor and enter the name of the data file (with extention) in the {\tt filename} field. 
\item Save and run this code and obtain the plot as shown in figure \ref{sorder}. 
\end{enumerate}
\begin{align}
G_{s}(s) & = \frac {0.235}{(57.39s+1)(1s+1)}
\label{eq:step-1400}
\end{align}

\begin{figure}
\centering
\includegraphics[width=\linewidth]{Step-test_manual/local-2-order.png}
\caption{Output of the Scilab code \ttfamily secondorder.sce}
\label{sorder}
\end{figure}

The fit is much better now.  In particular, the initial inflexion is well captured by this second
order transfer function.


\section{Discussion}
We summarize our findings now. For the first order analysis, the gain is 0.23 and the 
time constant $\tau$ is 58.64 seconds. For the second order analysis, the initial inflexion is 
well captured with the two time constants $\tau_1$=57.39, $\tau_2$= 1 and gain = 0.235. Negative steps 
can also be introduced to make the experiment more informative. One need not keep a particular 
input constant. By varying both the inputs, one can imagine it to be like a step varying disturbance signal.
 



\section{Scilab Code}\label{stepcodes}
\begin{code}
\ccaption{label.sci}{\ttfamily label.sci}
\lstinputlisting{Scilab/Analysis/Step_Analysis/Kp-tau-order1/label.sci}
\end{code}

\begin{code}
\ccaption{costf\_1.sci}{\ttfamily costf\_1.sci}
\lstinputlisting{Scilab/Analysis/Step_Analysis/Kp-tau-order1/costf_1.sci}
\end{code}


\begin{code}
\ccaption{firstorder.sce}{\ttfamily firstorder.sce}
\lstinputlisting{Scilab/Analysis/Step_Analysis/Kp-tau-order1/firstorder.sce}
\end{code}

\begin{code}
\ccaption{costf\_2.sci}{\ttfamily costf\_2.sci}
\lstinputlisting{Scilab/Analysis/Step_Analysis/Kp-tau-order2/costf_2.sci}
\end{code}

\begin{code}
\ccaption{order\_2\_heater.sci}{\ttfamily order\_2\_heater.sci}
\lstinputlisting{Scilab/Analysis/Step_Analysis/Kp-tau-order2/order_2_heater.sci}
\end{code}

\begin{code}
\ccaption{secondorder.sce}{\ttfamily secondorder.sce}
\lstinputlisting{Scilab/Analysis/Step_Analysis/Kp-tau-order2/secondorder.sce}
\end{code}

\begin{code}
\ccaption{ser\_init.sce}{\ttfamily ser\_init.sce}
\lstinputlisting{Scilab/local/Step_test/ser_init.sce}
\end{code}

\begin{code}
\ccaption{step\_test.sci}{\ttfamily step\_test.sci}
\lstinputlisting{Scilab/local/Step_test/step_test.sci}
\end{code}

\begin{code}
\ccaption{stepc.sce}{\ttfamily stepc.sce}
\lstinputlisting{Scilab/virtual/StepTest/stepc.sce}
\end{code}

\begin{code}
\ccaption{steptest.sci}{\ttfamily steptest.sci}
\lstinputlisting{Scilab/virtual/StepTest/steptest.sci}
\end{code}














%\chapter{Identification of Transfer Function of a Single Board Heater System through Ramp Response Experiment}\label{chap2}
The aim of this experiment is to perform ramp test on a Single Board Heater System and to identify system transfer 
function using ramp response data. The target group is anyone who has basic knowledge of control engineering.

\begin{figure}
\centering
\includegraphics[width=0.7\linewidth]{Ramp-test_manual/ramp_test.jpg}
\caption{Xcos for ramp test experiment}
\label{Xcos_rt}
\end{figure} 

\begin{figure}
\centering
\includegraphics[width=\linewidth]{Ramp-test_manual/ramp-test.png}
\caption{Screen shot of ramp test experiment}
\end{figure}

We have used Scilab and Xcos as an interface for sending and receiving data. 
Xcos diagram is shown in figure \ref{Xcos_rt}. Heater current and fan speed are the two inputs for this system. 
They are given in percentage of maximum. These inputs can be varied by setting the properties of the input block's properties 
in Xcos. The plots of their amplitude versus number of collected samples are also available on the scope windows. 
The output temperature profile, as read by the sensor, is also plotted. The data acquired in the process is stored on the 
local drive and is available to the user for further calculations.

 In the {\tt ramp\_test.xcos} file, open the heater block's parameters to give a ramp input to the system with some value for slope. For this experiment, we have chosen slope = $0.1$. Double click on the ramp input block labled as {\tt Heater input}. Change the following values in the respective fields: slope = 0.1, start time = 200, initial output = 20. Keep the fan constant at 100.


\begin{table}
\begin{verbatim}
1.0      30.0      50.0       28.1   1416462726532.0
2.0       30.0        50.0         28.1   1416462727574.0
.
999.0         100.0       50.0        47.6   1416463723533.0
1000.0        100.0      50.0       47.6   1416463724533.0
\end{verbatim}
\caption{Ramp data obtained after performing local Step Test}
\label{rampdata}
\end{table}

The ramp test data file will be saved in {\tt Ramp\_Test} folder. The name of the file will be the date and time at which the experiment was conducted. A sample data file is provided in the same folder. The sample data file is named as {\tt ramp-data-local.txt} and {\tt ramp-data-virtual.txt}. Refer to the one depending on wheather you are performing a local or a virtual experiment. Referring to the data file thus obtained as shown in table \ref{rampdata}, the first column in this table denotes samples. The second column in this table denotes heater in percentage. It starts at 30 and increases with a step size of 10 units. The third column denotes the fan in percentage. It has been held constant at 50 percent. The fourth column refers to the value of temperature. The fifth column denotes time stamp. The virtual data file will have four time stamp columns apart from first 3 columns. These four time stamp columns are client departure, server arrival, server departure and client arrival. These can be used for advanced control algorithms. These additional time stamps exist in virtual mode because of the presense of network delay.

\section{Conducting Ramp Test on SBHS locally}
The detailed procedure to perform a local experiment is explained in Chapter\ref{sercomm}. A summary of the same is provided in section \ref{local-summary} It is same for this section with following changes.

\begin{enumerate}
\item Step1: The working directory is {\tt  Ramp\_test}
\item Step2: Same
\item Step3: Same
\item Step4: Same
\item Step5: Load ramp test function by executing command\\ {\tt exec<space>ramp\_test.sci}
\item Step6: Load Xcos code for ramp test using the command\\ {\tt exec<space>ramp\_test.xcos}
\item Step7: Same
\end{enumerate}
\section{Conducting Ramp Test on SBHS, virtually}
The detailed procedure to perform a virtual experiment is explained in Chapter\ref{virtual}. A summary of the same is provided in section \ref{vlabsexpt}. It is same for this section with following changes.

\begin{enumerate}
\item Step1: The working directory is {\tt  RampTest}. Open this directory.
\item Step2: Same
\item Step3: Same
\item Step4:  Switch to the RampTest experiment directory and double-click on the file {\tt ramptest.sce}. This will launch scilab and also open the file {\tt ramptest.sce} in the scilab editor. Linux users will have to launch scilab manually. They also have to change the working directory to {\tt  RampTest} and then open the {\tt  RampTest} file in the scilab editor.
\item Step5: Same
\item Step6: Execute the file {\tt ramptest.sce}.  Expect the ramp test xcos diagram to open automatically. If this doesnt happen, check the scilab console for error message.
\item Step7: Execute the ramptest xcos diagram.
\item Step8: Same
\end{enumerate}


 The virtual experiment response is shown in figure \ref{ramp-virtual}. The corresponding data file is shown in table \ref{rampdata}. The time stamps shown are cut short for better viewing. This data file can be found in {\tt RampTest} folder for virtual experiments. The name of this file is {\tt step-data-virtual.txt}.


\begin{figure}
\centering
\includegraphics[width=\linewidth]{Ramp-test_manual/ramp-virtual-plot.png}
\caption{Ramp test Virtual experiment response}
\label{ramp-virtual}
\end{figure}


\begin{table}
\begin{verbatim}
0 0 100 28.80 14...8993 14...0301 14...0318 14...9040 0.10000E+01
1 30 50 28.80 14...2861 14...4172 14...4189 14...2908 0.10000E+01
.
.
617 66 50 39.10 14...7141 14...8476 14...8494 14...7188 0.61700E+03
618 66 50 39.10 14...8120 14...9456 14...9473 14...8167 0.61800E+03
\end{verbatim}
\caption{Ramp data obtained after performing virtual Ramp Test}
\label{rampdata}
\end{table}



\section{Identifying First Order Transfer Function}
In this section we shall determine the first order transer function model using the data obtained after performing step test experiment. Please note that this procedure is common for data obtained using both local and virtual experiments.

Identification of the transfer function of a system is important as it helps us to 
represent the physical system mathematically. Once the transfer function is obtained, one can acquire 
the response of the system for various inputs without actually applying them to the system.
Consider the standard first order transfer function given below
\begin{align}
G(s) &= \frac{ C(s)}{ R(s)}\\ 
G(s)&=\frac K{\tau s+1}\\               
\intertext{Combining the previous two equations, we get}
C(s)  &= K \left\{\frac {R(s)}{\tau s + 1}\right\}\label{fotf}
\intertext{Let us consider the case of giving a ramp input to this first order system. 
The Laplace transform of a ramp function with slope = $\upsilonup$ is $ \frac \upsilonup {s^2}$. 
Substituting $ R(s) = \frac \upsilonup {s^2}$ in equation \ref{fotf}, we obtain}
C(s) & =  \frac K{\tau s + 1}\frac \upsilonup {s^2}\\
&= \frac A{s} + \frac B{s^2} +\frac C{\tau s + 1}\\
\intertext{Solving $C(s)$ using Heaviside expansion approach, we get}
C(s) &= K\upsilonup \left\{\frac1{s^2} -  \frac \tau s + \frac {\tau^2}{\tau s + 1}\right\}\label{Heaviside}\\
\intertext{Taking the Inverse Laplace transform of the above equation, we get}
c(t)&= K\upsilonup \left\{t -\tau   + \tau e^{\frac {-t}\tau }\right\}\label{ct} \\
\intertext{The difference between the reference and output signal is the error signal $e(t)$. Therefore,}
e(t)&= r(t) - c(t)\\
e(t)&= K\upsilonup t - K\upsilonup t + K\upsilonup \tau  - K\upsilonup \tau e^\frac {-t}\tau   \\
e(t)&= K\upsilonup \tau (1 - e^{\frac {-t}\tau})\label{et}\\
\intertext{Normalizing equation \ref{et} for $t>>\tau$, we get}
e(t) &= \tau
\end{align}
This means that the error in following the ramp input is equal to $\tau$ for 
large value of $t$ \cite{ogt05}. Hence, smaller the time constant $\tau$, smaller the steady state error.


\subsection{Procedure}
\begin{enumerate}
\item Download the Analysis folder from the sbhs website. It will be available under {\tt downloads} section.  Download the file for {\tt SBHS Analysis Code (local \& virtual)}. The name of the file is {\tt scilab\_codes\_analysis}. The download will be in zip format. Extrat the downloaded zip file. You will get a folder {\tt scilab\_codes\_analysis}. 
\item Open the {\tt scilab\_codes\_analysis} folder and then locate and open the folder {\tt Ramp\_Analysis}.
 \item Copy the ramp test data file to this folder.
 \item Change the Scilab working directory to  {\tt Ramp\_Analysis}
 \item Open the file {\ttfamily ramp\_virtual.sce} in scilab editor and enter the name of the data file (with extention) in the {\tt filename} field. 
\item Save and run this code and obtain the plot as shown in figure \ref{firstorder_ramp}. 
\end{enumerate}
This code uses the routines {\ttfamily label.sci} and {\ttfamily costf\_1.sci}

\begin{figure} 
\centering
\includegraphics[width=\linewidth]{Ramp-test_manual/ramp-analysis.png}
\caption{Output of the Scilab code {\ttfamily ramp\_virtual.sce}}
\label{firstorder_ramp}
\end{figure} \label{firstorderplot}

\begin{figure} 
\centering
\includegraphics[width=\linewidth]{Ramp-test_manual/ramp-console.png}
\caption{Scilab console after executing code{\ttfamily ramp\_virtual.sce}}
\label{firstorder_ramp_console}
\end{figure} \label{firstorderplot}

\begin{align}
\intertext{The results presented are obtained for the data file {\tt ramp-data-virtual.txt}. This data file is present under the {\tt Ramp\_Test} directory for local experiments. The plot thus obtained is reasonably good. See the Scilab console to get the values of $\tau$ and $K$. It is as shown in figure \ref{firstorder_ramp_console}
The figure \ref{firstorder_ramp} shows a screen shot of the same. We obtain $\tau$ = 78.92, K = 0.22. The transfer function 
obtained here is at the operating point of enterValue percentage of heat. If the experiment is repeated at a different operating point, the transfer function obtained will be different. The gain will correspondingly be more at a higher operating point. This means that the plant is faster at higher temperature. Thus the transfer function of the plant varies with the operating 
point. Let the transfer function we obtain in this experiment be denoted as $G_s$. We obtain}
G_s(s) =  \frac{0.22}{78.92s+1} \label{12}
\end{align}

\section{Discussion}
We summarize our findings now. The experiment has been performed by varying the heater current and keeping the fan 
speed constant. However, the user is encouraged to experiment using different combinations of fan speed 
and heater current. Negative ramp can also be used to make the experiment more informative. 
It is not necessary to keep a particular input constant. For example, you can try giving a ramp input to the 
disturbance signal, i.e., the fan input. The system can also be treated as a second order system. This consideration is necessary as it increases the accuracy of the acquired transfer function \cite{kmm09}.

The necessary codes are listed in the section \ref{rampcodes}.


%###################################
\section{Scilab Code}\label{rampcodes}
\begin{code}
\ccaption{ramp\_test.sci}{\ttfamily ramp\_test.sci}
\lstinputlisting{Scilab/local/Ramp_Test/ramp_test.sci}
\end{code}

\begin{code}
\ccaption{label.sci}{\ttfamily label.sci}
\lstinputlisting{Scilab/Analysis/Ramp_Analysis/label.sci}
\end{code}

\begin{code}
\ccaption{cost.sci}{\ttfamily cost.sci}
\lstinputlisting{Scilab/Analysis/Ramp_Analysis/cost.sci}
\end{code}

\begin{code}
\ccaption{cost\_approx.sci}{\ttfamily cost\_approx.sci}
\lstinputlisting{Scilab/Analysis/Ramp_Analysis/cost_approx.sci}
\end{code}

\begin{code}
\ccaption{ramptest.sci}{\ttfamily ramptest.sci}
\lstinputlisting{Scilab/virtual/RampTest/ramptest.sci}
\end{code}

\begin{code}
\ccaption{ramptest.sce}{\ttfamily ramptest.sce}
\lstinputlisting{Scilab/virtual/RampTest/ramptest.sce}
\end{code}

\begin{code}
\ccaption{ramp\_virtual.sce}{\ttfamily ramp\_virtual.sce}
\lstinputlisting{Scilab/Analysis/Ramp_Analysis/ramp_virtual.sce}
\end{code}


%\bibliography{New} % Adding References

%\chapter{Frequency Response Analysis of a Single Board Heater System by the Application of Sine Wave}
The aim of this experiment is to do a frequency response analysis of a Single Board Heater System by the 
application of sine wave. The target group is anyone who has basic knowledge of control engineering.


%%%%%%%%%%%%%

\begin{figure}
\centering
\includegraphics[width=\linewidth]{sinetest_manual/sine_test.jpg}
\caption{Xcos for local Sine Test}
\label{xcos_sine}
\end{figure}

\begin{figure}
\includegraphics[width=\linewidth]{sinetest_manual/sine-007-plot.png}
\caption{Plot for sine input 0.007Hz}
\label{fig:scope}
\end{figure}

We have used Scilab and Xcos as an interface for sending and receiving data. 
Xcos diagram is shown in figure \ref{xcos_sine}.
The heater current is varied sinusoidally. They are given in percentage of maximum. These inputs can be varied by setting the properties of the input block's properties in Xcos. A provision is made to set the parameters related to it like frequency, amplitude 
and offset. The temperature profile thus obtained is the output. In this experiment we are applying a sine change in the heater current by keeping the fan speed constant. After application of sine change, wait for sufficient amount of time to allow the temperature to reach a steady-state. The plots of their amplitude versus number of collected samples are also available on the scope windows. The output temperature profile, as read by the sensor, is also plotted. The data acquired in the process is stored on the local drive and is available to the user for further calculations.

 In the {\tt sine\_test.xcos} file, open the {\tt sinusoid generator} block's parameters to set the value of sine magnitude and frequency. For the experiment results shown, we have chosen Magnitude = $10$, Frequency = $2*3.14*0.007$. Note that the frequency is to be put in rad/sec. We keep the Phase = $0$. There is also a provision to give the sine input with an offset in amplitude. This can be set using the {\tt Offset} block. We have choosen offset of 20. The time at which the sine input is given after the experiment is started can also be set. This can be done using the {\tt Initial Time} block. Open the {\tt Initial Time} block's parameters. To make the sine input appear after 200 samples of start of the experiment, keep Step time = $200$, Initial Value = $0$ and Final Value = $1$. The initial value and final value will never change for any other value of step time.


\begin{table}
\begin{verbatim}
1.0     20.0       50.0      20.0   1416642805261.0
2.0       20.0       50.0        20.0   1416642806332.0
.
13093.0    28.5       50.0         26.6   1416656557353.0
13094.0     28.5       50.0        26.6   1416656558363.0
\end{verbatim}
\caption{Data obtained after application of sine input of $0.007Hz$}
\label{sinedata}
\end{table}

The sine test data file will be saved in {\tt Sine\_Test} folder. The name of the file will be the date and time at which the experiment was conducted. A sample data file is provided in the same folder. The sample data file is named as {\tt sine-data-local.txt} and {\tt sine-data-virtual.txt}. Refer to the one depending on wheather you are performing a local or a virtual experiment. Referring to the data file thus obtained as shown in table \ref{sinedata}, the first column in this table denotes samples. The second column in this table denotes heater in percentage. It starts at 20 and then varies sinosoidally. The third column denotes the fan in percentage. It has been held constant at 50 percent. The fourth column refers to the value of temperature. The fifth column denotes time stamp. The virtual data file will have four time stamp columns apart from first 3 columns. These four time stamp columns are client departure, server arrival, server departure and client arrival. These can be used for advanced control algorithms. These additional time stamps exist in virtual mode because of the presense of network delay.

\section{Conducting Sine Test on SBHS locally}
The detailed procedure to perform a local experiment is explained in Chapter\ref{sercomm}. A summary of the same is provided in section \ref{local-summary} It is same for this section with following changes.

\begin{enumerate}
\item Step1: The working directory is {\tt  Sine\_test}
\item Step2: Same
\item Step3: Same
\item Step4: Same
\item Step5: Load sine test function by executing command\\ {\tt exec<space>sine\_test.sci}
\item Step6: Load Xcos code for sine test using the command\\ {\tt exec<space>sine\_test.xcos}
\item Step7: Same
\end{enumerate}


\section{Conducting Sine Test on SBHS, virtually}
The detailed procedure to perform a local experiment is explained in Chapter\ref{virtual}. A summary of the same is provided in section \ref{vlabsexpt}. It is same for this section with following changes.

\begin{enumerate}
\item Step1: The working directory is {\tt  SineTest}. Open this directory.
\item Step2: Same
\item Step3: Same
\item Step4:  Switch to the SineTest experiment directory and double-click on the file {\tt sinetest.sce}. This will launch scilab and also open the file {\tt sinetest.sce} in the scilab editor. Linux users will have to launch scilab manually. They also have to change the working directory to {\tt  SineTest} and then open the {\tt  sinetest.sce} file in the scilab editor.
\item Step5: Same
\item Step6: Execute the file {\tt sinetest.sce}.  Expect the sine test xcos diagram to open automatically. If this doesnt happen, check the scilab console for error message.
\item Step7: Execute the sinetest xcos diagram.
\item Step8: Same
\end{enumerate}


 The virtual experiment response is shown in figure \ref{sine-virtual}. The corresponding data file is shown in table \ref{sinedata}. The time stamps shown are cut short for better viewing. This data file can be found in {\tt SineTest} folder for virtual experiments. The name of this file is {\tt sine-data-virtual.txt}.


\begin{figure}
\centering
\includegraphics[width=\linewidth]{sinetest_manual/sine-plot-virtual-007.png}
\caption{Sine test Virtual experiment response}
\label{sine-virtual}
\end{figure}


\begin{table}
\begin{verbatim}
0 0 100 28.80 14...4739 14...6427 14...6445 14...4786 0.10000E+01
1 20 50 28.90 14...5247 14...6938 14...6956 14...5294 0.10000E+01
.
.
999 18 50 31.00 14...2253 14...3973 14...3990 14...2299 0.99900E+03
1000 19 50 31.00 14...3268 14...4989 14...5007 14...3315 0.10000E+04
\end{verbatim}
\caption{Sine data obtained after performing virtual Sine Test}
\label{rampdata}
\end{table}

%%%


\section{Frequency Analysis of sine test data}
 Frequency response of a system means its steady-state response to a sinusoidal input. 
 For obtaining a frequency response of a system, we vary the frequency of the input signal over a spectrum of interest. 
 The analysis is useful and simple because it can be carried out with the available signal generators and measuring devices. Let us see the theory and procedure.  Please note that this procedure is common for data obtained using both local and virtual experiments.

Consider a sinusoidal input
\begin{align}
U(t) &= Asin \omega t
\intertext{The Laplace transform of the above equation yields}
U(s) &= \frac{A\omega}{s^2 + \omega^2} \label{lap_tran}
\intertext{Consider the standard first order transfer function given below}
G(s) &= \frac {Y(s)}{U(s)} = \frac K{s + 1}
\intertext{Replacing the value of U(s) from equation \ref{lap_tran}, we get}
Y(s) &= \frac{KA\omega}{(\tau s + 1)(s^2 + \omega ^2)}\\
&=\frac{KA}{\omega ^2\tau ^2 + 1}\left[\frac{\omega \tau ^2}{\tau s +1}- \frac{\tau s \omega}{s^2 + \omega^2}+\frac{\omega}
{s^2 + \omega^2}\right]
\intertext{Taking Laplace Inverse, we get}
y(t) &= \left[\frac {KA}{\omega^2\tau^2+ 1}\right]\left[\omega \tau e^{\frac {-t}{\tau}}-\omega \tau cos(\omega t)+
sin(\omega t)\right] 
\intertext{The above equation has an exponential term $e^\frac{-t}{\tau}$. Hence, for large value of time, its value will 
approach to zero and the equation will yield a pure sine wave. One can also use trigonometric identities to make the equation 
look more simple.}
y(t) &= \left[\frac{KA}{\sqrt{\omega^2 \tau^2 + 1}}\right]\left[sin (\omega t) + \phi \right]
\intertext{where,}
\phi &= -tan^{-1}(\omega \tau)
\intertext{By observing the above equation, one can easily make out that for a sinusoidal input the output is also sinusoidal
but has some phase difference. 
Also, the amplitude of the output signal, $\hat{A}$, has become a function of the input signal frequency, $\omega$.}
\hat{A}&=\frac{KA}{\sqrt{\omega^2 \tau^2 + 1}}
\intertext{The amplitude ratio (AR) can be calculated by dividing both sides by the input signal amplitude A.}
AR &=\frac{\hat{A}}{A}=\frac{K}{\sqrt{\omega^2 \tau^2 + 1}}
\intertext{Dividing the above equation by the process gain K yields the normalized amplitude ratio $(AR_n)$}
AR_n &=\frac{AR}{K}=\frac{1}{\sqrt{\omega^2 \tau^2 + 1}}
\end{align}
Because the process steady state gain is constant, the normalized amplitude ratio is often used for 
frequency response analysis \cite{dale04}.

\subsection{Procedure}
Now let us calculate amplitude ratio and phase difference. 

\begin{enumerate}
\item Download the Analysis folder from the sbhs website. It will be available under {\tt downloads} section. The download will be in zip format. Extrat the downloaded zip file. You will get a folder {\tt Analysis}. 
\item Open the {\tt Analysis} folder and then locate and open the folder {\tt Sine\_Analysis}.
 \item Copy the sine test data file to this folder.
 \item Change the Scilab working directory to  {\tt Sine\_Analysis}
 \item Open the file {\ttfamily sine-analysis.sce} in scilab editor and enter the name of the data file (with extention) in the {\tt filename} field.
\item Put the value of frequency {\ttfamily f} for the calculation of amplitude ratio and phase difference and execute it. Here {\ttfamily f} means input frequency.
\item Expect the values of amplitude ratio and phase difference on the scilab console.
\end{enumerate}

  
\begin{figure}
\includegraphics[width=\linewidth]{sinetest_manual/sine-local-analysis-console.png}
\caption{Amplitude ratio and Phase difference for local data file}
\label{scilab_op}
\end{figure}
The results shown are for the data file {sine-data-local.txt}. It could be seen from figure \ref{scilab_op} that the amplitude ratio turns out to be $-0.915$dB and phase difference to be $-57.267$\textdegree.
The plot thus obtained is shown in figure \ref{plot0.4}
\begin{figure}
\includegraphics[width=\linewidth]{sinetest_manual/sine-local-analysis}
\caption{Plot of Input and Output vs time}
\label{plot0.4}
\end{figure}

Repeat this calculation over a range of frequencies and note down the values of amplitude ratio in dB and phase difference. 
Input these values for the appropriate frequencies into the Scilab code {\ttfamily TFbode.sce} and execute it to get a 
Bode plot of the plant which is illustrated in figure \ref{bode_plot}.
\begin{figure}
\includegraphics[width=\linewidth]{sinetest_manual/bodeplot}
\caption{Bode plot obtained from the plant}
\label{bode_plot}
\end{figure}

Bode plot can be obtained directly from the plant's second order transfer function \cite{kmm09} with the help of Scilab code
{\ttfamily TFbode.sce}, as shown in figure \ref{tfbode}. A visual comparison of the two Bode plots can be done to 
validate the Bode diagram obtained from the plant.
\begin{figure}
\includegraphics[width=\linewidth]{sinetest_manual/plant_bode_tf}
\caption{Bode plot obtained through plant's transfer function}
\label{tfbode}
\end{figure}

To compare the two plots, we plot it on the same graph as shown in figure \ref{compare_bode}
\begin{figure}
\centering
\includegraphics[width=\linewidth]{sinetest_manual/bode_comparison}
\caption{Comparison of Bode plots}
\label{compare_bode}
\end{figure}



\section{Scilab Code}\label{sinecodes}

\begin{code}
\ccaption{sine\_test.sci}{\ttfamily sine\_test.sci}
\lstinputlisting{Scilab/local/Sine_Test/sine_test.sci}
\end{code}


\begin{code}
\ccaption{sinetest.sce}{\ttfamily sinetest.sce}
\lstinputlisting{Scilab/virtual/SineTest/sinetest.sce}
\end{code}

\begin{code}
\ccaption{sinetest.sci}{\ttfamily sinetest.sci}
\lstinputlisting{Scilab/virtual/SineTest/sinetest.sci}
\end{code}

\begin{code}
\ccaption{sine2.sce}{\ttfamily sine-analysis.sce}
\lstinputlisting{Scilab/Analysis/Sine_Analysis/sine-analysis.sce}
\end{code}

\begin{code}
\ccaption{lable.sci}{\ttfamily label.sci}
\lstinputlisting{Scilab/Analysis/Sine_Analysis/label.sci}
\end{code}

\begin{code}
\ccaption{bodeplot.sce}{\ttfamily bodeplot.sce}
\lstinputlisting{Scilab/Analysis/Sine_Analysis/bodeplot.sce}
\end{code}


\begin{code}
\ccaption{labelbode.sci}{\ttfamily labelbode.sci}
\lstinputlisting{Scilab/Analysis/Sine_Analysis/labelbode.sci}
\end{code}


\begin{code}
\ccaption{TFbode.sce}{\ttfamily TFbode.sce}
\lstinputlisting{Scilab/Analysis/Sine_Analysis/TFbode.sce}
\end{code}

\begin{code}
\ccaption{comparison.sce}{\ttfamily comparison.sce}
\lstinputlisting{Scilab/Analysis/Sine_Analysis/comparison.sce}
\end{code}

%\begin{code}
%\ccaption{sine2\_virtual.sce}{\ttfamily sine2\_virtual.sce}
%\lstinputlisting{Scilab/Analysis/Sine_Analysis/sine2_virtual.sce}
%\end{code}


\chapter{PRBS Modeling and Implementation of Pole Placement Controller}
The aim of this chapter is to do PRBS testing on Single Board Heater System by the application of PRBS signal 
and to design a pole-placement controller. The target group is anyone who has basic knowledge of control engineering. The first half of this chapter is dedicated to do system identification of the SBHS system using the response obtained for a 
PRBS (Pseudo Random Binary Sequence) input. In the second half, a pole-placement controller is designed using this model and 
implemented on SBHS.

\section{PRBS Modelling}

Similar to Chapter \ref{chap1} and \ref{chap2}, we will find the transfer function model of SBHS. 
But there are two major differences. First difference is that we will give a Pseudo Random Binary Sequence to the 
heater input of SBHS and the second difference is that we will find the discrete time transfer function. A Pseudo Random 
Binary Sequence is nothing but a signal whose amplitude varies between two limits randomly at any given time. An 
illustration of the same is given in figure \ref{prbs-fig}. A PRBS signal can be easily generated using the $rand()$ 
function in Scilab. Scilab code to generate the PRBS signal is given at the end of this chapter.

We have used Scilab with Xcos as an interface for sending and receiving data. This interface is shown in figure \ref{prbs-xcos}.
Heater current and fan speed are the two inputs to the system. The heater current is varied with a PRBS signal. A provision
is made to set the parameters like PRBS amplitude and offset value. A provision is also made to time the occurance of the PRBS 
input using a step block. The value of step time in the step block has to be chosen carefully. Sufficient amount of time
should be given to allow the temperature to reach a steady-state before the PRBS signal is applied. In this experiment we are
keeping the fan speed constant at 50\%. The temperature profile thus obtained is the output.


\begin{figure}
\centering
\includegraphics[width=0.7\linewidth]{prbs/prbs-xcos.png}
\caption{Xcos for PRBS testing experiment}
\label{prbs-xcos}
\end{figure}

\subsection{Issues with Step Test and an Alternate Approach}

SBHS is an example of a heater. Suppose you are working in a full scale plant. Current control system designed to control
one of the heaters of the plant is lousy and your supervisor asks you to design a new controller from scratch. The first step
you need to do is identification of the heater transfer function. The catch is, the plant is currently
operational. You can't shut the plant down to identify the heater transfer function. You have to do it while the heater is
operating in the plant. You might think of giving the heater a positive step and measuring the response in the controlled 
temperature. This will increase the temperature of the component being heated for the period of time step is applied. However,
if the process is sensitive to temperature of the component (distillation, for example), it will go off the desired course and
the output of the whole plant will be affected and will be undesirable.\\

There is an alternate approach which is widely used in industry. The input given to the heater for identification is not 
step, but a \textbf{pseudo-random binary sequence (PRBS)}. The concept behind PRBS is that the input is perturbed in such
a way that the time average of the input is the value at which it is being operated currently. Thus, some positive and some
negative steps can be given. This results in some positive and some negative changes in the temperature which leads to the time
average of the performance of the plant remaining the same. Thus, PRBS testing can be done in a working plant without 
affecting the plant performance unlike step testing. A typical PRBS and corresponding plant output is shown in 
figure \ref{sample-prbs}

\begin{figure}[H]
\begin{centering}
\includegraphics[width=0.6\textwidth]{prbs/PRBS}
\par\end{centering}

\caption{PRBS testing input and output \textit{{[}Image source: CL 686 Advanced
Process Control, Spring 2013-14 lecture slides. Prof. S. C. Patwardhan, IIT Bombay{]}}}
\label{sample-prbs}
\end{figure}

\section{Conducting PRBS Test on SBHS locally}
The detailed procedure to perform a local experiment is explained in Chapter\ref{sercomm}. A summary of the same is provided in section \ref{local-summary} It is same for this section with following changes.

\begin{enumerate}
\item Step1: The working directory is {\tt  prbs/identification}
\item Step2:  Load the functions available in common files directory by executing the command
\item Step3: Same
\item Step4: Same
\item Step5: Load prbs test function by executing command\\ {\tt exec<space>sine\_test.sci}
\item Step6: Load Xcos code for sine test using the command\\ {\tt exec<space>sine\_test.xcos}
\item Step7: Same
\end{enumerate}



\begin{figure}
\centering
\includegraphics[width=0.7\linewidth]{prbs/prbs-illustration.png}
\caption{A Pseudo Random Binary Sequence}
\label{prbs-fig}
\end{figure}

\begin{figure}
\centering
\includegraphics[width=0.7\linewidth]{prbs/prbs-expt.png}
\caption{PRBS testing response}
\label{prbs-res}
\end{figure}


\subsection{Determination of Discrete Time Transfer Function}\label{prbs-model}

System identification is carried out to identify the transfer function between the input signal to the system and output 
from the system. Firstly, a transfer function with unknown parameters is assumed. The system is given a known input and its
response is obtained and then the values of the unknown parameters is chosen such that the sum of squares of the errors is 
minimized. Here, the error is the difference between the actual output and the output predicted by the transfer function model
assumed.
For the given SBHS system, we assume a second order transfer function:

\begin{align}\label{DTF}
G(z)=\frac{b_{1}+b_{2}z^{-1}}{1+a_{1}z^{-1}+a_{2}z^{-2}}z^{-d}
\end{align}


The unknown parameters $a_1, a_2, b_1, b_2$ and $d$ are to be obtained through the response of the system to the known inputs.
$a_1, a_2, b_1, b_2$ are real numbers and $d$ is the plant delay which is an integer.  For these model parameters estimation, we
use a pseudo random binary sequence (PRBS) input. Since the optimization over discrete variables ($d$ in this case) is a very 
difficult routine for computers, we assume a value for $d$ and then optimize over  $a_1, a_2, b_1, b_2$. The optimization 
problem, then, becomes:


\begin{align}
(\hat{b_1}, \hat{b_2}, \hat{a_1}, \hat{a_2})=\underset{b_1, b_2, a_1, a_2}{argmin}\sum_{i=0}^{N}(y(k)-\hat{y}(k))^{2}
\end{align}


Here, $y(k)$ is the output obtained from the system- so it is known. $\hat{y(k)}$ is the estimated output using $y$ the model 
assumed, which can be written as a difference equation:

\begin{align}
\hat{y}(k) = -a_1\hat{y}(k - 1) - a_2\hat{y}(k - 2) + b_1 u(k - d) + b_2 u(k - 1 - d)
\end{align}

The optimization is performed using the optimization routine “optim” of Scilab. Copy the  data file to the folder 
{\tt prbs\_analysis}. Change the Scilab working directory to {\tt prbs\_analysis} folder. Open the file {\tt optimize.sce} 
and put the name of the data file (with extention) in the filename field. Save and run this code and obtain the plot as 
shown in figure 4.3. This code uses the routines {\tt label.sci}, {\tt costfunction.sci} and {\tt second\_order.sci}. This 
code will give optimized values for $a_1, a_2, b_1, b_2$ which can be used to define a second order discrete time transfer 
function as given in equation \ref{DTF}. The results generated after executing optimization routine over the data file 
obtained earlier is shown in figure \ref{prbs-fit} and figure \ref{prbs-model}. The initial values were chosen to be 
[0.5 0.5 0.5 0.5]. The value of $err$ along with the fit obtained has to be used to change the initial values and the value 
of $delay$. The transfer function thus obtained is 

\begin{align}\label{model}
G(z)=\frac{0.0057384 - 0.0057355 z^{-1}}{1.9529968z^{-1}+0.9162750z^{-2}}z^{-4}
\end{align}



\begin{figure}
\centering
\includegraphics[width=0.7\linewidth]{prbs/prbs-fit.png}
\caption{PRBS testing response}
\label{prbs-fit}
\end{figure}

\begin{figure}
\centering
\includegraphics[width=0.7\linewidth]{prbs/prbs-model.png}
\caption{PRBS testing response}
\label{prbs-model}
\end{figure}



\section{Performing PRBS testing on SBHS, locally}
The step by step procedure for conducting an experiment locally remains same as explained in section \ref{local-steps} with the following changes
\begin{itemize}
\item The working folder is prbs/identification/
\item Use the getd command getd ../../common\_files
\item In section \ref{local-steps} read all instances of step\_test as prbstest
\end{itemize}
The steps to identify the transfer remains same as explained in the section \ref{prbs-model}

\section{Implementing 2DOF pole-placement controller using PRBS model, virtually}

For deriving the Two degrees of freedom control law, please refer to the chapter \ref{2dof}
The controller was designed for the given transient conditions, rise time = 10 sec, overshoot = 0.1. The experimental result and performance of the controller for setpoint temperature change from 38.00 to 43.00 degree C, i.e. 5 degree C positive step change, has been shown below in Fig \ref{2dof-controller}

\begin{figure}
\centering
\includegraphics[width=0.9\linewidth]{prbs/prbs-2dof-controller.png}
\caption{2dof controller response}
\label{2dof-controller}
\end{figure}

 The parameters for the 2-DOF pole-placement controller obtained are shown here
\begin{align*}
Tc &= 1 - 1.9444137 z^{ -1} + 0.9447818 z^{ -2}\\
Sc &= 0.0337719 - 0.0656666z^{ -1}+ 0.0319071z^ { -2}\\
Rc &= 10^{-9} (4377900 - 12034140 z^{ -1} + 11094713 z^ {-2} - 3436740.5 z^{-3} + 3.469D^{-09} z ^{-4} \\&- 147850.06 z^ {-5} + 146117.57 z^{ -6} )\\
\gamma &= 0.0337719
\end{align*}

As can be observed from the graph of temperature vs. time (third subplot) in Fig \ref{2dof-controller}, the overshoot criteria was satisfied very easily. The rise time criteria is observed to be more than 30 sec. This can be satisfied with experimentation.
The paramenters are computed by the file {\tt twodof\_para.sce}.  

The steps to be followed to conduct PRBS test experiment virtually remains same as explained in section \ref{vlabsexpt}. only for the following differences
\begin{itemize}
\item The working folder is prbs/controller/
\item Use the getd command getd ../../common\_files
\item In section \ref{vlabsexpt} read all instances of step\_test as prbstest
\end{itemize}

\section{Implementing 2DOF pole-placement controller using PRBS model, locally}
The step by step procedure for conducting an experiment locally remains same as explained in section \ref{local-steps} with the following changes
\begin{itemize}
\item The working folder is prbs/controller/
\item Use the getd command getd ../../common\_files
\item In section \ref{vlabsexpt} read all instances of step\_test as prbstest
\end{itemize}

\section{Scilab Local codes}\label{prbs-local-codes}
\subsection{Identification codes}
\begin{code}
\ccaption{ser\_init.sce}{\ttfamily ser\_init.sce}
\lstinputlisting{Scilab/local/prbs/identification/ser_init.sce}
\end{code}

\begin{code}
\ccaption{costfunction.sci}{\ttfamily costfunction.sci}
\lstinputlisting{Scilab/local/prbs/identification/costfunction.sci}
\end{code}

\begin{code}
\ccaption{optimize.sce}{\ttfamily optimize.sce}
\lstinputlisting{Scilab/local/prbs/identification/optimize.sce}
\end{code}

\begin{code}
\ccaption{prbs.sci}{\ttfamily prbs.sci}
\lstinputlisting{Scilab/local/prbs/identification/prbs.sci}
\end{code}

\begin{code}
\ccaption{prbstest.sci}{\ttfamily prbstest.sci}
\lstinputlisting{Scilab/local/prbs/identification/prbstest.sci}
\end{code}

\begin{code}
\ccaption{second\_order.sci}{\ttfamily second\_order.sci}
\lstinputlisting{Scilab/local/prbs/identification/second_order.sci}
\end{code}

\begin{code}
\ccaption{start.sce}{\ttfamily start.sce}
\lstinputlisting{Scilab/local/prbs/identification/start.sce}
\end{code}

\subsection{Controller codes}

\begin{code}
\ccaption{prbs.sce}{\ttfamily start.sce}
\lstinputlisting{Scilab/local/prbs/controller/prbs.sce}
\end{code}

\begin{code}
\ccaption{prbs\_pp.sci}{\ttfamily prbs\_pp.sce}
\lstinputlisting{Scilab/local/prbs/controller/prbs_pp.sci}
\end{code}

\begin{code}
\ccaption{ser\_init.sce}{\ttfamily ser\_init.sce}
\lstinputlisting{Scilab/local/prbs/controller/ser_init.sce}
\end{code}

\begin{code}
\ccaption{start.sce}{\ttfamily start.sce}
\lstinputlisting{Scilab/local/prbs/controller/start.sce}
\end{code}

\begin{code}
\ccaption{twodof\_para.sce}{\ttfamily twodof\_para.sce}
\lstinputlisting{Scilab/local/prbs/controller/twodof_para.sce}
\end{code}



\section{Scilab Virtual codes}\label{prbs-virtual-codes}

\subsection{Identification codes}


\begin{code}
\ccaption{costfunction.sci}{\ttfamily costfunction.sci}
\lstinputlisting{Scilab/virtual/prbs/identification/costfunction.sci}
\end{code}


\begin{code}
\ccaption{optimize.sce}{\ttfamily optimize.sce}
\lstinputlisting{Scilab/virtual/prbs/identification/optimize.sce}
\end{code}

\begin{code}
\ccaption{prbs.sci}{\ttfamily prbs.sci}
\lstinputlisting{Scilab/virtual/prbs/identification/prbs.sci}
\end{code}

\begin{code}
\ccaption{prbstest.sci}{\ttfamily prbstest.sci}
\lstinputlisting{Scilab/virtual/prbs/identification/prbstest.sci}
\end{code}

\begin{code}
\ccaption{prbstest.sce}{\ttfamily prbstest.sce}
\lstinputlisting{Scilab/virtual/prbs/identification/prbstest.sce}
\end{code}

\begin{code}
\ccaption{second\_order.sci}{\ttfamily second\_order.sci}
\lstinputlisting{Scilab/virtual/prbs/identification/second_order.sci}
\end{code}

\subsection{Controller codes}

\begin{code}
\ccaption{prbs.sce}{\ttfamily prbs.sce}
\lstinputlisting{Scilab/virtual/prbs/controller/prbs.sce}
\end{code}

\begin{code}
\ccaption{prbscontrol-virtual.sci}{\ttfamily prbscontrol-virtual.sci}
\lstinputlisting{Scilab/virtual/prbs/controller/prbscontrol-virtual.sci}
\end{code}


\begin{code}
\ccaption{twodof\_para.sce}{\ttfamily twodof\_para.sce}
\lstinputlisting{Scilab/virtual/prbs/controller/twodof_para.sce}
\end{code}





%\bibliography{imc} 


%\input{pid_manual/PID}
%\chapter{Implementing \textquoteleft Two Degrees of Freedom\textquoteright Controller for First order systems on a
Single Board Heater System}
The aim of this experiment is to implement a 2DOF controller on a
single board heater system.  The target group is anyone who has basic
knowledge of Control Engineering.
\begin{figure}
\centering
\includegraphics[width=0.9\linewidth]{2-DOF_manual/2dof_xcos.png}
\caption{Xcos interface for this experiment}
\label{Xcos_2dof}
\end{figure}
We have used Scilab with Xcos as an interface for sending and receiving data. This interface is shown in Fig.\ref{Xcos_2dof}. Fan speed and Heater current are the two inputs to the system. For this experiment, the heater current is used as a control effort generated by inputting the various 2-DOF controller parameters like Rc, Sc, Tc and gamma. The fan input could be thought of as an external disturbance.
\section{Theory}
 Degree of freedom as far as the control theory is concerned is the number of parameters on which the plant is no more dependent or the number of parameters that are free to vary. This means that a higher degree of freedom controller makes the plant less susceptible to disturbances. 
Controllers are broadly classified as feedback and feed forward controllers. Feedback controllers are further classified as One Degree of Freedom controller and Two Degree of Freedom controller. Feed forward controllers are those who take the control action before a disturbance disturbs the plant. But this implies an ability to sense the disturbance. Moreover, exact knowledge about the plant is also needed. Nevertheless, due to these restrictions, it is rarely used alone.
A feedback control strategy is as shown in figure \ref{fb}. The reference and the output is continuously compared to generate error which is fed to the controller to take the appropriate control action. Here, exact knowledge about the plant, $G(z)$ and the disturbance, $v$ is not necessary.
\begin{figure}
\begin{center}
\begin{tikzpicture}[auto, node distance=2cm]
\node[input, name=input]{};
\node[sum,right of=input](sum){};
\draw[->](input) -- node{$r$}(sum);
\node[block, right of=sum](gc){$G_c(z)$};
\draw[->](sum)--node{$e$}(gc);
\node[block, right of=gc, node distance=3cm](g){$G(z)$};
\draw[->](gc)--node{$u$}(g);
\node[sum, right of=g,pin={[pinstyle] above:$v$},node distance=2cm](sum2){};
\draw[->](g)--node{}(sum2);
\node[output,right of=sum2,node distance=1cm](output){};
\draw[->](sum2)--node[name=y]{$y$}(output);
\draw [->] (y)--(8.58,-1.5)-|(sum)node[pos=0.88] {$-$};
\end{tikzpicture}
\end{center}
\caption{Feed back control strategy}
\label{fb}
\end{figure}

Solving for y(n), we get
\begin{align}
y(n)&=\frac{G(z)G_c(z)}{1+G(z)G_c(z)}r(n)+\frac 1{1+G(z)G_c(z)}v(n)
\intertext{let,}
T(z)&=\frac{G(z)G_c(z)}{1+G(z)G_c(z)}\\
S(z)&=\frac 1{1+G(z)G_c(z)}
\intertext{this implies}
y(n)&=T(z)r(n)+S(z)v(n)
\intertext{Here it could be seen that the controller has to track the reference input as well as eliminate the effect of external disturbance. But, however from the above equation it could be seen that}
S + T &= 1
\end{align}
Hence it is not possible to achieve both of the requirements, simultaneously in this particular control arrangement. This control arrangement is called {\ttfamily One Degree of Freedom}, abbreviated as 1-DOF.
A {\ttfamily Two Degrees of Freedom} strategy is as shown in figure \ref{2dof}. Here, $G_b$ and $G_f$ together constitute the controller. $G_b$ is in the feedback path and is used to eliminate the effect of disturbances, whereas,$G_f$ is in the feed forward path and is used to help the output track reference input. We need a control law something of the form,
\begin{figure}
\begin{center}
\begin{tikzpicture}[auto,node distance=2cm]
\node[input,name=input](input){};
\node[block,right of=input](gf){$G_f$};
\draw[->](input) -- node{$r$}(gf);
\node[sum,right of=gf,node distance=2cm](sum1){};
\draw[->](gf)--node{}(sum1);
\node[block,right of=sum1](g){$G$};
\draw[->](sum1)--node{$u$}(g);
\node[sum,right of=g,node distance=2cm](sum2){};
\draw[->](g)--node{}(sum2);
\node[output, right of=sum2,node distance=1cm](output){};
\draw[->](sum2)--node[name=y]{$y$}(output){};
\node[block, above of=sum2](h){$H$};
\draw[->](h)--node{$v$}(sum2);
\draw[->](8.015,3.8)--node{$d$}(h);
\node[block, below of=g](gb){$G_b$};
\draw[->](y)|-(gb);
\draw[->](gb)-|node[pos=0.99]{$-$}(sum1);
\end{tikzpicture}
\end{center}
\caption{2DOF feed back control strategy}
\label{2dof}
\end{figure}

\begin{align}
R_c(z)u(n)&=T_c(z)r(n)-S_c(z)y(n)\label{desired}
\intertext{The terms $R_c$, $S_c$ and $T_c$ are all in polynomials of $z^{-1}$.}
\intertext{It could be seen that,}
G_b &= \frac{S_c}{R_c}
\intertext{and}
G_f &= \frac{T_c}{R_c}\\
\intertext{Consider a plant with model}
A(z)y(n)&=z^{-k}B(z)u(n)+v(n)\label{model}
\intertext{Substituting equation \ref{desired} in equation \ref{model}, we get}
Ay(n)&=z^{-k}\frac{B}{R_c}\bigg[T_cr(n)-S_cy(n)\bigg]+v(n)
\intertext{solving for $y(n)$, we get}
\bigg(\frac{R_cA+z^{-k}BS_c}{R_c}\bigg)y(n)&=z^{-k}\frac{BT_c}{R_c}r(n)+v(n)
\intertext{This can also be written as}
y(n)&=z^{-k}\frac{BT_c}{\phi _{cl}}r(n)+\frac{R_c}{\phi _{cl}}v(n)
\intertext{where}
\phi _{cl}&=R_c(z)A(z)+z^{-k}B(z)S_c(z)
\end{align}
and is known as the closed-loop characteristic polynomial.

Now, we want the following conditions to be satisfied.
\begin{enumerate}
\item The zeros of $\phi _{cl}$ should be inside the unit circle, so that the closed-loop system becomes stable. 
\item The value of $z^{-k}\frac{BT_c}{\phi _{cl}}$ must be close to unity so that reference tracking is achieved 
\item The value of $\frac{R_c}{\phi _{cl}}$ must be as small as possible to achieve disturbance rejection
\end{enumerate}
We would now see the pole placement controller approach to design a 2DOF controller.\cite{kmmdc09}

\section{Designing 2-DOF controller using pole placement control approach}
A 2DOF pole placement controller is as shown in the figure \ref{2dofppc}
\begin{figure}
\begin{center}
\begin{tikzpicture}[auto,node distance=2cm]
\node[input](input){};
\node[block,right of=input](tcbyrc){$\gamma \dfrac{T_c(z)}{R_c(z)}$};
\draw[->](input)--node{$r$}(tcbyrc);
\node[sum,right of=tcbyrc,node distance=2cm](sum1){};
\draw[->](tcbyrc)--node{}(sum1);
\node[block,right of=sum1](plant){$G=z^{-k}\dfrac{B(z)}{A(z)}$};
\draw[->](sum1)--node{$u$}(plant);
\node[output,right of=plant](output){};
\draw[->](plant)--node[name=y]{$y$}(output);
\node[block,below of=plant](scbyrc){$\dfrac{S_c(z)}{R_c(z)}$};
\draw[->](y)|-node{}(scbyrc);
\draw[->](scbyrc)-|node[pos=0.99]{$-$}(sum1);
\end{tikzpicture}
\end{center}
\caption{2-DOF pole placement controller}
\label{2dofppc}
\end{figure}

It should be noted that the effect of external disturbance will not be considered for this section.
We want the closed loop transfer function to behave in such a way so that the output $y$ is related to the setpoint $r$ in the following manner
\begin{align}
Y_m(z)&=\gamma z^{-k}\frac{B_r}{\phi_{cl}}R(z)\label{modeloutput}
\intertext{Here, $Y_m(z)$ means the model output. $\phi_{cl}$ is nothing but the closed loop characteristic polynomial obtained by the desired location analysis.}
\intertext{The value of gamma is chosen in such a way so that at steady-state the output of the model is equal to the setpoint.}
\gamma&=\frac{\phi_{cl(1)}}{B_r(1)}
\intertext{Simplifying the block diagram shown in figure \ref{2dofppc} yields}
Y&=\gamma z^{-k}\frac{BT_c}{AR_c+z^{-k}BS_c}R\label{blkdigoutput}
\intertext{Here we have dropped the argument of $z$ for convenience}
\intertext{On comparing equation \ref{modeloutput} and \ref{blkdigoutput} we can see that}
\frac{BT_c}{AR_c+z^{-k}BS_c}&=\frac{B_r}{\phi_{cl}}\label{comparison}
\intertext{Here after factorization of the LHS we can expect some cancellations between the numerator and the denominator  thereby making the $deg B_r < deg B$. But the cancellations ,if any, must be between $stable$ poles and zeros. One should avoid the cancellation of an unstable pole with a zero.}
\intertext{Hence, we differentiate the factors as $good$ and $bad$ factors. Therefore we write $A$ and $B$ as }
A&=A^gA^b\\
B&=B^gB^b
\intertext{We also split $R_c,S_c$ and $T_c$ as shown}
R_c&=B^gR_1\\
S_c&=A^gS_1\\
T_c&=A^gT_1
\intertext{Hence, the equation \ref{comparison} becomes}
\frac{B^gB^bA^gT_1}{A^gA^bB^gR_1+z^{-k}B^gB^bA^gS_1}&=\frac{B_r}{\phi_{cl}}
\intertext{After appropriate cancellations, we obtain}
\frac{B^bT_1}{A^bR_1+z^{-k}B^bS_1}&=\frac{B_r}{\phi_{cl}}\label{aftercancelation}
\intertext{Equating the LHS and RHS of equation \ref{aftercancelation} we obtain}
B^bT_1&=B_r\label{Br}\\
A^bR_1+z^{-k}B^bS_1&=\phi_{cl}\label{aryabhatta}
\intertext{Equation \ref{aryabhatta} is known as the aryabhatta's identity and can be used to solve for $R_1$ and $S_1$. There are many options to choose for the value of $T_1$. By choosing $T_1$ to be equal to $S_1$ the 2-DOF controller is reduced to 1-DOF controller. We usually choose $T_1$=1.}
\intertext{Equation \ref{Br} becomes}
B^b&=B_r
\intertext{hence the expression of gamma is now changed to}
\gamma&=\frac{\phi_{cl(1)}}{B^b(1)}
\intertext{and the desired closed loop transfer function now becomes}
Y_m(z)&=\gamma z^{-k}\frac{B^b}{\phi_{cl}}R(z)
\end{align}
This implies that the open loop model imposes two limitations on the closed loop model.
\begin{itemize}
\item The bad portion of the open loop model cannot be canceled out and it appears in the closed loop model. 
\item The open loop plant delay cannot be removed or minimized,i.e. the closed loop model cannot be made faster then the open loop model.  
\end{itemize}
\section{Step by step procedure to design and implement a 2-DOF controller}
We obtain a first order transfer function of the plant using the step test approach.The model so obtained is
\begin{align}
G(s)&=\frac{0.42}{35.61s+1}
\intertext{with time constant $\tau = 35.6 sec$ and gain $K=0.42$}
\intertext{After discretization with sampling time = 1 second, we obtain}
G(z)&=\frac{0.0116304}{z-0.9723086}\\
&=\frac{0.0116304z^{-1}}{1-0.9723086z^{-1}}
\end{align}
%Discretization can be done using the scilab code {\ttfamily c2d.sce}.
We would now define good and bad terms
\begin{align*}
A^g&=1-0.9723086z^{-1}\\
A^b&=1\\
B^{g}&=0.0116304\\
B^{b}&=1
\intertext{Let us now define the transient specifications. We choose,}
\text{Rise time} &=100 \text{ seconds}
\intertext{No. of samples per rise time ($N_r$) is calculated as}
N_r&\le\frac{\text{Rise time}}{\text{Sampling time}}\\
&=100
\intertext{next}
\omega&=\frac{\pi}{2N_r}\\
&=0.015708
\intertext{We choose,}
Overshoot(\epsilon)&=0.05.........i.e 5\%\\
\rho&\le \epsilon ^{\omega / \pi}\\
&=0.860
\intertext{Let us now calculate 2DOF Controller parameters.
The closed loop characteristic polynomial is given by}
\phi _{cl}&= 1-z^{-1}2\rho cos\omega + \rho ^2z^{-2}\\
&=1-1.7198065z^{-1}+0.7396z^{-2}
\intertext{But according to equation \ref{aryabhatta}}
A^bR_1+z^{-k}B^bS_1&=\phi_{cl}
\intertext{Recall that we had not considered external disturbance in the block diagram shown in figure\ref{2dofppc}. However, we can still, up to some extent, take care of the disturbances. This is achieved by using the internal model principle. If a model of step is present inside the loop, step disturbances can be rejected. We can apply this by forcing $R_c$ to have this term. A step model is given by}
1(z)&=\frac{1}{1-z^{-1}}
\intertext{Let the denominator of the step model be denoted as $\Delta$}
\Delta &= 1-z^{-1}
\intertext{Therefore,}
R_c&=B^g\Delta R_1
\intertext{$\Delta$ has a root which lies on the unit circle. Hence it has to be treated as a bad part and should not be canceled out. Hence, we should make sure that all of the occurrences of $R_1$ have this term.}
\end{align*}
Therefore,
\begin{align}
\phi_{cl}&=A^b\Delta R_1+z^{-k}B^bS_1
\end{align}
Hence,
\begin{align*}
A^b\Delta R_1+z^{-k}B^bS_1&=1-1.7198065z^{-1}+0.7396z^{-2}
\end{align*}
The expression is known as the Aryabhatta Identity and is solved using rigorous Matrix calculations. The explanation of this operation is not considered here. You may refer to the book "Digital Control" by Prof. Kannan Moudgalya \cite{kmmdc09} 
%\intertext{The expression, however, does not satisfy the conditions required for solving the Aryabhatta Identity.} 
%\intertext{Let,}
%\begin{align*}
%R_1&=1-0.7396z^{-1}
%\intertext{therefore}
%S_1&=0.0198\\
%R_c&=B^g\Delta R_1
%\intertext{therefore}
%R_c&=0.0116304-0.0229175z^{-1}+0.0112871z^{-2}\\
%S_c&=A^gS_1
%\intertext{hence}
%S_c&=0.0004641-0.0004512z^{-1}\\
%T_c&=A^gT_1
%\intertext{therefore}
%T_c&=1-0.9723z^{-1}\\
%\gamma&=\frac{\phi_{cl(1)}}{B^b(1)}\\
%&=0.0004641
%\intertext{$\phi_{cl(1)}$ means for $z=1$, steady-state. So, we get}
\begin{align*}
R_c&=R_{c1}+R_{c2}z^{-1}+R_{c3}z^{-2}\\
&=0.0116304-0.0229175z^{-1}+0.0112871z^{-2}\\
S_c&=S_{c1}+S_{c2}z^{-1}\\
&=0.0004641-0.0004512z^{-1}\\
T_c&=T_{c1}+T_{c2}z^{-1}\\
&=1-0.9723z^{-1}\\
\gamma&=0.0004641
\end{align*}

Scilab code {\ttfamily twodof\_para.sce} does these calculations.  This code utilizes various other scilab codes provided at the end of this document. Execute this scilab code with the first order transfer function for your SBHS. You would obtain a Z-Transformed transfer function for the continuous time transfer function you input. You would also obtain the various parameters of 2dof controller as shown in figure \ref{2-DOF_para}
\begin{figure}
\centering
\includegraphics[width=0.8\linewidth]{2-DOF_manual/2dof_console}
\caption{Scilab output for \ttfamily 2DOF\_para.sce}
\label{2-DOF_para}
\end{figure}
\footnote{ NOTE:- The scilab codes are given at the end of this document.}
After execution of {\ttfamily twodof\_para.sce}, run the Xcos code {\ttfamily twodof.xcos} with required setpoint value and observe the temperature profile. The performance of the controller is shown in figure \ref{rt_127}Make sure that you input the sampling time(Clock period) same as the one you used for discretization of the continuous time plant transfer function.
\begin{figure}
\centering
\includegraphics[width=\linewidth]{2-DOF_manual/2dof_resp.png}
\caption{Implementation of 2DOF controller}
\label{rt_127}
\end{figure}
It could be seen that the output (temperature) tracks the setpoint irrespective of the step changes in the fan speed.
We can see that the Over shoot turns out to be 6\% and rise time turns out to be 60 seconds, which is acceptable.

 To implement a second order transfer function, input the correct second order transfer function in {\tt twodof\_para.sce}. Also, make sure you comment the first order control law equation and uncomment the second order control law equation in {\tt twodof.sci} file.

\subsection{Implementing 2dof controller on SBHS, virtually}
The step by step procedure for conducting an experiment virtually is explained in section \ref{vlabsexpt}. The required .sce file is twodof.sce. You will find this file in the {\tt 2dof\_controller} directory under virtual folder. The necessary code is listed in the section \ref{2dofcode}


\section{Scilab Code}\label{2dofcode}
\begin{code}
  \ccaption{c2d.sce}{\ttfamily c2d.sce}
\lstinputlisting{2-DOF_manual/2dof/c2d.sce}
\end{code}
\begin{code}
\ccaption{2-DOF\_para.sce}{\ttfamily 2-DOF\_para.sce}
\lstinputlisting{2-DOF_manual/2dof/2-DOF_para.sce}
\end{code}
\begin{code}
  \ccaption{2dof.sci }{\ttfamily 2dof.sci}
\lstinputlisting{2-DOF_manual/2dof/2dof.sci}
\end{code}
\begin{code}
  \ccaption{cindep.sci}{\ttfamily cindep.sci}
\lstinputlisting{2-DOF_manual/2dof/cindep.sci}
\end{code}
\begin{code}
  \ccaption{clcoef.sci}{\ttfamily clcoef.sci}
\lstinputlisting{2-DOF_manual/2dof/clcoef.sci}
\end{code}
\begin{code}
 \ccaption{clcoef.sci}{\ttfamily clcoef.sci}
\lstinputlisting{2-DOF_manual/2dof/clcoef.sci}
\end{code}
\begin{code}
  \ccaption{cosfil\_ip.sci}{\ttfamily cosfil\_ip.sci}
\lstinputlisting{2-DOF_manual/2dof/cosfil_ip.sci}
\end{code}
\begin{code}
  \ccaption{desired.sci}{\ttfamily desired.sci}
\lstinputlisting{2-DOF_manual/2dof/desired.sci}
\end{code}
\begin{code}
  \ccaption{indep.sci}{\ttfamily indep.sci}
\lstinputlisting{2-DOF_manual/2dof/indep.sci}
\end{code}
\begin{code}
  \ccaption{left\_prm.sci}{\ttfamily left\_prm.sci}
\lstinputlisting{2-DOF_manual/2dof/left_prm.sci}
\end{code}
\begin{code}
  \ccaption{makezero.sci}{\ttfamily makezero.sci}
\lstinputlisting{2-DOF_manual/2dof/makezero.sci}
\end{code}
\begin{code}
 \ccaption{move\_sci.sci}{\ttfamily move\_sci.sci}
\lstinputlisting{2-DOF_manual/2dof/move_sci.sci}
\end{code}
\begin{code}
 \ccaption{polmul.sci}{\ttfamily polmul.sci}
\lstinputlisting{2-DOF_manual/2dof/polmul.sci}
\end{code}
\begin{code}
 \ccaption{polsize.sci}{\ttfamily polsize.sci}
\lstinputlisting{2-DOF_manual/2dof/polsize.sci}
\end{code}
\begin{code}
  \ccaption{polsplit3.sci}{\ttfamily polsplit3.sci}
\lstinputlisting{2-DOF_manual/2dof/polsplit3.sci}
\end{code}
\begin{code}
  \ccaption{polyno.sci}{\ttfamily polyno.sci}
\lstinputlisting{2-DOF_manual/2dof/polyno.sci}
\end{code}
\begin{code}
 \ccaption{pp\_im.sci}{\ttfamily pp\_im.sci}
\lstinputlisting{2-DOF_manual/2dof/pp_im.sci}
\end{code}
\begin{code}
  \ccaption{rowjoin.sci}{\ttfamily rowjoin.sci}
\lstinputlisting{2-DOF_manual/2dof/rowjoin.sci}
\end{code}
\begin{code}
  \ccaption{seshft.sci}{\ttfamily seshft.sci}
\lstinputlisting{2-DOF_manual/2dof/seshft.sci}
\end{code}
\begin{code}
  \ccaption{t1calc.sci}{\ttfamily t1calc.sci}
\lstinputlisting{2-DOF_manual/2dof/t1calc.sci}
\end{code}
\begin{code}
 \ccaption{xdync.sci}{\ttfamily xdync.sci}
\lstinputlisting{2-DOF_manual/2dof/xdync.sci}
\end{code}
\begin{code}
 \ccaption{zpowk.sci}{\ttfamily zpowk.sci}
\lstinputlisting{2-DOF_manual/2dof/zpowk.sci}
\end{code}
%\bibliography{2-DOF}


%\chapter{Implementing Internal Model Controller for first order systems}
The aim of this experiment is to implement an Internal Model Controller(IMC) for first order systems on a single board heater system. The target group is anyone who has a basic knowledge of Control Engineering.
\begin{figure}
	\centering
		\includegraphics[width=\linewidth]{IMC/imc_xcos.png}
	\caption{Xcos interface for this experiment}
	\label{Xcos_imc}
\end{figure}
Scilab with Xcos has been used as an interface for sending and receiving data. This interface is shown in Fig.\ref{Xcos_imc}.Fan speed and Heater current are the two inputs to the system. For this experiment, the heater current is used as a control effort and the fan input could be thought of as an external disturbance.

\section{IMC Design for Single Board Heater System}
Internal Model Controller contains the explicit model of the plant as its part, hence it is named as Internal Model Controller \cite{kmm09}. 
If the open loop transfer function and controller are stable, the closed loop system can be stabilized. The IMC is generally used for stable plants.\\
\begin{figure}
\begin{center}
\begin{tikzpicture}[auto,node distance=2cm]
\node[input](input){};
\node[sum,right of=input](sum1){};
\draw[->](input)--node{$r$}(sum1);
\node[block,right of=sum1](gqz){$G_Q(z)$};
\draw[->](sum1)--node{$e$}(gqz);
\node[branch,right of=gqz, node distance=2cm](b1){};
\node[block, right of=b1](gpz){$G_p(z)$};
\draw[->](gqz)--node{$u$}(gpz);
\node[sum, right of=gpz,pin={[pinstyle] above:$\xi$},node distance=2cm](sum2){};
\draw[->](gpz)--(sum2);
\node[output, right of=sum2](output){};
\draw[->](sum2)--node{$y$}(output);
\node[branch, right of=sum2, node distance=1cm](b2){};
\node[sum, below of=b2,node distance=2cm](sum3){};
\draw[->](b2)--(sum3);
\node[block,below of=gpz](gz){$G(z)$};
\draw[->](b1)|-(gz);
\draw[->](gz)--node{$\bar{y}$\hspace{0.7cm} $-$}(sum3);
\draw[->](sum3)--(10,-3)-|node[pos=0.88]{$-$}(sum1);
\end{tikzpicture}
\end{center}
\caption{IMC feedback configuration}
\label{imcfeedback}
\end{figure}

Let the transfer function of the stable plant be denoted by $G_p (z)$ and it's model be denoted by $G(z)$.Then,
\begin{align}
y(n)=G(z)u(n)+\xi(n) 
\end{align}
where; \\
y(n)=plant output;\\
u(n)=plant input;\\
$\xi$(n)=noise.
      
Fig.\ref{imcfeedback} shows the block diagram representation of an IMC.
For noise rejection ($\xi$=0) and no plant-model mismatch($G=G_p$), $G_Q=G_p^{-1}$ i.e. for stable $G_Q$ an approximate inverse of G is required.Also, for internal stability, transfer function between any two points in the feedback loop must be stable.\cite{kmmdc09}
\begin{figure}
\begin{center}
\begin{tikzpicture}[auto,node distance=2cm]
\node[input](input){};
\node[sum,right of=input](sum1){};
\draw[->](input)--node{$r$}(sum1);
\node[sum, right of=sum1](sum2){};
\draw[->](sum1)--(sum2);
\node[block,right of=sum2](gqz){$G_Q(z)$};
\draw[->](sum2)--node{$e$}(gqz);
\node[branch,right of=gqz, node distance=2cm](b1){};
\node[block, right of=b1](gpz){$G_p(z)$};
\draw[->](gqz)--node{$u$}(gpz);
\node[sum, right of=gpz,pin={[pinstyle] above:$\xi$},node distance=2cm](sum3){};
\draw[->](gpz)--(sum3);
\node[output, right of=sum3](output){};
\draw[->](sum3)--node{$y$}(output);
\node[block, below of=gqz](gz){$G(z)$};
\draw[->](b1)|-(gz);
\draw[->](gz)-|(sum2);
\draw[->](sum3)--(10,-3)-|node[pos=0.88]{$\bar{\xi}$}(sum1);
\end{tikzpicture}
\end{center}
\caption{IMC feedback configuration}
\label{feedback}
\end{figure}

\section{IMC designing of a stable plant}
The plant model must be delay free for $G_Q$ to be realizable. For non-minimum phase part of the plant, reciprocal polynomial is used to get a stable controller. The negative real part of the plant should be replaced with the steady state equivalent of that part to avoid oscillatory nature of the control effort. Low pass filter must be used to avoid the high frequency components because of model mismatch.
Thus IMC designing means obtaining a realizable $G_Q$ that is stable and approximately inverse of G. \\
To illustrate how an IMC is modeled, consider a SBHS whose model is given by,
\begin{align}
	G&=Z^{-1} \frac{0.01163}{1-0.9723Z^{-1}}
\intertext{Inverting delay free plant, We get}
\frac{A}{B}&=\frac{1-0.9723Z^{-1}}{0.01163}
\intertext{Now, Comparing plant model with equation,}
G&=Z^{{-1}}\frac{B^g B^- B^{nm+}}{A}\\
B^g&=0.01163\\
B^-&=1\\
B^{nm+}&=1\\
A&=1-0.9723Z^{-1}
\intertext{For a stable system internal model controller is give by}
G_Q&=\frac{A}{B^gB^-_s B_r^{nm+}}G_f\\
G_Q&=\frac{1-0.9723Z^{-1}}{0.01163}\frac{1-\alpha}{1-\alpha Z^{-1}}
\intertext{Now,}
G_c&=\frac{G_Q}{1-GG_Q}\\
\frac{u}{e}&=\frac{\frac{1-0.9723Z^{-1}}{0.01163}\frac{1-\alpha}{1-\alpha Z^{-1}}}{1-Z^{-1}\frac{0.01163}{1-0.9723Z^{-1}}\frac{1-0.9723Z^{-1}}{0.01163}\frac{1-\alpha}{1-\alpha Z^{-1}}}\\
\intertext{On simplification}
\frac{u}{e}&=\frac{1-\alpha}{0.01163}\frac{1-0.9723Z^{-1}}{1-Z^{-1}}\\
\frac{u}{e}&=b\frac{1-0.9723Z^{-1}}{1-Z^{-1}}
\intertext{Where,}\\
b&=\frac{1-\alpha}{0.01163}
\intertext{Hence,}
u(n)&=u(n-1)+b[e(n)-0.9723e(n-1)]
\end{align}
which shows the change in value of the manipulated variable by the controller.

\section{Implementing IMC on SBHS}
\subsection{Locally}
The scilab code {\tt imc.sci} file,used for implementing the IMC is listed at the end of the chapter. To implement the IMC locally, first, change the current working directory to {\tt imc\_controller}. Next, execute the file {\tt ser\_init.sce} with the appropriate com port. Refer to chapter 2 in case of any doubts.Execute the file {\tt imc.sci} ,which pops up an xcos diagram similar to fig.\ref{fig:0.991}. Run this xcos file. After this the experiment begins and  a figure showing three plots appears.The  first subplot shows setpoint and output temperature profile. The second sub plot shows the control effort and the third subplot shows error between setpoint and plant output. \\
\begin{figure}[h]
	\centering
		\includegraphics[width=\linewidth]{IMC/imc_092_resp.png}
	\caption{Experimental Results with IMC for $\alpha=0.92$}
	\label{fig:0.991}
\end{figure}[t]
\begin{figure}
	\centering
		\includegraphics[width=\linewidth]{IMC/imc_085_resp.png}
		\caption{Experimental Results with IMC for $\alpha=0.85$}
	\label{fig:0.98}
\end{figure}
\\By comparing above two graph we can say that for $\alpha=0.92$ the response of the controller is sluggish. For $\alpha=0.85$ the controller starts responding quickly and no overshoots are seen in the temperature profile.


\subsection{Virtually}
The step by step procedure for conducting an experiment virtually is explained in section \ref{vlabsexpt}. The required .sce file is {\tt imc\_virtual.sce}.  You will find this file in the {\tt imc\_controller} directory under {\tt virtual} folder. The necessary codes are listed in the section \ref{imccodes}


\section{Scilab Code}\label{imccodes}
\begin{code}
\ccaption{ser\_init.sce}{\ttfamily ser\_init.sce}
\lstinputlisting{Scilab/local/imc_controller/ser_init.sce}
\end{code}

\begin{code}
 \ccaption{imc.sci}{\ttfamily imc.sci}
\lstinputlisting{Scilab/local/imc_controller/imc.sci}
\end{code}


\begin{code}
 \ccaption{imc\_virtual.sce}{\ttfamily imc\_virtual.sce}
\lstinputlisting{Scilab/virtual/imc_controller/imc_virtual.sce}
\end{code}


\begin{code}
 \ccaption{imc\_virtual.sci}{\ttfamily imc\_virtual.sci}
\lstinputlisting{Scilab/virtual/imc_controller/imc_virtual.sci}
\end{code}


%\bibliography{imc} 


%
\chapter{Design and Implementation of Self Tuning PI and PID Controllers}

\section{Introduction}%\label{hints}
This chapter presents design and implementation of self tuning PI and PID Controllers on Single Board Heater System done by Mr. Vikas Gayasen.\footnote{Copyright: Mr. Vikas Gayasen, student of Prof. Kannan Moudgalya, IIT Bombay for process control course, 2010}
When a plant is wired in a close loop with a PID controller, the parameters, $K_c$, $\tau_i$ and $\tau_d$ determine the variation of the manipulated input that is given by the controller. This, in turn, determines the variation of the controlled variable, when a set point is given. Suitable values of these parameters can be found out when plant transfer function is known. However, with large changes in the controlled variable, there may be appreciable changes in the plant transfer function itself. Therefore, it is required to dynamically update the controller parameters according to the transfer function.





\section{Theory}
\subsection{Why a Self Tuning Controller?}
The transfer function of SBHS is assumed as 


\begin{align}
	%\Delta$T = frac {K_p}{(\tau$_p$s+1)}\Delta$H + frac {K_f}{(\tau$_f&s+1)}\Delta$F
\Delta T = \frac {K_p}{\tau_ps+1} \Delta H + \frac {K_f}{\tau_fs+1} \Delta F 
\end{align}
 
$\Delta$T: Temperature Change

$\Delta$F: Fan Input Change

$\Delta$H: Heater Input Change\\

The values of $K_p$, $K_f$, $\tau_s$ and $\tau_f$ can be found by conducting step test experiments. Using these values, the parameters ($K_c$,  $\tau_i$ and  $\tau_d$) of the PID controller can be defined using methods like Direct Synthesis of Ziegler Nichols Tuning.
However, when the apparatus is used in over a large range of temperature, the values of the plant parameters ($K_p$, $K_f$, $\tau_s$ and $\tau_f$) may change. The new values would give new values of PID controller parameters. However, in a conventional PID controlled system, the parameters $K_c$,$\tau_i$ and $\tau_d$ are defined beforehand and are not changed when the system is working. This may lead to a situation where the PID controller is working with unsuitable values that may not give the desired performance.Therefore, it becomes necessary to update the values of the PID parameters so that the plant gives optimum performance.

\subsection{Self tuning controller design procedure}
The variable description is as follows:
\begin{itemize}
	\item  Manipulated Variable: Heater Input
	\item  Disturbance Variable: Fan Input
	\item  Controlled Variable: Temperature
\end{itemize}

Perrform several open loop step test experiments (giving step changes in the heater input) and note the values of $K_p$ and $\tau_p$  from the results for each experiment by fitting the inverse laplace transform of the assumed transfer function with the experimental data. Plot these values with respect to the corresponding average temperatures. From these plots, correlations for both $K_p$ and $\tau_p$ as functions of temperature can be found. 
From correlations of  $K_p$ and $\tau_p$, the PID parameters could be found as functions of temperature. Thus, in the new PID controller, the values of $K_c$, $\tau_i$ and $\tau_d$ are calculated using the temperature of the system. For the calculation of PID settings, two approaches: Direct Synthesis and Ziegler-Nichols Tuning can be followed.



\subsubsection{Direct synthesis}

\begin{figure}[h]
	\centering
		\includegraphics[scale = 20,width = 1\linewidth]{Vikas_self/report_tex/Closed Loop Circuit.jpg}
	\caption{Closed Loop Circuit}
\end{figure}

\begin{align}
V(s) = \frac {G_c(s) G(s)}{1+G_c(s) G(s)}
\end{align}

Where\\
V(s) : Overall closed-loop transfer function\\
$G_c$(s) : Controller transfer function\\
G(s) : System transfer function.\\ \\\\
Therefore,

\begin{align*}
G_c(s) = \frac 1{G(s)} \frac {V(s)}{1-V(s)}
\end{align*}
Let the desired closed loop transfer function be of form
\begin{align}
V(s)=\frac 1{(\tau_{cl}s+1)}\\
G(s)=\frac {K_p}{(\tau\_p s+1)}
\end{align}
By using the equations for G(s) and V(s), we get:

\begin{align}
G(c)=K_c(1 + \frac {1}{\tau s})
\end{align}

Where,\\
$K_c = \frac 1{K_p} (\tau_p / \tau_{cl} )$\\
$\tau_i = \tau_p$ \\
\text{When $K_p$ and $\tau_p$ are known as a function of time, the values of $K_c$ and $\tau_i$ can be found as function of temperature as well.}
\subsubsection{Ziegler Nichols Tuning}
For the Ziegler Nichols Tuning, the step response of the open loop experiment is used.

\begin{figure}[h]
	\centering
\includegraphics[width = 0.7\linewidth]{Vikas_self/report_tex/ziegler.jpg}
	\caption{Tangent Approach to Ziegler Nichols Tuning}
	\label{ziegler}
\end{figure}

%\begin{figure}[h]
%\centering
%	\includegraphics[scale = .5,width=0.50\linewidth]{pidziegler.jpg}
%	\label{fig:pidziegler}
%\end{figure}

\begin{table}[h]
	\centering
	\begin{tabular}{|l||c|c|c|}\hline
		  & $K_c$ & $\tau_i$ & $\tau_d$ \\ \hline \hline
		P & 1/RL & & \\ \hline
		PI & 0.9/RL & 3L& \\ \hline
		PID & 1.2/RL & 2L & .5L\\ \hline
	\end{tabular}
	\caption{Ziegler Nichols PID Settings}
	\label{ziegler}
\end{table}


Table \ref{ziegler} gives the PID settings. In this approach too, for every open loop step test, K and $\tau$ are found and correlated as function of average temperature and PID settings are then found as functions of temperature.
\\The reader must note that for a first order transfer function the following is assumed,
\begin{itemize}
	\item $K_p$ $\approx$ K 
	\item $\tau_p$ $\approx$ $\tau$
\end{itemize}

\section{Step Test Experiments and Parmeter Estimation}
This section illustrates the procedure to determine the controller paraemters both, for a conventional controller and a self tuning controller. To design a conventional controller, only a single step test is sufficient. For designing a self tuning controller, several open loop step test experiments are carried out and the values of the open loop parameters are found by curve fitting. These parameters in turn are used to design a self tuning controller.The procedure is described with the help of an example below.
\subsection{Step Test Experiments}
As an example, the results of three open loop experiments is presented.


\begin{table}[h]
	\begin{tabular}{|c|c|c|c|c|}\hline
	Initial Heater Reading&Final Heater Reading&Average Temperature($^0$C)&$K_p$&$\tau_p$\\ \hline \hline
	10	&15	&31.57	&0.41	&53.37\\ \hline
	20	&25	&36.00	&0.50	&52.64\\ \hline
	30	&35	&41.79	&0.58	&49.21\\ \hline
		
	\end{tabular}
	\caption{Open Loop Parameters}
	\label{tab:OpenLoopParameters}
\end{table}


\subsection{Conventional Controller Design}
The controller can be designed conventionally by using a single open loop experiment data by Direct Sybthesis, Ziegler Nichols Tuning methods. A few cases have been illustrated below.
\begin{enumerate}
	\item PI Controller using Ziegler Nichols Tuning with the results of the first step test experiment: 
\begin{itemize}
	\item Kc  = 19.75 
	\item $\tau_i$ = 18

\end{itemize}


	\item PID Controller using Ziegler Nichols Tuning with the results of the first step test experiment: 
\begin{itemize}
	\item Kc  = 26.327 
	\item $\tau_i$ = 12
	\item $\tau_d$ = 3

\end{itemize}


	\item PI Controller Using Direct Synthesis on the results of the second step test experiment ($\tau_{cl}$ is taken as $\tau_p$/2):
\begin{itemize}
	\item Kc  = 4.02
	\item $\tau_i$ = 52.645

\end{itemize}


\end{enumerate}
\subsection{Self Tuning Controller Design}
\label{selftuningdesign}
The functions relating the values of  Kp and $\tau_p$ as a function of temperature are determined by curve fitting.These functions, in turn are used for designing self tuning controllers.

\begin{figure}[h]
\centering
	\includegraphics[width = \textwidth]{Vikas_self/report_tex/parameter_estimation/kp.jpg}
		\caption{Variation of $K_p$ with temperature}
	\label{kp}
\end{figure}

\begin{figure}
\centering
	\includegraphics[width = \textwidth]{Vikas_self/report_tex/parameter_estimation/taup.jpg}
		\caption{Variation of $\tau_p$ with temperature}
	\label{taup}
\end{figure}
\newpage
\begin{enumerate}
	\item \textbf{PI Controller using Ziegler Nichols Tuning:} \\
	
	L = 6\\
	R = (0.016$\times$T-0.114)/(66.90-0.415$\times$T) where T is the temperature\\
	Kc = 0.9(66.90-0.415T)/6(0.016T-0.114) \\
		 = (60.21 - 0.3735T)/(0.096T - 0.684)\\
	% 0.9$\times$(66.90-0.415$\times$T)/6$\times$(0.016$\times$T-0.114)\\
	$\tau_i$ = 3 $\times$ 6 = 18\\

	\item \textbf{PID Controller using Ziegler Nichols Tuning:} \\
	
	L = 6\\
	R = (0.016 $\times$ T-0.114)/(66.90-0.415 $\times$ T) where T is the temperature\\
	K = 1.2(66.90-0.415T)/6(0.016T-0.114)\\
		= (80.28 - 0.498T)/(0.096T - 0.684)\\
	%1.2$\times$(66.90-0.415 $\times$ T)/6$\times$(0.016 $\times$ T-0.114)\\
	$\tau_i$ = 2 $\times$ 6 = 12\\
	$\tau_d$ = 0.5$\times$6 = 3\\

	\item \textbf{PI Controller using Direct Synthesis ($\tau_{cl}$ is taken as $\tau_p$/2):}\\

	K = 2/(0.016$\times$T-0.114)\\
	$\tau_i$ = (66.90-0.415$\times$T) where T is the temperature\\

\end{enumerate}


\section{Implementation}
\subsection{PI Controller}
\begin{figure}[h]
\centering
	\includegraphics[ width =0.7\textwidth]{Vikas_self/report_tex/implementation/Pi_dist_xcos.jpg}
		\caption{Xcos Diagram for PI Controller}
	\label{PI}
\end{figure}

The PI Controller in Continuous Time is given by:
\begin{align*}
	u(t) = K \left[e(t) + \frac 1{\tau_i}\int_0^t e(t)dt\right]
\end{align*}
On taking Laplace Transform, we obtain:
\begin{align*}
	u(s) = K \left[1 + \frac 1{\tau_i s}\right]e(s)
\end{align*}
By mapping the above to discrete time interval using Backward Difference Approximation
\begin{align*}
	u(n) = K \left[1 + \frac{T_s}{\tau_i} \frac{z}{z-1}\right]e(n)
\end{align*}
On Cross Multiplication, we obtain:
\begin{align*}
	(z-1)\times u(n) = K \left[(z-1) + \frac{T_s}{\tau_i} (z)\right]e(n)
\end{align*}
We devide by z, and using the shifting theorem, we obtain:
\begin{align*}
	 u(n) - u(n-1) = K \left[e(n) - e(n-1) + \frac{T_s}{\tau_i} e(n)\right]
\end{align*}
The PI Controller is usually written as:
\begin{align}
	 u(n) = u(n-1) + s_0 e(n) + s_1 e(n-1)
\end{align}
Where,
\begin{align*}
s_0 &= K\left(1+ \frac{T_s}{\tau_i}\right)\\
s_1 &= -K
\end{align*}



\subsection{PID Controller}
\begin{figure}[h]
\centering
	\includegraphics[width =0.7\textwidth]{Vikas_self/report_tex/implementation/pid_dist.png}
		\caption{Xcos Diagram for PID Controller}
	\label{PID}
\end{figure}

The PID Controller in Continuous Time is given by:
\begin{align*}
	u(t) = K \left[e(t) + \frac 1{\tau_i}\int_0^t e(t)dt + \tau_d \frac{de(t)}{dt}\right]
\end{align*}
On taking Laplace Transform, we obtain:
\begin{align*}
	u(s) = K \left[1 + \frac 1{\tau_i s} + \tau_d s\right]e(s)
\end{align*}
By mapping the above to discrete time interval by using the Trapezoidal Approximation for integral mode and Backward Difference Approximation for Derivative mode
\begin{align*}
	u(n) = K \left[1 + \frac{T_s}{\tau_i} \frac{z}{z-1} + \frac{\tau_d}{T_s} \frac{z-1}{z}\right]e(n)
\end{align*}
On Cross Multiplication, we obtain:
\begin{align*}
	(z^2-z)\times u(n) = K \left[(z^2-z) + \frac{T_s}{\tau_i} (z^2) + \frac{\tau_d}{T_s} (z-1)^2\right]e(n)
\end{align*}
We devide by z, and using the shifting theorem, we obtain:
\begin{align*}
	 u(n) - u(n-1) = K \left[e(n) - e(n-1) + \frac{T_s}{\tau_i} e(n) + \frac{\tau_d}{T_s}\left\{e(n) - 2e(n-1) + e(n-1)\right\}\right]
\end{align*}
The PID Controller is usually written as:
\begin{align}
	 u(n) = u(n-1) + s_0 e(n) + s_1 e(n-1) + s_2 e(n-2)
\end{align}
Where,
\begin{align*}
s_0 &= K\left(1+ \frac{T_s}{\tau_i} + \frac{\tau_d}{T_s}\right)\\
s_1 &= K\left[-1 - 2\frac{\tau_d}{T_s}\right]\\
s_2 &= K\left[\frac{\tau_d}{T_s}\right]
\end{align*}

\subsection{Self Tuning Controller}
\begin{figure}[h]
\centering
	\includegraphics[width = \textwidth]{Vikas_self/report_tex/implementation/pi_dist_self.png}
		\caption{Xcos Diagram for Self Tuning Controller}
	\label{selftuning}
\end{figure}

The parameters of the Controller are determined dynamically using the temperature values during every sampling time. For this, the formulae derived in section \ref{selftuningdesign} are used. The formulae for the control effort are same as the conventional PI and PID controllers. So the PI/PID settings are calculated for every sampling time and the control effort is calculated thereafter using the formulae derived for conventional controllers.
%
%
%
	
\section{Set Point Tracking and Disturbance Rejection}
Once a controller is designed, its effectiveness is checked by its ability to track the set point and reject disturbance. In the following sections the set point tracking and disturbance rejection experiments are carried out for both conventional and self tuning controllers.  

\subsection{Set Point Tracking}
The main aim of the controller is to track the set point and to reject disturbances. When the set point of the controlled variable (temperature in this case) is changed, the controller should work in such a manner that the actual temperature follows the set point as close as possible.\\

In this project, several experiments were conducted with the self tuning and conventional PI/PID Controllers. Table \ref{spt} shows the set point changes given during the various experiments that were conducted with conventional and self tuning controllers designed using several methods.
\begin{table}[h]
	\centering
		\begin{tabular}{||c|c|c|}\hline
			&Conventional Controller&Self Tuning Controller\\\hline \hline
		Direct Synthesis PI&32$^0$C to $37^0$C&32$^0$C to 37$^0$C\\
											 &35$^0$C to 45$^0$C&35$^0$C to 45$^0$C\\\hline
		Ziegler Nichols PI&32$^0$C to 37$^0$C&32$^0$C to 37$^0$C\\
												&35$^0$C to 45$^0$C&35$^0$C to 45$^0$C\\
												&40$^0$C to 45$^0$C&35$^0$C to 45$^0$C\\\hline
		Ziegler Nichols PID&31$^0$C to 45$^0$C&32$^0$C to 46$^0$C\\
												&32$^0$C to 37$^0$C&32$^0$C to 37$^0$C\\\hline
		\end{tabular}
	\caption{Set Point Changes in experiments conducted for Set Point Tracking}
	\label{spt}
\end{table}
\newpage
\subsubsection{PI Controller designed by Direct Synthesis}
The results of the experiments carried out for the self tuning PI controller using direct synthesis method are shown. The upper plot shows the variations of the set point temperature (the black line) and the actual temperature (the green line) in the SBHS. The lower plot shows the control effort.
\begin{figure}[h]
	\centering
\includegraphics[width=0.7\textwidth]{Vikas_self/report_tex/PID_results/self_tuning/NewSetpoint_change/DirectSynthesis/step32to37.jpg}
	\caption{Result for Self Tuning Controller designed using Direct Synthesis for Set Point going from 32$^0$C to 37$^0$C}
	\label{fig:step32to37}
\end{figure}

Although there is a small overshoot, the controller is able to make the actual temperature follow the set point temperature quiet closely. Looking at higher values of set point changes, the result for set point change going from 35$^0$C to 45$^0$C is shown.
\begin{figure}[h]
	\centering
\includegraphics[width=0.7\textwidth]{Vikas_self/report_tex/PID_results/self_tuning/NewSetpoint_change/DirectSynthesis/step35to45.jpg}
	\caption{Result for Self Tuning Controller designed using Direct Synthesis for Set Point going from 35$^0$C to 45$^0$C }
	\label{fig:step35to45}
\end{figure}

For a higher set point change also, the controller is able to make the temperature follow the set point closely. Notice the abrupt change in the control effort as soon as the step change in the set point is encountered.\\For comparison, results of experiments done with conventional PI controller designed using the Direct Synthesis method are also shown.

\begin{figure}[h]
	\centering
\includegraphics[width=0.7\textwidth]{Vikas_self/report_tex/PID_results/Conventional_Tuning/Setpointchange/Direct_Synthesis/step32to37.jpg}
	\caption{Result for Conventional Controller designed using Direct Synthesis for Set Point going from 32$^0$C to 37$^0$C }
	%\label{fig:step32to37}
\end{figure}

\begin{figure}[h]
	\centering
\includegraphics[width=.7\textwidth]{Vikas_self/report_tex/PID_results/Conventional_Tuning/Setpointchange/Direct_Synthesis/step35to45.jpg}
	\caption{Result for Conventional Controller designed using Direct Synthesis for Set Point going from 35$^0$C to 45$^0$C }
\end{figure}
As can been seen from the graph, the self tuning controller stabilised the temperature faster.
\newpage


%PI ZN
\subsubsection{PI Controller using Ziegler Nichols Tuning}
The results of the of the experiments carried out for the self tuning PI controller using Ziegler Nichols tuning method are shown. The upper plot shows the variations of the set point temperature (the black line) and the actual temperature (the green line) in the SBHS. The lower plot shows the control effort.
\begin{figure}[h]
	
		\centering
\includegraphics[width=0.7\textwidth]{Vikas_self/report_tex/PID_results/self_tuning/NewSetpoint_change/PI/step32to37.jpg}
		\caption{Result for Self Tuning Controller designed using Ziegler Nichols Tuning for Set Point going from 32$^0$C to 37$^0$C}
\end{figure}

Although there are oscillations, the temperature remains near the set point. The result for a higher value of set point change is also shown.
\begin{figure}[h]
	\centering
\includegraphics[width=0.7\textwidth]{Vikas_self/report_tex/PID_results/self_tuning/NewSetpoint_change/PI/step35to45.jpg}
		\caption{Result for Self Tuning Controller designed using Ziegler Nichols Tuning for Set Point going from 35$^0$C to 45$^0$C}

	
\end{figure}

For this experiment, the controller is able to make the temperature follow the set point closely. The fluctuations may be due to noises and the surrounding conditions.The plot for result of an experiment with another value of set point change is also shown.

\begin{figure}[h]
\centering
	\includegraphics[ width=0.7\textwidth]{Vikas_self/report_tex/PID_results/self_tuning/NewSetpoint_change/PI/step40to45.jpg}
		\caption{Result for Self Tuning Controller designed using Ziegler Nichols Tuning for Set Point going from 40$^0$C to 45$^0$C}
\end{figure}
In this experiment too, the controller is able to keep the temperature close to the set point and it stabilises fast.\\
For comparison, results of experiments done with conventional PI controller designed using the Ziegler Nichols method are also shown.

\begin{figure}[h]
	\centering	\includegraphics[width=0.7\textwidth]{Vikas_self/report_tex/PID_results/Conventional_Tuning/Setpointchange/PI/step32to37.jpg}
	\caption{Result for Conventional Controller designed using Ziegler Nichols Tuning for Set Point going from 32$^0$C to 37$^0$C }
\end{figure}
\newpage
\begin{figure}[h]
	\centering	\includegraphics[width=0.7\textwidth]{Vikas_self/report_tex/PID_results/Conventional_Tuning/Setpointchange/PI/step35to45.jpg}
	\caption{Result for Conventional Controller designed using Ziegler Nichols Tuning for Set Point going from 35$^0$C to 45$^0$C }
\end{figure}

\begin{figure}[h]
		\centering
\includegraphics[width=0.7\textwidth]{Vikas_self/report_tex/PID_results/Conventional_Tuning/Setpointchange/PI/step40to45.jpg}
	\caption{Result for Conventional Controller designed using Ziegler Nichols Tuning for Set Point going from 40$^0$C to 45$^0$C }
\end{figure}

For set point change from 40$^0$C to 45$^0$C, the self tuning controller showed small oscillations, but the conventional controller shows bigger oscillations.
\newpage


%pid zn
\subsubsection{PID Controller using Ziegler Nichols Tuning}\label{pidzn}
The results of the of the experiments carried out for the self tuning PID controller using Ziegler Nichols tuning method are shown. The upper plot shows the variations of the set point temperature (the black line) and the actual temperature (the purple line) in the SBHS. The lower plot shows the control effort.

\begin{figure}[h]
	\centering
\includegraphics[width=0.7\textwidth]{Vikas_self/report_tex/PID_results/self_tuning/NewSetpoint_change/PID/step32to37.jpg}
	\caption{Result for Self Tuning PID Controller designed using Ziegler Nichols Tuning for Set Point going from 32$^0$C to 37$^0$C}
\end{figure}


\begin{figure}[h]
	\centering
\includegraphics[width=0.7\textwidth]{Vikas_self/report_tex/PID_results/self_tuning/NewSetpoint_change/PID/step32to46.jpg}
	\caption{Result for Self Tuning PID Controller designed using Ziegler Nichols Tuning for Set Point going from 32$^0$C to 46$^0$C}
\end{figure}

From the graph it can be seen that for both the above experiments, the self tuning PID controller is able to keep the temperature close to the set point and the stabilisation is also fast. For comparison, plots for experiments conducted with conventional PID controller designed using Ziegler Nichols method are also shown.

\begin{figure}[h]
	\centering
\includegraphics[width=0.7\textwidth]{Vikas_self/report_tex/PID_results/Conventional_Tuning/Setpointchange/PID/step32to37.jpg}
	\caption{Result for Conventional PID Controller designed using Ziegler Nichols Tuning for Set Point going from 32$^0$C to 37$^0$C }
	\label{fig:step31to45}
\end{figure}


\begin{figure}[h]
	\centering
\includegraphics[width=0.7\textwidth]{Vikas_self/report_tex/PID_results/Conventional_Tuning/Setpointchange/PID/step31to45.jpg}
	\caption{Result for Conventional PID Controller designed using Ziegler Nichols Tuning for Set Point going from 31$^0$C to 45$^0$C }
	\label{fig:step31to45}
\end{figure}

From the above graph we can see that the conventional PID controller is not able to make the temperature close to the set point when the set point value is 45$^0$C. The self tuning PID controller had successfully brought the temperature to 45$^0$C.

\subsubsection{Conclusion}
The self tuning PI controller is able to accomplish the aim of keeping the temperature as close as possible to the set point. Although it may show some initial overshoot or oscillation for some values of set point change, the time needed for stabilisation is low. There may also be some cases where the conventional controller shows bigger oscillations than the self tuning controller.\\

The PI controllers, both conventional and self tuning, show oscillations for some values of set point change. To eliminate the oscillations, when we use the PID Controller, the self tuning design definately seems to be a better option because for higher values of set point change, the self tuning PID controller shows a better performance than the conventional controller as seen in section \ref{pidzn}.

\subsection{Disturbance Rejection}

Apart from tracking the set point, the system should also be able to reject disturbances. There may be several factors influencing the controlled variable and not all of them can be manipulated. Therefore, it becomes necessary for the controller not to let the changes in the non-manipulated vraibles to affect the controlled variable. This is called Disturbance Rejection.\\

In this system, the disturbance variable is the fan input. Therefore, the controller has to work in such a way that changes in the fan input doesn't affect the temperature in the SBHS.\\

In this project, several experiments were conducted with the self tuning and conventional PI/PID Controllers. Table \ref{dist} shows the fan input changes given during the various experiments that were conducted with conventional and self tuning controllers designed using several methods.\\
\begin{table}[h]
	\centering
		\begin{tabular}{||c|c|c|}\hline
			&Conventional Controller&Self Tuning Controller\\\hline \hline
		Direct Synthesis PI&50 to 100&50 to 100\\
											 &100 to 50&100 to 50\\\hline
		Ziegler Nichols PI &50 to 100&50 to 100\\
												&100 to 50&100 to 50 \\\hline
		Ziegler Nichols PID&50 to 100&50 to 100\\
												&100 to 50&100 to 50\\\hline
		\end{tabular}
	\caption{Fan Input Changes in experiments conducted for Disturbance Rejection}
	\label{dist}
\end{table}


\subsubsection{PI Controller designed by Direct Synthesis}
The results of the experiments carried out for the self tuning PI controller using direct synthesis method are shown. The upper plot shows the variations of the set point temperature (the black line) and the actual temperature (the green line) in the SBHS. The second plot shows the control effort and the third shows the fan input.

\begin{figure}[h]
	\centering
\includegraphics[width=0.7\linewidth]{Vikas_self/report_tex/PID_results/self_tuning/FanDisturbance/DirectSynthesis/step50to100.jpg}
	\caption{Results for Fan Input Change from 50 to 100 for Self Tuning PI Controller  designed using Direct Synthesis}
\end{figure}
The change in the fan input introduces a small dent in the temperature. However, the controller brings the temperature back to the set point. Notice the slight change in the controller behaviour on encountering the fan input change. The time taken for stabilising back is also low.
\newpage
Here, results for fan input change from 100 to 50 are also shown.
\begin{figure}[h]
	\centering
\includegraphics[width=0.7\linewidth]{Vikas_self/report_tex/PID_results/self_tuning/FanDisturbance/DirectSynthesis/step100to50.jpg}
	\caption{Results for Fan Input Change from 100 to 50 for Self Tuning PI Controller  designed using Direct Synthesis}
\end{figure}

In this figure also, the temperature clearly increses a bit when the step change in the fan input is encountered. However, it quickly stabilises back and continues to be close to the set point.\\

From the above two results, it is clear that the self tuning controller designed with direct syntheis has successfully rejected the disturbance.\\

For comparison, results of the disturbance change for conventional PI Controller designed with direct synthesis are also shown.
\newpage
\begin{figure}[h]
	\centering
\includegraphics[width=.7\linewidth]{Vikas_self/report_tex/PID_results/Conventional_Tuning/Fan_disturbance/Direct_Systhesis/step50to100.jpg}
	\caption{Results for the Fan input change from 50 to 100 to Conventional PI Controller designed using Direct Synthesis}
\end{figure}

\begin{figure}[h]
	\centering
\includegraphics[width=.7\linewidth]{Vikas_self/report_tex/PID_results/Conventional_Tuning/Fan_disturbance/Direct_Systhesis/step100to50.jpg}
	\caption{Results for the Fan input change from 100 to 50 to Conventional PI Controller designed using Direct Synthesis}
\end{figure}

%pi zn
\newpage
\subsubsection{PI Controller using Ziegler Nichols Tuning}
The results of the experiments carried out for the self tuning PI controller using Ziegler Nichols method are shown. The upper plot shows the variations of the set point temperature (the black line) and the actual temperature (the green line) in the SBHS. The second plot shows the control effort and the third shows the fan input.
\begin{figure}[h]
	\centering
\includegraphics[width=0.7\textwidth]{Vikas_self/report_tex/PID_results/self_tuning/FanDisturbance/PI/step50to100final.jpg}
	
\caption{Results for Fan Input change from 50 to 100 given to Self Tuning PI Controller designed using Ziegler Nichols Method}
\end{figure}

Even on encountering the fan input change, the temperature remains close to the set point. Notice the change in the controller behaviour on encountering the fan input change. 
\newpage
Here, result for the fan input going from 100 to 50 is also shown.
\begin{figure}[h]
	\centering
\includegraphics[width=0.7\textwidth]{Vikas_self/report_tex/PID_results/self_tuning/FanDisturbance/PI/step100to50.jpg}
	
\caption{Results for Fan Input change from 100 to 50 given to Self Tuning PI Controller designed using Ziegler Nichols Method}
\end{figure}

Here, a change in the control effort can be noticed. This change has been brought by the PI Controller to keept the temperature close to the set point.\\
From the above two results, it is clear that the self tuning controller designed with direct syntheis has successfully rejected the disturbance.\\
\newpage
For comparison, correspoding results are also shown for Conventional PI Controllers designed using ziegler nichols tuning.

\begin{figure}[h]
	\centering
\includegraphics[width=.75\linewidth]{Vikas_self/report_tex/PID_results/Conventional_Tuning/Fan_disturbance/PI/step50to100.jpg}
	\caption{Results for the Fan input change from 50 to 100 to Conventional PI Controller designed using Ziegler Nichols Tuning}
\end{figure}

\begin{figure}[h]
	\centering
\includegraphics[width=.75\linewidth]{Vikas_self/report_tex/PID_results/Conventional_Tuning/Fan_disturbance/PI/step100to50.jpg}
	\caption{Results for the Fan input change from 100 to 50 to Conventional PI Controller designed using Ziegler Nichols Tuning}
\end{figure}


\newpage
\subsubsection{PID Controller using Ziegler Nichols Tuning}
The results of the experiments carried out for the self tuning PID controller using Ziegler Nichols method are shown. The upper plot shows the variations of the set point temperature (the black line) and the actual temperature (the purple line) in the SBHS. The second plot shows the control effort and the third shows the fan input.

\begin{figure}[h]
	\centering
		\includegraphics[width=0.7\linewidth]{Vikas_self/report_tex/PID_results/self_tuning/FanDisturbance/PID/step50to100.jpg}
\caption{Results for Fan Input change from 50 to 100 given to Self Tuning PID Controller designed using Ziegler Nichols Method}

\end{figure}

In this system also, on encountering the fan input change, the temperature remains close to the set point. Notice the change in the control effort profile when the change in the fan input is given.
\newpage
Here, result for the fan input going from 100 to 50 is also shown.
\begin{figure}[h]
	\centering
		\includegraphics[ width=0.7\linewidth]{Vikas_self/report_tex/PID_results/self_tuning/FanDisturbance/PID/step100to50.jpg}
\caption{Results for Fan Input change from 100 to 50 given to Self Tuning PID Controller designed using Ziegler Nichols Method}

\end{figure}

In this figure also, the temperature clearly increses a bit when the step change in the fan input is encountered. However, it quickly stabilises back and continues to be close to the set point.
\newpage
For comparison, correspoding results are also shown for Conventional PID Controllers designed using ziegler nichols tuning.
\begin{figure}[h]
	\centering
\includegraphics[width=.75\linewidth]{Vikas_self/report_tex/PID_results/Conventional_Tuning/Fan_disturbance/PID/step50to100.jpg}
	\caption{Results for the Fan input change from 50 to 100 to Conventional PID Controller designed using Ziegler Nichols Tuning}
	
\end{figure}

\begin{figure}[h]
	\centering
\includegraphics[width=.75\linewidth]{Vikas_self/report_tex/PID_results/Conventional_Tuning/Fan_disturbance/PID/step100to50.jpg}
	\caption{Results for the Fan input change from 100 to 50 to Conventional PID Controller designed using Ziegler Nichols Tuning}
	
\end{figure}

\subsubsection{Conclusion}
It was observed that the self tuning controller manages to keep the temperature close to the set point temperature, even when the change in fan input is encountered. This shows that it can reject the disturbances quite nicely.

\section{Implementing Self Tuning controller on SBHS}
The set point tracking and disturbance rejection experiments can be carried out on a SBHS both locally as well as virtually.
\subsection{Locally}
The following steps can be used for conducting the experiments:
\begin{enumerate}
	\item Open and run the program for serial communication in scilab. This opens the comm port.
	\item Open and execute the sci file corresponding to the experiment that is being done.These files can be found in the {\tt Self\_tuning\_controller} under the folder named {\tt local}
	\item Load the scicos diagram, ensure that the parameters are correct and run the experiment.
\end{enumerate}


\subsection{Virtually}
The step by step procedure for conducting an experiment virtually is explained in section \ref{vlabsexpt}. The required .sce file is {\tt pi\_bda\_tuned\_dist\_virtual.sce} for example if you want to run the {\tt PI Controller Fan disturbance} experiment.  Under the {\tt virtual} folder there are two folders {\tt Self\_tuning\_controller} and {\tt SelfTuning\_Vikas}. You will find this file in the {\tt PIControllerFandisturbance} directory. 
\section{Scilab Codes}
\subsection{Serial Communication}
\begin{code}
  \ccaption{ser\_init.sce}{\ttfamily ser\_init.sce}
\lstinputlisting{Vikas_self/ser_init.sce}
\end{code}


\subsection{Fan Disturbance in PI Controller}
\begin{code}
  \ccaption{pi\_bda\_dist.sci}{\ttfamily pi\_bda\_dist.sci}
\lstinputlisting{Scilab/local/Self_tuning_controller/ConventionalTuning_Vikas/PIControllerFandisturbance/pi_bda_dist.sci}
\end{code}


\subsubsection{Set Point Change in PI Controller}
\begin{code}
  \ccaption{pi\_bda.sci}{\ttfamily pi\_bda.sci}
\lstinputlisting{Scilab/local/Self_tuning_controller/ConventionalTuning_Vikas/PIControllersetpointchange/pi_bda.sci}
\end{code}



\subsubsection{Fan Disturbance to PID Controller}
\begin{code}
  \ccaption{pid\_bda\_dist.sci}{\ttfamily pid\_bda\_dist.sci}
\lstinputlisting{Scilab/local/Self_tuning_controller/ConventionalTuning_Vikas/PIDControllerFandisturbance/pid_bda_dist.sci}
\end{code}


\subsubsection{Set Point Change in PID Controller}
\begin{code}
  \ccaption{pid\_bda.sci}{\ttfamily pid\_bda.sci}
\lstinputlisting{Scilab/local/Self_tuning_controller/ConventionalTuning_Vikas/PIDControllersetpointchange/pid_bda.sci}
\end{code}



\subsection{Self Tuning Controller, local}\label{selfcode_local}
\subsubsection{Fan Discturbance to PI Controller}
\begin{code}
  \ccaption{pi\_bda\_tuned\_dist.sci}{\ttfamily pi\_bda\_tuned\_dist.sci}
\lstinputlisting{Scilab/local/Self_tuning_controller/SelfTuning_Vikas/PIControllerFandisturbance/pi_bda_tuned_dist.sci}
\end{code}


\subsubsection{Set Point Change to PI Controller}
\begin{code}
  \ccaption{pi\_bda\_tuned.sci}{\ttfamily pi\_bda\_tuned.sci}
\lstinputlisting{Scilab/local/Self_tuning_controller/SelfTuning_Vikas/PIControllersetpointchange/pi_bda_tuned.sci}
\end{code}

\subsubsection{Fan Disturbance to PID Controller}
\begin{code}
  \ccaption{pid\_bda\_tuned\_dist.sci}{\ttfamily pid\_bda\_tuned\_dist.sci}
\lstinputlisting{Scilab/local/Self_tuning_controller/SelfTuning_Vikas/PIDControllerFandisturbance/pid_bda_tuned_dist.sci}
\end{code}

\subsubsection{Set Point Change to PID Controller}
\begin{code}
  \ccaption{pid\_bda\_tuned.sci}{\ttfamily pid\_bda\_tuned.sci}
\lstinputlisting{Scilab/local/Self_tuning_controller/SelfTuning_Vikas/PIDControllersetpointchange/pid_bda_tuned.sci}
\end{code}






\subsection{Conventional Controller, virtual}\label{convcode_virtual}
\subsection{Fan Disturbance in PI Controller}
\begin{code}
  \ccaption{pi\_bda\_dist.sci}{\ttfamily pi\_bda\_dist.sci}
\lstinputlisting{Scilab/virtual/Self_tuning_controller/ConventionalTuning_Vikas/PIControllerFandisturbance/pi_bda_dist_virtual.sci}
\end{code}


\subsubsection{Set Point Change in PI Controller}
\begin{code}
  \ccaption{pi\_bda.sci}{\ttfamily pi\_bda.sci}
\lstinputlisting{Scilab/virtual/Self_tuning_controller/ConventionalTuning_Vikas/PIControllersetpointchange/pi_bda_virtual.sci}
\end{code}



\subsubsection{Fan Disturbance to PID Controller}
\begin{code}
  \ccaption{pid\_bda\_dist.sci}{\ttfamily pid\_bda\_dist.sci}
\lstinputlisting{Scilab/virtual/Self_tuning_controller/ConventionalTuning_Vikas/PIDControllerFandisturbance/pid_bda_dist_virtual.sci}
\end{code}


\subsubsection{Set Point Change in PID Controller}
\begin{code}
  \ccaption{pid\_bda.sci}{\ttfamily pid\_bda.sci}
\lstinputlisting{Scilab/virtual/Self_tuning_controller/ConventionalTuning_Vikas/PIDControllersetpointchange/pid_bda_virtual.sci}
\end{code}



\subsection{Self Tuning Controller, local}\label{selfcode_virtual}
\subsubsection{Fan Discturbance to PI Controller}
\begin{code}
  \ccaption{pi\_bda\_tuned\_dist.sci}{\ttfamily pi\_bda\_tuned\_dist.sci}
\lstinputlisting{Scilab/virtual/Self_tuning_controller/SelfTuning_Vikas/PIControllerFandisturbance/pi_bda_tuned_dist_virtual.sci}
\end{code}


\subsubsection{Set Point Change to PI Controller}
\begin{code}
  \ccaption{pi\_bda\_tuned.sci}{\ttfamily pi\_bda\_tuned.sci}
\lstinputlisting{Scilab/virtual/Self_tuning_controller/SelfTuning_Vikas/PIControllersetpointchange/pi_bda_tuned_virtual.sci}
\end{code}

\subsubsection{Fan Disturbance to PID Controller}
\begin{code}
  \ccaption{pid\_bda\_tuned\_dist.sci}{\ttfamily pid\_bda\_tuned\_dist.sci}
\lstinputlisting{Scilab/virtual/Self_tuning_controller/SelfTuning_Vikas/PIDControllerFandisturbance/pid_bda_tuned_dist_virtual.sci}
\end{code}

\subsubsection{Set Point Change to PID Controller}
\begin{code}
  \ccaption{pid\_bda\_tuned.sci}{\ttfamily pid\_bda\_tuned.sci}
\lstinputlisting{Scilab/virtual/Self_tuning_controller/SelfTuning_Vikas/PIDControllersetpointchange/pid_bda_tuned_virtual.sci}
\end{code}





%% <== End of hints
%%%%%%%%%%%%%%%%%%%%%%%%%%%%%%%%%%%%%%%%%%%%%%%%%%%%%%%%%%%%%



%%%%%%%%%%%%%%%%%%%%%%%%%%%%%%%%%%%%%%%%%%%%%%%%%%%%%%%%%%%%%
%% BIBLIOGRAPHY AND OTHER LISTS
%%%%%%%%%%%%%%%%%%%%%%%%%%%%%%%%%%%%%%%%%%%%%%%%%%%%%%%%%%%%%
%% A small distance to the other stuff in the table of contents (toc)
%\addtocontents{toc}{\protect\vspace*{\baselineskip}}



%% The Bibliography
%% ==> You need a file 'literature.bib' for this.
%% ==> You need to run BibTeX for this (Project | Properties... | Uses BibTeX)
%\addcontentsline{toc}{chapter}{Bibliography} %'Bibliography' into toc
%\nocite{*} %Even non-cited BibTeX-Entries will be shown.
%\bibliographystyle{alpha} %Style of Bibliography: plain / apalike / amsalpha / ...
%\bibliography{literature} %You need a file 'literature.bib' for this.



%%%%%%%%%%%%%%%%%%%%%%%%%%%%%%%%%%%%%%%%%%%%%%%%%%%%%%%%%%%%%
%% APPENDICES
%%%%%%%%%%%%%%%%%%%%%%%%%%%%%%%%%%%%%%%%%%%%%%%%%%%%%%%%%%%%%


%% ==> Write your text here or include other files.
%\input{FileName} %You need a file 'FileName.tex' for this.





%\input{mpc/mpc}



\bibliography{commonbib}

\end{document}


